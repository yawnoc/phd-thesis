\chapter{Curvilinear boundary tracing}
\label{ch:curvilinear}

The actual determination of the sought-after traced boundaries involves
recasting the flux boundary condition~(\ref{eq:flux-boundary-condition})
as a first-order ODE (or system of ODEs) for the traced boundaries.
In practice, the appropriate coordinate system and parametrisation
will depend on the geometry of the specific problem at hand.

Whilst Anderson~\etal~\cite{
  anderson-2007-boundary-tracing-i-theory,
  anderson-2007-boundary-tracing-ii-applications
}
did not restrict themselves to Cartesian coordinates in analytical work,
the boundary tracing ODE was separately derived
for each new coordinate system encountered;
no generalised version was given.
As for numerical boundary tracing using arc-length parametrisation,
Anderson~\cite{anderson-2002-thesis-boundary-tracing-pdes}
only considered Cartesian coordinates.

A generalised version of the boundary tracing ODE is much desired,
and will be most useful given the various coordinate systems
which shall be used in the remainder of this thesis.
I therefore derive in this chapter
the boundary tracing ODE for coordinate parametrisation
and also the corresponding system of ODEs for arc-length parametrisation,
with both applicable in any two-dimensional orthogonal coordinate system.

In the interest of keeping this thesis \emph{wholly self-contained},
I first go through the foundational definitions of
orthogonal curvilinear coordinates and vector calculus
in Sections~\ref{sec:curvilinear.orthogonal}
and~\ref{sec:curvilinear.calculus},
before proceeding to the derivations of the boundary tracing ODEs
in Section~\ref{sec:curvilinear.tracing}.

\section{Orthogonal curvilinear coordinates}
\label{sec:curvilinear.orthogonal}

First I run through some standard definitions and terminology.

Let~$(\basisvec{x}, \basisvec{y})$ be the standard basis vectors
of the usual Cartesian coordinates~$(x, y)$.
The \term{position vector} is
\begin{equation}
  \positionvec \defeq x \basisvec{x} + y \basisvec{y},
  \label{eq:position-vector-cartesian}
\end{equation}
and since $\basisvec{x}$ and~$\basisvec{y}$ are globally constant,
observe that
\begin{align}
  \basisvec{x} &\ideq \pder{\positionvec}{x},
    \label{eq:x-basis-vector} \\[\tallspace]
  \basisvec{y} &\ideq \pder{\positionvec}{y}.
    \label{eq:y-basis-vector}
\end{align}
Now consider the \term{curvilinear coordinates}~$(u, v)$
given by the transformation
\begin{align}
  x &= x (u, v), \label{eq:x-transformation-curvilinear} \\
  y &= y (u, v). \label{eq:y-transformation-curvilinear}
\end{align}
A \term{local basis}~$(\localvec{u}, \localvec{v})$
arises from the derivatives of position
in analogy to~(\ref{eq:x-basis-vector}) and~(\ref{eq:y-basis-vector}),
\begin{align}
  \localvec{u} &
    \defeq \pder{\positionvec}{u}
    \ideq \pder{x}{u} \basisvec{x} + \pder{y}{u} \basisvec{y},
      \label{eq:u-local-basis-vector} \\[\tallspace]
  \localvec{v} &
    \defeq \pder{\positionvec}{v}
    \ideq \pder{x}{v} \basisvec{x} + \pder{y}{v} \basisvec{y}.
      \label{eq:v-local-basis-vector}
\end{align}
It is assumed that the curvilinear coordinates~$(u, v)$ are \term{orthogonal},
meaning the local basis vectors are orthogonal to each other:
\begin{equation}
  \localvec{u} \dotp \localvec{v} \ideq 0.
  \label{eq:local-basis-orthogonal}
\end{equation}
The magnitudes of the local basis vectors may not be unity;
in fact, should either~$u$ or~$v$ not have the dimensions of length,
the corresponding local basis vector will not even be dimensionless.
Therefore, one defines the \term{scale factors}
(or \term{\lame{} coefficients})
\begin{align}
  \scalefac[u] &\defeq \norm{\localvec{u}}, \label{eq:u-scale-factor} \\
  \scalefac[v] &\defeq \norm{\localvec{v}}, \label{eq:v-scale-factor}
\end{align}
and then normalises the local basis
to obtain the \term{local orthonormal basis},
\begin{align}
  \basisvec{u} &\defeq \frac{\localvec{u}}{\scalefac[u]},
      \label{eq:u-basis-vector} \\[\tallspace]
  \basisvec{v} &\defeq \frac{\localvec{v}}{\scalefac[v]},
      \label{eq:v-basis-vector}
\end{align}
for which
\begin{equation}
  \basisvec{u} \dotp \basisvec{v} \ideq 0
  \label{eq:orthonormal-basis-orthogonal}
\end{equation}
and
\begin{equation}
  \norm{\basisvec{u}} \ideq \norm{\basisvec{v}} \ideq 1.
  \label{eq:orthonormal-basis-normalised}
\end{equation}

\begin{figure}
  \centredfigurecontent[width=0.4\textwidth]{%
    orthogonal-curvilinear-coordinates%
  }{
    Infinitesimal rectangle formed by differential displacements
    in the orthogonal curvilinear coordinates~$(u, v)$.
  }
\end{figure}

\section{Vector calculus}
\label{sec:curvilinear.calculus}

Next, I go through the standard mathematical machinery required
to do calculus in an orthogonal coordinate system.

\subsection{Differential displacement}
\label{sec:curvilinear.calculus.displacement}

Consider the $u$-contour and the $v$-contour through a point~$(u, v)$.
Independently incrementing~$u$ by~$\td u$ and~$v$ by~$\td v$
results in two new contours,
corresponding to each of the new coordinates.
Since the coordinate system is orthogonal,
the four contours (two old and two new)
shall mark out an infinitesimal rectangle,
aligned with the local orthonormal basis
and with sides of length~$\scalefac[u] \td u$ and~$\scalefac[v] \td v$
(Figure~\ref{fig:orthogonal-curvilinear-coordinates}).
Indeed the \term{differential displacement} is
\begin{align*}
  \td \positionvec
  &\ideq
    \pder{\positionvec}{u} \td u + \pder{\positionvec}{v} \td v
    \\[\tallspace]
  &\ideq \localvec{u} \td u + \localvec{v} \td v \\
  &\ideq \scalefac[u] \td u \basisvec{u} + \scalefac[v] \td v \basisvec{v}.
    \yesnumber
    \label{eq:differential-displacement}
\end{align*}
The area of the infinitesimal rectangle is
the \term{differential area element}
\begin{equation}
  \td V \ideq \scalefac[u] \scalefac[v] \td u \td v.
  \label{eq:differential-area-element}
\end{equation}

\subsection{Gradient}
\label{sec:curvilinear.calculus.gradient}

Consider a scalar field~$T$
(which, in the context of boundary tracing, shall be the known solution).
By the chain rule, the independent increments~$\td u$ and~$\td v$
effect the differential change
\begin{align*}
  \td T
  &\ideq \pder{T}{u} \td u + \pder{T}{v} \td v \\[\tallspace]
  &\ideq
    \roundbr*{\frac{1}{\scalefac[u]} \pder{T}{u}}
    \roundbr[\big]{\scalefac[u] \td u}
      +
    \roundbr*{\frac{1}{\scalefac[v]} \pder{T}{v}}
    \roundbr[\big]{\scalefac[v] \td v}.
    \yesnumber
    \label{eq:scalar-differential-change}
\end{align*}
By the definition of gradient,
and using~(\ref{eq:differential-displacement}),
\begin{align*}
  \td T
  &\ideq \del T \dotp \td \positionvec \\
  &\ideq
    \del T
      \dotp
    \roundbr[\big]{
      \scalefac[u] \td u \basisvec{u}
        +
      \scalefac[v] \td v \basisvec{v}
    }.
    \yesnumber
    \label{eq:scalar-differential-change-gradient}
\end{align*}
Comparing~(\ref{eq:scalar-differential-change})
and~(\ref{eq:scalar-differential-change-gradient}),
it follows that the \term{gradient} of~$T$ is given by
\begin{equation}
  \del T \ideq
    \frac{1}{\scalefac[u]} \pder{T}{u} \basisvec{u}
      +
    \frac{1}{\scalefac[v]} \pder{T}{v} \basisvec{v}.
  \label{eq:gradient}
\end{equation}

\subsection{Divergence}
\label{sec:curvilinear.calculus.divergence}

Consider a vector field~$%
\vec{E} \defeq \comp{E}{u} \basisvec{u} + \comp{E}{v} \basisvec{v}$,
and the infinitesimal rectangle
of Figure~\ref{fig:orthogonal-curvilinear-coordinates},
whose sides are of length~$\scalefac[u] \td u$ and~$\scalefac[v] \td v$.

To first order,
the net flux across the two edges normal to~$\basisvec{u}$
(which have length~$\scalefac[v] \td v$) is
\[
  \pder{}{u} \roundbr[\Big]{
    \comp{E}{u} \cdot \scalefac[v] \td v
  } \cdot \td u,
\]
and the net flux across the two edges normal to~$\basisvec{v}$
(which have length~$\scalefac[u] \td u$) is
\[
  \pder{}{v} \roundbr[\Big]{
    \comp{E}{v} \cdot \scalefac[u] \td u
  } \cdot \td v.
\]
Summing these results in the flux across the entire perimeter of the rectangle,
which, divided by its area,
the area element~(\ref{eq:differential-area-element}),
yields the \term{divergence} of~$\vec{E}$,
\begin{equation}
  \del \dotp \vec{E} \ideq
    \frac{1}{\scalefac[u] \scalefac[v]}
    \squarebr*{
      \pder{}{u} \roundbr[\Big]{\scalefac[v] \comp{E}{u}}
        +
      \pder{}{v} \roundbr[\Big]{\scalefac[u] \comp{E}{v}}
    }.
  \label{eq:divergence}
\end{equation}

\subsection{Laplacian}
\label{sec:curvilinear.calculus.laplacian}

Composing the divergence~(\ref{eq:divergence})
and the gradient~(\ref{eq:gradient}) (with~$\vec{E} = \del T$),
it follows that the \term{Laplacian} of~$T$ is given by
\begin{equation}
  \del^2 T \defeq \del \dotp \del T \ideq
    \frac{1}{\scalefac[u] \scalefac[v]}
    \squarebr*{
      \pder{}{u} \roundbr*{\frac{\scalefac[v]}{\scalefac[u]} \pder{T}{u}}
        +
      \pder{}{v} \roundbr*{\frac{\scalefac[u]}{\scalefac[v]} \pder{T}{v}}
    }.
  \label{eq:laplacian}
\end{equation}
In particular,
if the two scale factors~$\scalefac[u]$ and~$\scalefac[v]$ are equal,
say~$\scalefac[u] \ideq \scalefac[v] \ideq \scalefac$,
this simplifies to
\begin{equation}
  \del^2 T \ideq
    \frac{1}{\scalefac^2}
    \squarebr*{\pder[2]{T}{u} + \pder[2]{T}{v}}.
  \label{eq:laplacian-scale-factors-equal}
\end{equation}

\subsection{Boundary normal}
\label{sec:curvilinear.calculus.normal}

The \term{tangent vector}~$\tangentvec$ to a curve
is given by the unit vector of
the differential displacement~(\ref{eq:differential-displacement}),
i.e.\@
\begin{equation}
  \tangentvec \ideq
    \frac{
      \scalefac[u] \td u \basisvec{u}
        +
      \scalefac[v] \td v \basisvec{v}
    }{
      \td s
    },
  \label{eq:tangent-vector}
\end{equation}
where $\td s$~is the \term{differential arc length}
\begin{equation}
  \td s \defeq \norm{\td \positionvec} \ideq
  \sqrt{
    \roundbr[\big]{\scalefac[u] \td u}^2
      +
    \roundbr[\big]{\scalefac[v] \td v}^2
  }.
  \label{eq:differential-arc-length}
\end{equation}
The normal vector~$\normalvec$ to a curve
is perpendicular to the tangent~$\tangentvec$;
therefore
\begin{equation}
  \normalvec \ideq
    \frac{
      \scalefac[v] \td v \basisvec{u}
        -
      \scalefac[u] \td u \basisvec{v}
    }{
      \td s
    },
  \label{eq:normal-vector}
\end{equation}
up to sign.

\subsection{Abbreviations}
\label{sec:curvilinear.calculus.abbreviations}

Since the scale factors~$\scalefac[u]$ and~$\scalefac[v]$ appear regularly,
it is helpful to define abbreviations
before proceeding any further.

For the components of the differential displacement
vector~(\ref{eq:differential-displacement}),
which are the side lengths of the infinitesimal rectangle
of Figure~\ref{fig:orthogonal-curvilinear-coordinates},
let
\begin{align}
  \td \mu &\defeq \scalefac[u] \td u,
    \label{eq:differential-displacement-u-component} \\
  \td \nu &\defeq \scalefac[v] \td v,
    \label{eq:differential-displacement-v-component}
\end{align}
so that
\begin{equation}
  \td \positionvec \ideq \td \mu \basisvec{u} + \td \nu \basisvec{v}
  \label{eq:differential-displacement-abbreviated}
\end{equation}
and
\begin{equation}
  \td s \ideq \sqrt{{\td \mu}^2 + {\td \nu}^2}.
  \label{eq:differential-arc-length-abbreviated}
\end{equation}
For the components of the gradient vector~(\ref{eq:gradient}),
write
\begin{align}
  P &\defeq \frac{1}{\scalefac[u]} \pder{T}{u},
    \label{eq:gradient-u-component} \\[\tallspace]
  Q &\defeq \frac{1}{\scalefac[v]} \pder{T}{v},
    \label{eq:gradient-v-component}
\end{align}
so that
\begin{equation}
  \del T \ideq P \basisvec{u} + Q \basisvec{v}.
  \label{eq:gradient-abbreviated}
\end{equation}
Finally, for the components of
the tangent vector~(\ref{eq:tangent-vector}), define
\begin{align}
  \alpha &\defeq \tder{\mu}{s} \ideq \frac{\scalefac[u] \td u}{\td s},
    \label{eq:tangent-u-component} \\[\tallspace]
  \beta &\defeq \tder{\nu}{s} \ideq \frac{\scalefac[v] \td v}{\td s},
    \label{eq:tangent-v-component}
\end{align}
so that
\begin{align}
  \tangentvec
  &\ideq \frac{\td \mu \basisvec{u} + \td \nu \basisvec{v}}{\td s}
  \ideq \alpha \basisvec{u} + \beta \basisvec{v},
    \label{eq:tangent-vector-abbreviated} \\[\tallspace]
  \normalvec
  &\ideq \frac{\td \nu \basisvec{u} - \td \mu \basisvec{v}}{\td s}
  \ideq \beta \basisvec{u} - \alpha \basisvec{v}.
    \label{eq:normal-vector-abbreviated}
\end{align}
Note that
\begin{align}
  P^2 + Q^2 &\ideq (\del T)^2, \label{eq:gradient-pythagoras} \\
  \alpha^2 + \beta^2 &\ideq 1. \label{eq:tangent-pythagoras}
\end{align}

\section{Boundary tracing ODE}
\label{sec:curvilinear.tracing}

With the workings of orthogonal curvilinear coordinates now established,
I derive in this section
the boundary tracing ODE for coordinate parametrisation
and the corresponding system of ODEs for arc-length parametrisation.
Recall that the sought-after traced boundaries are to satisfy
the boundary condition~(\ref{eq:flux-boundary-condition}),
in which $T$~is the chosen known solution (to the PDE under consideration)
and $F$~is the prescribed flux function.

It is convenient to define the \term{viability function}
\begin{important}{equation}
  \Phi \defeq (\del T)^2 - F^2.
  \label{eq:viability-function}
\end{important}
This is not only algebriacally useful,
but also physically meaningful;
the viable domain is the region~$\Phi \ge 0$,
the non-viable domain, $\Phi < 0$,
and the terminal curve, $\Phi = 0$.

\subsection{Coordinate parametrisation}
\label{sec:curvilinear.tracing.coordinate}

Suppose the traced boundaries are to be parametrised
in the form~$v = v (u)$.
Using~(\ref{eq:normal-vector-abbreviated}),
(\ref{eq:differential-arc-length-abbreviated}),
and~(\ref{eq:gradient-abbreviated}),
the boundary condition~(\ref{eq:flux-boundary-condition}) becomes
\[
  \frac{
    \td \nu \basisvec{u} - \td \mu \basisvec{v}
  }{
    \sqrt{{\td \mu}^2 + {\td \nu}^2}
  }
    \dotp
  \roundbr[\big]{P \basisvec{u} + Q \basisvec{v}}
    =
  F,
\]
which expands into the quadratic equation
\begin{equation}
  \roundbr*{P^2 - F^2} \, {\td \nu}^2
  - 2 P Q \td \nu \td \mu
  + \roundbr*{Q^2 - F^2} \, {\td \mu}^2
    =
  0.
  \label{eq:tracing-coordinate-parametrisation-quadratic}
\end{equation}
Solving this for~$\td \nu / {\td \mu}$ results in the boundary tracing ODE
\begin{equation}
  \tder{\nu}{\mu} = \frac{P Q \pm F \sqrt{\Phi}}{P^2 - F^2},
  \label{eq:tracing-ode-coordinate-parametrisation-nu}
\end{equation}
or
\begin{important}{equation}
  \tder{v}{u} =
    \frac{\scalefac[u]}{\scalefac[v]}
      \cdot
    \frac{P Q \pm F \sqrt{\Phi}}{P^2 - F^2}.
  \label{eq:tracing-ode-coordinate-parametrisation-v}
\end{important}
Alternatively, for the parametrisation~$u = u (v)$,
one solves for~$\td \mu / {\td \nu}$, giving
\begin{equation}
  \tder{\mu}{\nu} = \frac{P Q \mp F \sqrt{\Phi}}{Q^2 - F^2},
  \label{eq:tracing-ode-coordinate-parametrisation-mu}
\end{equation}
or
\begin{important}{equation}
  \tder{u}{v} =
    \frac{\scalefac[v]}{\scalefac[u]}
      \cdot
    \frac{P Q \mp F \sqrt{\Phi}}{Q^2 - F^2}.
  \label{eq:tracing-ode-coordinate-parametrisation-u}
\end{important}
In both cases, the two possible choices of sign
correspond to the two branches of traced boundaries
which exist in the viable domain.
Note also the occurrence of the term~$\sqrt{\Phi}$,
corresponding to the absence of traced boundaries
in the non-viable domain~$\Phi < 0$.

Whichever of the symbols~$\pm$ and~$\mp$ occurs,
the \term{upper} branch shall refer to
the traced boundaries for the upper sign in that symbol,
and the \term{lower} branch to those for the lower sign
(e.g.~in~(\ref{eq:tracing-ode-coordinate-parametrisation-u}),
the upper branch corresponds to choosing the minus sign,
while the lower branch corresponds to choosing the plus sign).

\subsection{Arc-length parametrisation}
\label{sec:curvilinear.tracing.arc-length}

In numerical boundary tracing,
the coordinate parametrisations~$v = v (u)$ and~$u = u (v)$
will be problematic
if~$\td v / {\td u}$ or~$\td u / {\td v}$ ever become infinite,
corresponding to traced boundaries
which are parallel to~$\basisvec{v}$ or~$\basisvec{u}$.

To avoid such singularities in numerical work, one should instead use
the arc-length parametrisation~$u = u (s)$, $v = v(s)$
to put the two coordinates~$u$ and~$v$ on an equal footing.
Using~(\ref{eq:normal-vector-abbreviated})
and~(\ref{eq:gradient-abbreviated}),
the boundary condition~(\ref{eq:flux-boundary-condition}) becomes
\[
  \roundbr[\big]{\beta \basisvec{u} - \alpha \basisvec{v}}
    \dotp
  \roundbr[\big]{P \basisvec{u} + Q \basisvec{v}}
    =
  F,
\]
which, together with the unit-speed condition~(\ref{eq:tangent-pythagoras}),
forms the system%
\footnote{
  This system is also given
  in Anderson's thesis~\cite{anderson-2002-thesis-boundary-tracing-pdes},
  but only in Cartesian coordinates.
  Moreover, the system is passed directly to a numerical integrator
  without first solving for~$\alpha$ and~$\beta$,
  so that at every step of the integration,
  a numerical comparison with the previous root
  is required to select the correct root among the two possible.
  A better approach is to separate the two branches analytically
  before undertaking any numerical procedures,
  as I have done here
  in~(\ref{eq:tracing-ode-arc-length-parametrisation-mu})
  and~(\ref{eq:tracing-ode-arc-length-parametrisation-nu}).
}
\begin{align}
  P \beta - Q \alpha &= F,
    \label{eq:tracing-arc-length-parametrisation-flux-boundary-condition} \\
  \alpha^2 + \beta^2 &= 1.
    \label{eq:tracing-arc-length-parametrisation-unit-speed}
\end{align}
Solving~(\ref{eq:tracing-arc-length-parametrisation-flux-boundary-condition})
and~(\ref{eq:tracing-arc-length-parametrisation-unit-speed})
for~$\alpha$ and~$\beta$, one obtains
the boundary tracing system of ODEs under arc-length parametrisation,
\begin{align}
  \tder{\mu}{s} &\ideq \alpha = \frac{-Q F \pm P \sqrt{\Phi}}{(\del T)^2},
    \label{eq:tracing-ode-arc-length-parametrisation-mu} \\[\tallspace]
  \tder{\nu}{s} &\ideq \beta = \frac{\+P F \pm Q \sqrt{\Phi}}{(\del T)^2},
    \label{eq:tracing-ode-arc-length-parametrisation-nu}
\end{align}
or
\begin{important}{align}
  \tder{u}{s} &= \frac{-Q F \pm P \sqrt{\Phi}}{\scalefac[u] (\del T)^2},
    \label{eq:tracing-ode-arc-length-parametrisation-u} \\[\tallspace]
  \tder{v}{s} &= \frac{\+P F \pm Q \sqrt{\Phi}}{\scalefac[v] (\del T)^2}.
    \label{eq:tracing-ode-arc-length-parametrisation-v}
\end{important}
As was the case for coordinate parametrisation
(Section~\ref{sec:curvilinear.tracing.coordinate}),
notice the two possible choices of sign
(which correspond to the two branches of traced boundaries)
and the occurrences of the term~$\sqrt{\Phi}$
(which correspond to the absence of traced boundaries
in the non-viable domain~$\Phi < 0$).

\section{Summary}
\label{sec:curvilinear.summary}

Although the theory of boundary tracing
has not been restricted to Cartesian coordinates in the literature~\cite{
  anderson-2002-thesis-boundary-tracing-pdes,
  anderson-2007-boundary-tracing-i-theory,
  anderson-2007-boundary-tracing-ii-applications
},
the boundary tracing ODE has only been written down
in specific non-Cartesian coordinate systems.
A generalised version is desirable,
and in anticipation of the coordinate systems
which shall be used in the chapters to follow,
I have derived here the boundary tracing ODE for coordinate parametrisation
and the corresponding system of ODEs for arc-length parametrisation.
Both are applicable in any two-dimensional orthogonal coordinate system,
with the former appropriate for analytical work
and the latter useful in numerical boundary tracing.
