\chapter{Polar coordinates}
\label{ch:polar}

In this chapter, I apply boundary tracing to the thermal radiation problem,
using the fundamental line source solution
as the known solution to
Laplace's equation~(\ref{eq:laplace-steady-conduction}).

\section{Line source solution}
\label{sec:polar.line}

Consider the line source solution
to Laplace's equation~(\ref{eq:laplace-steady-conduction}),
which is logarithmic in the cylindrical radius~$r = \sqrt{x^2 + y^2}$.
In the context of steady conduction,
the only dimensionally consistent form for this is
\begin{important}{equation}
  T = T_0 \log \roundbr*{\frac{r_0}{r}}
  \label{eq:line-laplace-solution}
\end{important}
for some temperature~$T_0$ and radius~$r_0$.
Note that the region~$r > r_0$ is unphysical,
since~$T < 0$ therein.

The analysis to follow shall be performed
in the usual polar coordinates~$(r, \phi)$
(see Figure~\tbd),
given by the transformation
\begin{align}
  x &= r \cos\phi, \label{eq:polar-x-transformation} \\
  y &= r \sin\phi. \label{eq:polar-y-transformation}
\end{align}
The local basis vectors are
\begin{align}
  \localvec{r} &=
    \cos\phi \basisvec{x} + \sin\phi \basisvec{y},
    \label{eq:r-local-basis-vector} \\
  \localvec{\phi} &=
    r \roundbr[\big]{-\sin\phi \basisvec{x} + \cos\phi \basisvec{y}},
    \label{eq:phi-local-basis-vector}
\end{align}
the scale factors,
\begin{align}
  \scalefac[r] &= 1, \label{eq:r-scale-factor} \\
  \scalefac[\phi] &= r, \label{eq:phi-scale-factor}
\end{align}
and the local orthonormal basis vectors,
\begin{align}
  \basisvec{r} &= \cos\phi \basisvec{x} + \sin\phi \basisvec{y},
    \label{eq:r-basis-vector} \\
  \basisvec{\phi} &= -\sin\phi \basisvec{x} + \cos\phi \basisvec{y}.
    \label{eq:phi-basis-vector}
\end{align}
