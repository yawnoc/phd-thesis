\chapter{Polar coordinates}
\label{ch:polar}

In this chapter, I apply boundary tracing to the thermal radiation problem,
using the fundamental line-source solution
as the known solution to
Laplace's equation~(\ref{eq:laplace-steady-conduction}).

\section{Line-source solution}
\label{sec:polar.line}

Consider the line-source solution
to Laplace's equation~(\ref{eq:laplace-steady-conduction}),
which is logarithmic in the cylindrical radius~$r = \sqrt{x^2 + y^2}$.
In the context of steady conduction,
the only dimensionally consistent form for this is
\begin{important}{equation}
  T = T_0 \log \roundbr*{\frac{r_0}{r}}
  \label{eq:line-laplace-solution}
\end{important}
for some temperature~$T_0$ and radius~$r_0$.%
\footnote{
  Whereas the power law~$T = C r^n$ is scale-invariant,
  requiring only \emph{one} constant~$C$,
  the logarithm does not possess this property;
  separate constants are required for~$T$ and~$r$.
}
Note that the region~$r > r_0$ is unphysical,
since~$T < 0$ therein.

The analysis to follow shall be performed
in the usual polar coordinates~$(r, \phi)$
(see Figure~\tbd),
given by the transformation
\begin{align}
  x &= r \cos\phi, \label{eq:polar-x-transformation} \\
  y &= r \sin\phi. \label{eq:polar-y-transformation}
\end{align}
The local basis vectors are
\begin{align}
  \localvec{r} &=
    \cos\phi \basisvec{x} + \sin\phi \basisvec{y},
    \label{eq:r-local-basis-vector} \\
  \localvec{\phi} &=
    r \roundbr[\big]{-\sin\phi \basisvec{x} + \cos\phi \basisvec{y}},
    \label{eq:phi-local-basis-vector}
\end{align}
the scale factors,
\begin{align}
  \scalefac[r] &= 1, \label{eq:r-scale-factor} \\
  \scalefac[\phi] &= r, \label{eq:phi-scale-factor}
\end{align}
and the local orthonormal basis vectors,
\begin{align}
  \basisvec{r} &= \cos\phi \basisvec{x} + \sin\phi \basisvec{y},
    \label{eq:r-basis-vector} \\
  \basisvec{\phi} &= -\sin\phi \basisvec{x} + \cos\phi \basisvec{y}.
    \label{eq:phi-basis-vector}
\end{align}

\section{Scaling}
\label{sec:polar.scaling}

While the known solution~(\ref{eq:line-laplace-solution})
already has intrinsic temperature and length scales $T_0$ and~$r_0$,
the final scaling should also account for
the radiation boundary condition~(\ref{eq:radiation-boundary-condition}).

Let
\begin{align}
  \scaled{T} &= T / \tau, \label{eq:line-scaled-temperature} \\
  \scaled{r} &= r / \varrho, \label{eq:line-scaled-r}
\end{align}
with~$\tau$ and~$\varrho$ to be chosen later.
Noting that
\begin{equation}
  \scaleddel = \varrho \del,
  \label{eq:line-scaled-del}
\end{equation}
the radiation boundary condition~(\ref{eq:radiation-boundary-condition})
and the known solution~(\ref{eq:line-laplace-solution})
become
\begin{align}
  \normalvec \dotp \scaleddel \scaled{T}
    &= -\squarebr*{c \varrho \tau^3} \scaled{T}^4,
    \label{eq:line-scaled-radiation-boundary-condition-with-groups}
    \\[\fraclinespace]
  \scaled{T}
    &=
      \squarebr*{\frac{T_0}{\tau}}
      \log \roundbr*{\frac{\squarebr{r_0 / \varrho}}{\scaled{r}}}.
    \label{eq:line-scaled-laplace-solution-with-groups}
\end{align}
There are three dimensionless groups
but only two free scales~$\tau$ and~$\varrho$,
so one of the groups cannot be eliminated.
To keep the logarithmic as simple as possible,
put
\begin{align}
  \tau &= T_0,
    \label{eq:line-temperature-scale} \\
  \varrho &= r_0,
    \label{eq:line-length-scale}
\end{align}
and define the dimensionless group
\begin{equation}
  A = \frac{1}{c r_0 {T_0}^3}.
  \label{eq:line-dimensionless-group}
\end{equation}
Dropping \scalingaccents, the scaled boundary condition~%
  (\ref{eq:line-scaled-radiation-boundary-condition-with-groups})
and known solution~(\ref{eq:line-scaled-laplace-solution-with-groups})
become
\begin{important}{align}
  \normalvec \dotp \del T &= -\frac{T^4}{A},
    \label{eq:line-scaled-radiation-boundary-condition} \\[\fraclinespace]
  T &= -\log r.
    \label{eq:line-scaled-laplace-solution}
\end{important}

\section{Viable domain}
\label{sec:polar.viable}

Comparing the radiation condition~%
  (\ref{eq:line-scaled-radiation-boundary-condition})
to the generic flux condition~(\ref{eq:flux-boundary-condition}),
it follows that the flux function is
\begin{align*}
  F
  &= -\frac{T^4}{A} \\[\fraclinespace]
  &= -\frac{\log^4 r}{A}.
    \yesnumber
    \label{eq:line-flux-function}
\end{align*}
The radial and azimuthal components of the gradient~$\del T$ are
\begin{align}
  P &= \pder{T}{r} = -\frac{1}{r},
    \label{eq:line-gradient-u-component} \\[\fraclinespace]
  Q &= \frac{\pd T}{r \pd\phi} = 0,
    \label{eq:line-gradient-v-component}
\end{align}
and so the viability function is given by
\begin{align*}
  \Phi
  &= (\del T)^2 - F^2 \\[\fraclinespace]
  &= \frac{1}{r^2} - \frac{\log^8 r}{A^2} \\[\fraclinespace]
  &= \frac{A^2 - r^2 \log^8 r}{A^2 r^2} \\[\fraclinespace]
  &= \frac{A^2 - \psi^2}{A^2 r^2},
    \yesnumber
    \label{eq:line-viability-function}
\end{align*}
where $\psi$~is the auxiliary function
\begin{equation}
  \psi (r) \defeq r \log^4 r.
  \label{eq:line-auxiliary-function}
\end{equation}
The viable domain is therefore the region~$\psi (r) \le A$.

\subsection{Auxiliary function properties}
\label{sec:polar.viable.psi}

Since the region~$r > 1$ is unphysical
(corresponding to negative temperature),
it is only necessary to consider~$0 \le r \le 1$.
On this interval, $\psi$~is positive
except at the endpoints~$r = 0$ and~$r = 1$, where it vanishes.
Observing that the slope
\begin{equation}
  \tder{\psi}{r} = (4 + \log r) \log^3 r
  \label{eq:line-psi-derivative}
\end{equation}
changes sign from positive to negative
as $r$~increases through~$\ee^{-4}$,
it follows that $\psi$~has a single maximum on~$0 \le r \le 1$ at
\begin{equation}
  r = r_\nat \defeq \ee^{-4} = 0.01832,
  \label{eq:line-r-natural}
\end{equation}
where it takes the maximal value
\begin{equation}
  \psi
  = A_\nat
  \defeq \psi (r_\nat)
  = (4 / \ee)^4
  = 4.6888.
  \label{eq:line-a-natural}
\end{equation}
This is shown in Figure~\tbd.

\subsection{Three regimes}
\label{sec:polar.viable.regimes}

The topology of the viable domain~$\psi (r) \le A$ will depend
on whether the dimensionless group~$A$ is
greater than, equal to or less than
the critical value~(\ref{eq:line-a-natural}),
as this decides the number of roots of the equation~$\psi (r) = A$:
\begin{enumerate}
  \item
    \emph{Cold regime}, $A > A_\nat$ (Figure~\tbd):
    $\psi (r)$~is never equal to~$A$.
    The entire space~$0 \le r \le 1$ is viable,
    and there is no terminal curve.
  \item
    \emph{Transition}, $A = A_\nat$ (Figure~\tbd):
    $\psi (r)$~is equal to~$A$ at~$r = r_\nat$ only.
    The entire space is still viable,
    but now the terminal curve~$r = r_\nat$ has appeared.
    The terminal curve~$r = r_\nat$ is in fact a critical terminal curve,
    since it is a contour of the known solution~%
      (\ref{eq:line-scaled-laplace-solution}).
  \item
    \emph{Hot regime}, $A < A_\nat$ (Figure~\tbd):
    $\psi (r)$~is equal to~$A$ at~$r = r_\flat$
    and at~$r = r_\sharp$,
    both dependent on~$A$,
    with~$0 <  r_\flat < r_\nat < r_\sharp < 1$.
    The critical terminal curve now consists of
    the two circles~$r = r_\flat$ and~$r = r_\sharp$.
    A non-viable moat~$r_\flat < r < r_\sharp$
    separates an inner viable island~$0 \le r \le r_\flat$
    from an outer viable mainland~$r \ge r_\sharp$.
\end{enumerate}
Figure~\tbd{}
shows the regimes displayed in the style of a bifurcation diagram,
with state space on the vertical axis
and the parameter~$A$ on the horizontal axis.

\section{Boundary tracing}
\label{sec:polar.tracing}

Using~(\ref{eq:r-scale-factor}) and~(\ref{eq:phi-scale-factor})
(which are unchanged by the scaling)
along with~(\ref{eq:line-flux-function})
through~(\ref{eq:line-viability-function}),
the boundary tracing ODE~(\ref{eq:tracing-ode-coordinate-parametrisation-u})
becomes
\begin{important}{equation}
  \tder{r}{\phi} = \mp \frac{\sqrt{A^2 - \psi^2}}{\log^4 r},
  \label{eq:line-tracing-ode-coordinate-parametrisation-r}
\end{important}
so that the traced boundaries are given by
\begin{important}{equation}
  \phi = \mp \int \frac{\log^4 r \td r}{\sqrt{A^2 - \psi^2}}.
  \label{eq:line-traced-boundary-integral}
\end{important}
This integral is not elementary,
and the traced boundaries must be determined numerically.
