\chapter*{Abstract}
\label{ch:abstract}

Boundary tracing is an unconventional method
of tackling flux boundary value problems,
where one takes a known solution to the field equation
and seeks new boundaries satisfying the prescribed boundary condition.
In this thesis, the method of boundary tracing is applied
in the study of two physical phenomena:
thermal radiation and capillarity.

For the first of these,
the Stefan--Boltzmann law gives rise
to a boundary flux which is quartic in temperature.
This nonlinearity renders
even the simplest of conduction--radiation problems
analytically insurmountable in more than one dimension.
By applying boundary tracing to various known solutions of Laplace's equation,
we construct a diverse range of new domains
which exactly admit those known solutions.
Physically reasonable lengths and temperatures can be realised,
and the results obtained are of practical use
in contexts such as space travel,
where thermal radiation is the primary mode of heat transfer.

For the second phenomenon,
we are specifically interested in an industrial dip-coating problem
in which undesirable arching of the coating layer
occurs near the corners of the dipped object.
The capillary problem in downward gravity
consists of the highly nonlinear Laplace--Young equation
and the constant-contact-angle condition.
No exact solutions known in wedge-shaped domains.
We apply a numerical version of boundary tracing
to analyse the effect of corner rounding
on the height rise profile in a wedge.
Through a novel observation,
we are able to produce a one-parameter family of rounding candidates
from a single numerical wedge solution in the convex case.
Previous work is extended
in the areas of corner modification and roughness modelling.
By applying the latter in a re-entrant wedge,
we show that a near-level height rise profile can be obtained
through corner rounding followed by the etching of small grooves.
Thus we have a highly practical method
of achieving a near-level coating profile
in the dip-coating problem.
