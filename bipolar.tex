\chapter{Bipolar coordinates}
\label{ch:bipolar}

In this chapter,
I apply boundary tracing to the conduction--radiation problem
starting from the known solution
to Laplace's equation~(\ref{eq:laplace-steady-conduction})
for equal and opposite line sources.
As before,
I determine traced boundaries along which
the radiation condition~(\ref{eq:radiation-boundary-condition}) holds
and use the convex portions to construct conduction--radiation domains.

\section{Known solution}
\label{sec:bipolar.known}

The problem at hand is best tackled using the bipolar coordinate system.
While it would be well to introduce bipolar coordinates
before writing down the known solution,
a better understanding of the physics is obtained
by first writing down the known solution
and then observing how bipolar coordinates arise.

Suppose that the equal and opposite line sources
are of the same strength as the line source~(\ref{eq:line-laplace-solution})
but located at~$(x, y) = (\pm a, 0)$.
Since the distance to each source is
\begin{equation}
  r_\pm = \sqrt{(x \mp a)^2 + y^2},
  \label{eq:bipolar-source-distances}
\end{equation}
the known solution is given by
\begin{equation}
  T \ideq
    T_0 \log \roundbr*{\frac{r_0}{r_+}}
      -
    T_0 \log \roundbr*{\frac{r_0}{r_-}},
  \label{eq:bipolar-laplace-solution-terms}
\end{equation}
which reduces%
\footnote{
  The remarkable cancellation of the reference length~$r_0$
  occurs only for equal and opposite line sources.
}
to
\begin{important}{equation}
  T \ideq T_0 \log \roundbr*{\frac{r_-}{r_+}}.
  \label{eq:bipolar-laplace-solution-source-distances}
\end{important}
This can be rewritten as
\begin{equation}
  T \ideq T_0 \tanh^{-1} \roundbr*{\frac{2 a x}{x^2 + y^2 + a^2}},
  \label{eq:bipolar-laplace-solution-inverse-tanh}
\end{equation}
and one sees
(by rearranging and completing the square)
that the $T$-contours are circles.
These are not just any circles,
but are in fact the curves marked out
by the second bipolar coordinate~$v$.
