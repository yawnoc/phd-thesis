\chapter{Bipolar coordinates}
\label{ch:bipolar}

In this chapter,
I apply boundary tracing to the conduction--radiation problem
starting from the known solution
to Laplace's equation~(\ref{eq:laplace-steady-conduction})
for equal and opposite line sources.
As before,
I determine traced boundaries along which
the radiation condition~(\ref{eq:radiation-boundary-condition}) holds
and use the convex portions to construct conduction--radiation domains.

\section{Known solution}
\label{sec:bipolar.known}

The problem at hand is best tackled using bipolar coordinates.
While it would be well to introduce the bipolar coordinate system
before writing down the known solution,
a better understanding of the physics is obtained
by first writing down the known solution
and then observing how bipolar coordinates arise as a result.

Suppose that the equal and opposite line sources
are the same strength as the line source~(\ref{eq:line-laplace-solution})
but located at~$(x, y) = (\pm a, 0)$.
Since the distance to each source is
\begin{equation}
  r_\pm = \sqrt{(x \mp a)^2 + y^2},
  \label{eq:bipolar-source-distances}
\end{equation}
the known solution is given by
\begin{equation}
  T \ideq
    T_0 \log \roundbr*{\frac{r_0}{r_+}}
      -
    T_0 \log \roundbr*{\frac{r_0}{r_-}},
  \label{eq:bipolar-laplace-solution-terms}
\end{equation}
which reduces%
\footnote{
  The remarkable cancellation of the reference length~$r_0$
  occurs only for equal and opposite line sources.
}
to
\begin{important}{equation}
  T \ideq T_0 \log \roundbr*{\frac{r_-}{r_+}}.
  \label{eq:bipolar-laplace-solution-source-distances}
\end{important}
This can be rewritten as
\begin{equation}
  T \ideq T_0 \tanh^{-1} \roundbr*{\frac{2 a x}{x^2 + y^2 + a^2}},
  \label{eq:bipolar-laplace-solution-inverse-tanh}
\end{equation}
and one sees
(by rearranging and completing the square)
that the $T$-contours are simply circles.
The bipolar coordinate system~$(u, v)$ arises
from simply using the dimensionless quantity~$T / T_0$
as the second coordinate,
\begin{equation}
  v \ideq \tanh^{-1} \roundbr*{\frac{2 a x}{x^2 + y^2 + a^2}}.
  \label{eq:v-transformation-bipolar}
\end{equation}
so that the $T$-contours coincide with the $v$-contours
and the known solution is simply given by
\begin{important}{equation}
  T \ideq T_0 \cdot v.
  \label{eq:bipolar-laplace-solution}
\end{important}
Desiring an orthogonal coordinate system,
and knowing that a potential and its corresponding stream function
cross everywhere at right angles,
the other coordinate~$u$ is determined by computing
the harmonic conjugate of~$v$,
yielding the angular coordinate%
\footnote{
  The arctangent in~(\ref{eq:u-transformation-bipolar})
  is to be returned in the quadrant corresponding to
  abscissa~$x^2 + y^2 - a^2$ and ordinate~$2 a y$.
}
\begin{equation}
  u \ideq \tan^{-1} \roundbr*{\frac{2 a y}{x^2 + y^2 - a^2}}.
  \label{eq:u-transformation-bipolar}
\end{equation}

\subsection{Bipolar coordinates}
\label{sec:bipolar.known.coordinates}

Of course it is the forward transformation which is more useful,
and upon inverting~(\ref{eq:u-transformation-bipolar})
and~(\ref{eq:v-transformation-bipolar})
one obtains
\begin{align}
  x &\ideq \frac{a \sinh v}{\cosh v - \cos u},
    \label{eq:x-transformation-bipolar}
    \\[\tallspace]
  y &\ideq \frac{a \sin u}{\cosh v - \cos u}.
    \label{eq:y-transformation-bipolar}
\end{align}
The length scale~$a$ is intrinsic to the coordinate system,
as it encodes the locations of the singularities~$(x, y) = (\pm a, 0)$.
The scale factors for the two coordinates turn out to be the same:
\begin{equation}
  \scalefac \ideq \frac{a}{\cosh v - \cos u}.
  \label{eq:scale-factor-bipolar}
\end{equation}

Figure~X shows the contour lines % TODO
of each bipolar coordinate.
The $v$-contours are non-intersecting circles
which enclose the nearest singularity.
The limiting case~$v = -\infty$ corresponds to the negative singularity,
and as $v$~increases the circular contour grows until~$v = 0$
where it coincides with the $y$-axis.
As $v$~waxes positive the circular contour shrinks
until it becomes the positive singularity at~$v = +\infty$.

The $u$-contours on the other hand are \emph{not} full circles,
but rather circular arcs which terminate at the singularities.
Geometrically, $u$~is the angle inscribed within each circular arc
(Figure~X), % TODO
with $u < \SI{180}{\degree}$~for arcs above the $x$-axis
and $u > \SI{180}{\degree}$~for arcs below it.
Note that $u$~increases anticlockwise around the positive singularity
and clockwise around the negative singularity.

\subsection{Scaling}
\label{sec:bipolar.known.scaling}

Since the length scale has already been fixed at~$a$,
define
\begin{align}
  \scaled{x} &\ideq x / a, \label{eq:bipolar-scaled-x} \\
  \scaled{y} &\ideq y / a, \label{eq:bipolar-scaled-y}
\end{align}
so that
\begin{equation}
  \scaleddel \ideq a \del.
  \label{eq:bipolar-scaled-del}
\end{equation}
For the bipolar scale factor~$\scalefac$ it is sensible to put
\begin{equation}
  \scaled{\scalefac} \ideq \scalefac / a.
  \label{eq:bipolar-scaled-scale-factor}
\end{equation}
Also putting
\begin{equation}
  \scaled{T} \ideq T / \Theta,
  \label{eq:bipolar-scaled-temperature}
\end{equation}
the radiation boundary condition~(\ref{eq:radiation-boundary-condition})
and the known solution~(\ref{eq:bipolar-laplace-solution})
become
\begin{align}
  \normalvec \dotp \scaleddel \scaled{T}
    &= -\group{c a \Theta^3} \scaled{T}^4,
    \label{eq:bipolar-scaled-radiation-boundary-condition-with-groups}
    \\[\tallspace]
  \scaled{T} &\ideq \group{\frac{T_0}{\Theta}} v.
    \label{eq:bipolar-scaled-laplace-solution-with-groups}
\end{align}
Given two dimensionless groups but only one free scale~$\Theta$,
only one group can be set to unity,
and as in the polar case (Section~\ref{sec:polar.line.scaling})
I choose
\begin{equation}
  \Theta = T_0
    \label{eq:bipolar-temperature-scale}
\end{equation}
and define the dimensionless group
\begin{equation}
  A = \frac{1}{c a {T_0}^3}.
  \label{eq:bipolar-dimensionless-group}
\end{equation}
\atten{Dropping \scalingmarks},
one has boundary condition and known solution
\begin{important}{align}
  \normalvec \dotp \del T &= -\frac{T^4}{A},
    \label{eq:bipolar-scaled-radiation-boundary-condition} \\
  T &\ideq v,
    \label{eq:bipolar-scaled-laplace-solution}
\end{important}
along with the dimensionless scale factor
\begin{equation}
  \scalefac \ideq \frac{1}{\cosh v - \cos u}.
  \label{eq:dimensionless-scale-factor-bipolar}
\end{equation}

\section{Viable domain}
\label{sec:bipolar.viable}

With the bipolar coordinate system,
the known solution,
and the scaling
all established,
I now determine the geometry of the viable domain.
Note that the half-plane~$x < 0$ is excluded from the analysis
since~$v < 0$ and hence~$T < 0$ in that region,
which is unphysical in the context of thermal radiation.

The bipolar components of the temperature gradient~$\del T$ are
\begin{align}
  P &\ideq \frac{1}{h} \pder{T}{u} \ideq 0,
    \label{eq:bipolar-gradient-u-component} \\[\tallspace]
  Q &\ideq \frac{1}{h} \pder{T}{v} \ideq \frac{1}{h}.
    \label{eq:bipolar-gradient-v-component}
\end{align}
Comparing the radiation condition~%
  (\ref{eq:bipolar-scaled-radiation-boundary-condition})
to the generic~(\ref{eq:flux-boundary-condition}),
it follows that the flux function is
\begin{align*}
  F
  &\ideq -\frac{T^4}{A} \\[\tallspace]
  &\ideq -\frac{v^4}{A},
    \yesnumber
    \label{eq:bipolar-flux-function}
\end{align*}
and therefore the viability function is given by
\begin{align*}
  \Phi
  &\ideq (\del T)^2 - F^2 \\[\tallspace]
  &\ideq \frac{1}{h^2} - \frac{v^8}{A^2} \\[\tallspace]
  &\ideq \roundbr[\bulkysize]{\cosh v - \cos u}^2 - \frac{v^8}{A^2}.
    \yesnumber
    \label{eq:bipolar-viability-function}
\end{align*}
Thus the viable domain~$\Phi \ge 0$ is the region
\begin{equation}
  \cosh v - \cos u \ge \frac{v^4}{A}.
  \label{eq:bipolar-viable-domain}
\end{equation}

\subsection{Critical terminal points}
\label{sec:bipolar.viable.critical}

The inequality~(\ref{eq:bipolar-viable-domain}) is complicated,
and the shape of the viable domain cannot be seen easily.
There is, however, sufficient symmetry
to identify (by hand) all of the critical terminal points,
which will be useful later on when constructing convex radiation boundaries,
and whose number shall determine the geometry of the viable domain.

Recall that a terminal point is critical
if the local $T$-contour touches
the local portion of the terminal curve~$\Phi = 0$ tangentially.
Now, the terminal curve is given by
\begin{equation}
  \cosh v - \cos u = \frac{v^4}{A},
  \label{eq:bipolar-terminal-curve}
\end{equation}
and implicit differentiation yields the equation
\begin{equation}
  \roundbr*{\sinh v - \frac{4 v^3}{A}} \tder{v}{u} + \sin u = 0
  \label{eq:bipolar-terminal-curve-implicit-derivative}
\end{equation}
for the (bipolar) slope~$\td v / {\td u}$
along the terminal curve.
Since the $T$-contours
of the known solution~(\ref{eq:bipolar-scaled-laplace-solution})
are simply~$v = \const$,
a $T$-contour will only touch the terminal curve tangentially
where $\td v / {\td u} = 0$
in~(\ref{eq:bipolar-terminal-curve-implicit-derivative}).
Therefore the critical terminal points are necessarily located
at~$u = 0$ or~$u = \pi$ along the terminal curve.
