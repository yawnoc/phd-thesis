\chapter{Bipolar coordinates}
\label{ch:bipolar}

In this chapter,
I apply boundary tracing to the conduction--radiation problem
starting from the known solution
to Laplace's equation~(\ref{eq:laplace-steady-conduction})
for equal and opposite line sources.
As before,
I determine traced boundaries along which
the radiation condition~(\ref{eq:radiation-boundary-condition}) holds
and use the convex portions to construct conduction--radiation domains.

\section{Known solution}
\label{sec:bipolar.known}

The problem at hand is best tackled using bipolar coordinates.
While it would be well to introduce the bipolar coordinate system
before writing down the known solution,
a better understanding of the physics is obtained
by first writing down the known solution
and then observing how bipolar coordinates arise as a result.

Suppose that the equal and opposite line sources
are the same strength as the line source~(\ref{eq:line-laplace-solution})
but located at~$(x, y) = (\pm a, 0)$.
Since the distance to each source is
\begin{equation}
  r_\pm = \sqrt{(x \mp a)^2 + y^2},
  \label{eq:bipolar-source-distances}
\end{equation}
the known solution is given by
\begin{equation}
  T \ideq
    T_0 \log \roundbr*{\frac{r_0}{r_+}}
      -
    T_0 \log \roundbr*{\frac{r_0}{r_-}},
  \label{eq:bipolar-laplace-solution-terms}
\end{equation}
which reduces%
\footnote{
  The remarkable cancellation of the reference length~$r_0$
  occurs only for equal and opposite line sources.
}
to
\begin{important}{equation}
  T \ideq T_0 \log \roundbr*{\frac{r_-}{r_+}}.
  \label{eq:bipolar-laplace-solution-source-distances}
\end{important}
This can be rewritten as
\begin{equation}
  T \ideq T_0 \tanh^{-1} \roundbr*{\frac{2 a x}{x^2 + y^2 + a^2}},
  \label{eq:bipolar-laplace-solution-inverse-tanh}
\end{equation}
and one sees
(by rearranging and completing the square)
that the $T$-contours are simply circles.
The bipolar coordinate system~$(u, v)$ arises
from simply using the dimensionless quantity~$T / T_0$
as the second coordinate,
\begin{equation}
  v \ideq \tanh^{-1} \roundbr*{\frac{2 a x}{x^2 + y^2 + a^2}}.
  \label{eq:v-transformation-bipolar}
\end{equation}
so that the $T$-contours coincide with the $v$-contours
and the known solution is simply given by
\begin{important}{equation}
  T \ideq T_0 \cdot v.
  \label{eq:bipolar-laplace-solution}
\end{important}
Desiring an orthogonal coordinate system,
and knowing that a potential and its corresponding stream function
cross everywhere at right angles,
the other coordinate~$u$ is determined by computing
the harmonic conjugate of~$v$,
yielding the coordinate
\begin{equation}
  u \ideq \tan^{-1} \roundbr*{\frac{2 a y}{x^2 + y^2 - a^2}}.
  \label{eq:u-transformation-bipolar}
\end{equation}
