\chapter{Bipolar coordinates}
\label{ch:bipolar}

In this chapter,
I apply boundary tracing to the conduction--radiation problem
starting from the known solution
to Laplace's equation~(\ref{eq:laplace-steady-conduction})
for equal and opposite line sources.
As before,
I determine traced boundaries along which
the radiation condition~(\ref{eq:radiation-boundary-condition}) holds
and use the convex portions to construct conduction--radiation domains.

\section{Known solution}
\label{sec:bipolar.known}

The problem at hand is best tackled using the bipolar coordinate system.
While it would be well to introduce bipolar coordinates
before writing down the known solution,
a better understanding of the physics is obtained
by first writing down the known solution
and then observing how bipolar coordinates arise as a result.
