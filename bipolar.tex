\chapter{Bipolar coordinates}
\label{ch:bipolar}

In this chapter
we apply boundary tracing to the conduction--radiation problem
starting from the known solution
to Laplace's equation~(\ref{eq:laplace-steady-conduction})
for equal and opposite line sources.
As before,
we determine traced boundaries along which
the radiation condition~(\ref{eq:radiation-boundary-condition}) holds
and use the convex portions to construct conduction--radiation domains.

\section{Known solution}
\label{sec:bipolar.known}

The problem at hand is best tackled using bipolar coordinates.
While it would be well to introduce the bipolar coordinate system
before writing down the known solution,
a better understanding of the physics is obtained
by first writing down the known solution
and then observing how bipolar coordinates arise as a result.

Suppose that the equal and opposite line sources
are the same strength as the line source~(\ref{eq:line-laplace-solution})
but located at~$(x, y) = (\pm a, 0)$.
Since the distance to each source is
\begin{equation}
  r_\pm = \sqrt{(x \mp a)^2 + y^2},
  \label{eq:bipolar-source-distances}
\end{equation}
the known solution is given by
\begin{equation}
  T =
    T_0 \log \roundbr*{\frac{r_0}{r_+}}
      -
    T_0 \log \roundbr*{\frac{r_0}{r_-}},
  \label{eq:bipolar-laplace-solution-terms}
\end{equation}
which reduces%
\footnote{
  The remarkable cancellation of the reference length~$r_0$
  occurs only for equal and opposite line sources.
}
to
\begin{important}{equation}
  T = T_0 \log \roundbr*{\frac{r_-}{r_+}}.
  \label{eq:bipolar-laplace-solution-source-distances}
\end{important}
This can be rewritten as
\begin{equation}
  T = T_0 \tanh^{-1} \roundbr*{\frac{2 a x}{x^2 + y^2 + a^2}},
  \label{eq:bipolar-laplace-solution-inverse-tanh}
\end{equation}
and the bipolar coordinate system~$(u, v)$ arises
from using the dimensionless quantity
\begin{equation}
  v
    = \frac{T}{T_0}
    = \tanh^{-1} \roundbr*{\frac{2 a x}{x^2 + y^2 + a^2}}
  \label{eq:v-transformation-bipolar}
\end{equation}
as the second coordinate,
so that $T$-contours are also $v$-contours,
and the known solution is simply given by
\begin{important}{equation}
  T = T_0 \cdot v.
  \label{eq:bipolar-laplace-solution}
\end{important}
By rearranging~(\ref{eq:v-transformation-bipolar}) and completing the square,
we see that the $T$-contours (and hence $v$-contours) are circles of the form
\begin{equation}
  (x - a \coth v)^2 + y^2 = a^2 \csch^2 v.
  \label{eq:bipolar-circle-v}
\end{equation}
Thus we will have constant-temperature (heat-supplying) boundaries
of highly practical circular shape.
Desiring an orthogonal coordinate system,
and knowing that a potential and its associated stream function
cross everywhere at right angles,
the other bipolar coordinate~$u$ is determined by computing
the harmonic conjugate of~$v$,
yielding the angular coordinate
\begin{equation}
  u = \tan^{-1} \roundbr*{\frac{2 a y}{x^2 + y^2 - a^2}}.
  \label{eq:u-transformation-bipolar}
\end{equation}
Here, the arctangent is to be returned in the quadrant corresponding to
an abscissa of~$x^2 + y^2 - a^2$ and an ordinate of~$2 a y$,
so that $u$~is reckoned modulo~$2 \pi$ rather than~$\pi$.

\subsection{Bipolar coordinates}
\label{sec:bipolar.known.coordinates}

Of course it is the forward transformation which is more useful,
and upon inverting~(\ref{eq:v-transformation-bipolar})
\&~(\ref{eq:u-transformation-bipolar})
we have
\begin{align}
  x &= \frac{a \sinh v}{\cosh v - \cos u},
    \label{eq:x-transformation-bipolar}
    \\[\tallspace]
  y &= \frac{a \sin u}{\cosh v - \cos u}.
    \label{eq:y-transformation-bipolar}
\end{align}
The length scale~$a$ is intrinsic to the coordinate system,
as it encodes the locations of the singularities~$(x, y) = (\pm a, 0)$.
The scale factors for the two coordinates turn out to be the same:
\begin{equation}
  \scalefac = \frac{a}{\cosh v - \cos u}.
  \label{eq:scale-factor-bipolar}
\end{equation}

\begin{figure}
  \centredfigurecontent[width=0.87\textwidth]{bipolar-coordinates}{
    Contours of the bipolar coordinates~$(u, v)$.
    The negative singularity~($v = -\infty$) is~$(x, y) = (-a, 0)$.
    The positive singularity~($v = +\infty$) is~$(x, y) = (+a, 0)$.
  }
\end{figure}

Figure~\ref{fig:bipolar-coordinates} shows the contours
of the two bipolar coordinates.
The $v$-contours are the non-intersecting circles~(\ref{eq:bipolar-circle-v})
which enclose the nearest singularity.
The limiting case~$v = -\infty$ corresponds to the negative singularity.
As $v$~increases the circular contour grows,
until~$v = 0$ where it coincides with the $y$-axis.
As $v$~waxes positive the circular contour shrinks,
becoming the positive singularity at~$v = +\infty$.

The equation~(\ref{eq:u-transformation-bipolar})
for the angular coordinate~$u$
may likewise undergo rearrangement and completion of the square
to become
\begin{equation}
  x^2 + (y - a \cot u)^2 = a^2 \csc^2 u.
  \label{eq:bipolar-circle-u}
\end{equation}
Despite this, the $u$-contours are \emph{not} full circles.
Since the arctangent in~(\ref{eq:u-transformation-bipolar})
is to be returned in the quadrant of~$(x^2 + y^2 - a^2, \, 2 a y)$,
the $u$-contours are in fact
only circular \emph{arcs} which terminate at the singularities
(Figure~\ref{fig:bipolar-u}),
with $u < \SI{180}{\degree}$~for arcs above the $x$-axis (i.e.~$y > 0$)
and $u > \SI{180}{\degree}$~for arcs below it (i.e.~$y < 0$).
Geometrically, $u$~is the angle on the underside of the two chords
from the two singularities.
Note that $u$~increases anticlockwise around the positive singularity
and clockwise around the negative singularity.

\begin{figure}
  \newcommand*{\subfigurewidth}{0.35\textwidth}
  \centering
  \hspace*{\fill}
  \begin{subfigure}[t]{\subfigurewidth}
    \centredfigurecontent{bipolar-u-less-than-pi}{%
      $u = \const < \SI{180}{\degree}$
    }
  \end{subfigure}
    \hfill
  \begin{subfigure}[t]{\subfigurewidth}
    \centredfigurecontent{bipolar-u-more-than-pi}{%
      $u = \const > \SI{180}{\degree}$
    }
  \end{subfigure}
  \hspace*{\fill}
  \caption{
    Circular arcs formed by curves of constant~$u$.
  }
  \label{fig:bipolar-u}
\end{figure}

\subsection{Scaling}
\label{sec:bipolar.known.scaling}

Since the length scale has already been fixed at~$a$,
we let
\begin{align}
  x &= a \scaled{x}, \label{eq:bipolar-x-scaling} \\
  y &= a \scaled{y}, \label{eq:bipolar-y-scaling}
\end{align}
so that
\begin{equation}
  \del = \scaleddel / a.
  \label{eq:bipolar-del-scaling}
\end{equation}
For the bipolar scale factor~$\scalefac$ it is sensible to put
\begin{equation}
  \scalefac = a \scaled{\scalefac}.
  \label{eq:bipolar-scale-factor-scaling}
\end{equation}
Also putting
\begin{equation}
  T = \Theta \scaled{T},
  \label{eq:bipolar-temperature-scaling}
\end{equation}
the radiation boundary condition~(\ref{eq:radiation-boundary-condition})
and the known solution~(\ref{eq:bipolar-laplace-solution})
become
\begin{align}
  \normalvec \dotp \scaleddel \scaled{T}
    &= -\group{c a \Theta^3} \scaled{T}^4,
    \label{eq:bipolar-scaled-radiation-boundary-condition-with-groups}
    \\[\tallspace]
  \scaled{T} &= \group{\frac{T_0}{\Theta}} v.
    \label{eq:bipolar-scaled-laplace-solution-with-groups}
\end{align}
Given two dimensionless groups but only one free scale~$\Theta$,
only one group can be set to unity,
and as in the polar case (Section~\ref{sec:polar.line.scaling})
we choose
\begin{equation}
  \Theta = T_0
    \label{eq:bipolar-temperature-scale}
\end{equation}
and define the dimensionless group
\begin{equation}
  A = \frac{1}{c a {T_0}^3}.
  \label{eq:bipolar-dimensionless-group}
\end{equation}
Dropping \scalingmarks,
we have boundary condition and known solution
\begin{important}{align}
  \normalvec \dotp \del T &= -\frac{T^4}{A},
    \label{eq:bipolar-scaled-radiation-boundary-condition} \\
  T &= v,
    \label{eq:bipolar-scaled-laplace-solution}
\end{important}
along with the dimensionless scale factor
\begin{equation}
  \scalefac = \frac{1}{\cosh v - \cos u}.
  \label{eq:dimensionless-scale-factor-bipolar}
\end{equation}

\section{Viable domain}
\label{sec:bipolar.viable}

With the bipolar coordinate system,
the known solution,
and the scaling
all established,
we now determine the geometry of the viable domain.
Note that the half-plane~$x < 0$ is excluded from the analysis
since~$v < 0$ and hence~$T < 0$ in that region,
which is unphysical in the context of thermal radiation.

The bipolar components of the temperature gradient~$\del T$ are
\begin{align}
  P &= \frac{1}{h} \pder{T}{u} = 0,
    \label{eq:bipolar-gradient-u-component} \\[\tallspace]
  Q &= \frac{1}{h} \pder{T}{v} = \frac{1}{h}.
    \label{eq:bipolar-gradient-v-component}
\end{align}
Comparing the radiation condition~%
  (\ref{eq:bipolar-scaled-radiation-boundary-condition})
to the generic~(\ref{eq:flux-boundary-condition}),
it follows that the flux function is
\begin{align*}
  F
  &= -\frac{T^4}{A} \\[\tallspace]
  &= -\frac{v^4}{A},
    \yesnumber
    \label{eq:bipolar-flux-function}
\end{align*}
and therefore the viability function is given by
\begin{align*}
  \Phi
  &= (\del T)^2 - F^2 \\[\tallspace]
  &= \frac{1}{h^2} - \frac{v^8}{A^2} \\[\tallspace]
  &= \roundbr[\bulkysize]{\cosh v - \cos u}^2 - \frac{v^8}{A^2}.
    \yesnumber
    \label{eq:bipolar-viability-function}
\end{align*}
Thus the viable domain~$\Phi \ge 0$ is the region
\begin{equation}
  \cosh v - \cos u \ge \frac{v^4}{A}.
  \label{eq:bipolar-viable-domain}
\end{equation}

\subsection{Critical terminal points}
\label{sec:bipolar.viable.critical}

The inequality~(\ref{eq:bipolar-viable-domain}) is complicated,
and the shape of the viable domain cannot be seen easily.
There is, however, sufficient symmetry
to identify (by hand) all of the critical terminal points,
which will be useful later on when constructing convex radiation boundaries,
and whose number shall determine the geometry of the viable domain.

Recall that a terminal point is critical
if the local $T$-contour touches
the local portion of the terminal curve~$\Phi = 0$ tangentially.
Now, the terminal curve is given by
\begin{equation}
  \cosh v - \cos u = \frac{v^4}{A},
  \label{eq:bipolar-terminal-curve}
\end{equation}
and implicit differentiation yields the equation
\begin{equation}
  \roundbr*{\sinh v - \frac{4 v^3}{A}} \tder{v}{u} + \sin u = 0
  \label{eq:bipolar-terminal-curve-implicit-derivative}
\end{equation}
for the (bipolar) slope~$\td v / {\td u}$
along the terminal curve.
Since the $T$-contours
of the known solution~(\ref{eq:bipolar-scaled-laplace-solution})
are simply~$v = \const$,
a $T$-contour will only touch the terminal curve tangentially
where $\td v / {\td u} = 0$
in~(\ref{eq:bipolar-terminal-curve-implicit-derivative}).%
\footnote{
  Alternatively $\td v / {\td u}$~may be indeterminate
  if~$(\sinh v - 4 v^3 / A) = \sin u = 0$,
  as is the case for the two transition cases
  in Section~\ref{sec:bipolar.viable.regimes} to follow.
  However, this still implies~$u = 0$ or~$u = \pi$.
}
Therefore the critical terminal points are necessarily located
at~$u = 0$ or~$u = \pi$ along the terminal curve,
and are given by
\begin{align}
  \cosh v - 1 &= \frac{v^4}{A} \eqnspace \text{along~$u = 0$},
  \label{eq:bipolar-critical-terminal-point-0}
    \\[\tallspace]
  \cosh v + 1 &= \frac{v^4}{A} \eqnspace \text{along~$u = \pi$}.
  \label{eq:bipolar-critical-terminal-point-pi}
\end{align}
Here it is noted that the trivial root~$v = 0$
of~(\ref{eq:bipolar-critical-terminal-point-0})
corresponds to an isolated critical terminal point
at the origin~$(u, v) = (0, 0)$.
The full terminal curve actually consists of the origin
in union with a \term{proper} terminal curve
whose points are not isolated,
a situation already encountered
in Section~\ref{sec:cartesian.cosine.simple.viable}.
Again we find that the two traced boundaries through the origin
form a (useless) non-convex spike,
so for the remainder of this analysis
the origin is ignored as a critical terminal point.

\begin{figure}
  \centering
  \begin{minipage}[b]{0.1\textwidth}
    \includegraphics[height=13.6\textwidth]{bipolar-viable-arrow}
  \end{minipage}
  \begin{minipage}[b]{0.8\textwidth}
    \newcommand*{\legendtrimwidth}{0.03\textwidth}
    \newcommand*{\legendoffsetheight}{0.025\textwidth}
    \includegraphics[
      width={\textwidth-\legendtrimwidth},
      trim={\legendtrimwidth} {-\legendoffsetheight} 0 0,
    ]{bipolar-viable-legend}
    \includegraphics[width=\textwidth]{bipolar-viable}
  \end{minipage}
  \caption{
    Critical terminal points and non-viable domain~$\Phi < 0$
    for the known solution~(\ref{eq:bipolar-scaled-laplace-solution})
    and viability function~(\ref{eq:bipolar-viability-function}),
    as $A$~increases.
    In the right-hand column,
    the $u$-contours~\figurestyle{grey curves} meet
    at the positive singularity~$v = +\infty$.
  }
  \label{fig:bipolar-viable}
\end{figure}

Each of~(\ref{eq:bipolar-critical-terminal-point-0})
and~(\ref{eq:bipolar-critical-terminal-point-pi})
has between zero and two roots
(ignoring~$v = 0$ for~(\ref{eq:bipolar-critical-terminal-point-0}))
depending on the size of the dimensionless group~$A$,
leading to five cases for the number of critical terminal points
and the geometry of the viable domain
(Figure~\ref{fig:bipolar-viable}):

\subsection{Five regimes}
\label{sec:bipolar.viable.regimes}

\begin{enumerate}
  \item
    \term{Hot regime}, $A < A_{\nat \pi}$:
    Both equations have two roots:
    $v_{\flat 0}$ and~$v_{\sharp 0}$
    for~(\ref{eq:bipolar-critical-terminal-point-0})
    and
    $v_{\flat \pi}$ and~$v_{\sharp \pi}$
    for~(\ref{eq:bipolar-critical-terminal-point-pi}),
    with~$v_{\flat 0} < v_{\flat \pi} < v_{\sharp \pi} < v_{\sharp 0}$.
    Thus there are four critical terminal points,
    having bipolar coordinates~$(0, v_{\flat 0})$, $(\pi, v_{\flat \pi})$,
    $(\pi, v_{\sharp \pi})$, and~$(0, v_{\sharp 0})$.
    The non-viable domain forms an avocado-like moat
    which surrounds an inner viable island
    (containing the singularity~$v = +\infty$)
    and is surrounded by an outer viable mainland.
  \item
    \term{Hot-to-warm transition}, $A = A_{\nat \pi}$:
    The two roots of~(\ref{eq:bipolar-critical-terminal-point-pi})
    merge together
    and the non-viable moat is pincered along~$u = \pi$,
    leaving the three critical terminal points~$(0, v_{\flat 0})$,
    $(\pi, v_{\nat \pi})$, and~$(0, v_{\sharp 0})$.
    The inner viable island and the outer viable mainland
    touch at the second of these.
  \item
    \term{Warm regime}, $A_{\nat \pi} < A < A_{\nat 0}$:
    The remaining root of~(\ref{eq:bipolar-critical-terminal-point-pi})
    has disappeared
    and the inner viable island is now robustly connected
    to the outer viable mainland.
    Equation~(\ref{eq:bipolar-critical-terminal-point-0}) still has two roots,
    corresponding to the two critical terminal points~%
      $(0, v_{\flat 0})$ and~$(0, v_{\sharp 0})$.
    The non-viable domain is now a crescent-shaped lake.
  \item
    \term{Warm-to-cold transition}, $A = A_{\nat 0}$:
    The two roots of~(\ref{eq:bipolar-critical-terminal-point-0})
    merge together
    and the non-viable lake dries up completely.
    The entire plane is viable,
    though a single (isolated) critical terminal point still exists
    at~$(0, v_{\nat 0})$.
  \item
    \term{Cold regime}, $A > A_{\nat 0}$:
    The last remaining root of~(\ref{eq:bipolar-critical-terminal-point-0})
    disappears.
    The entire plane is viable,
    and there are no critical terminal points left.
\end{enumerate}
The two transition cases occur
when~(\ref{eq:bipolar-critical-terminal-point-0})
and~(\ref{eq:bipolar-critical-terminal-point-pi})
each have a merging of roots,
at the special value of~$A$
for which the curves~$\cosh v \pm 1$ and~$v^4 / A$
touch tangentially,
given by the extra condition
\begin{equation}
  \sinh v - \frac{4 v^3}{A} = 0.
  \label{eq:bipolar-critical-terminal-point-transition}
\end{equation}
Note how this accounts for the situation in which
the terminal curve has an indeterminate local tangent
in equation~(\ref{eq:bipolar-terminal-curve-implicit-derivative}).
By numerically solving each of~(\ref{eq:bipolar-critical-terminal-point-0})
and~(\ref{eq:bipolar-critical-terminal-point-pi})
in conjunction with the tangency condition~%
  (\ref{eq:bipolar-critical-terminal-point-transition}),
the values of~$A$ for the two transitions are determined to be
\begin{align}
  A_{\nat \pi} &= 9.06433 \eqnspace \text{hot-to-warm},
    \label{eq:bipolar-transition-a-hot-to-warm}
    \\
  A_{\nat 0} &= 9.76206 \eqnspace \text{warm-to-cold}.
    \label{eq:bipolar-transition-a-warm-to-cold}
\end{align}

\section{Boundary tracing}
\label{sec:bipolar.tracing}

In this section we write down the boundary tracing ODE
and determine the class of convex domains which can be constructed.

\subsection{Candidate boundaries}
\label{sec:bipolar.tracing.candidates}

\begin{figure}
  \newcommand*{\subfigurewidth}{0.45\textwidth}
  \newcommand*{\subfigureoffsettop}{0.08\textwidth}
  \newcommand*{\subfigureoffsetbottom}{0.04\textwidth}
  \centering
  \begin{subfigure}{\subfigurewidth}
    \centredfigurecontent[
      trim=0 {\subfigureoffsetbottom} 0 0,
    ]{bipolar-traced-boundaries-hot}{Hot regime}
  \end{subfigure}
  \hfill
  \begin{subfigure}{\subfigurewidth}
    \centredfigurecontent[
      trim=0 {\subfigureoffsetbottom} 0 0,
    ]{bipolar-traced-boundaries-warm}{Warm regime}
  \end{subfigure}
  
  \begin{subfigure}{\subfigurewidth}
    \centredfigurecontent[
      trim=0 {\subfigureoffsetbottom} 0 {-\subfigureoffsettop},
    ]{bipolar-traced-boundaries-cold}{Cold regime}
  \end{subfigure}
  \caption{
    Traced boundaries obtained by integrating~%
      (\ref{eq:bipolar-tracing-ode-coordinate-parametrisation-v}).
  }
  \label{fig:bipolar-traced-boundaries}
\end{figure}

Using~(\ref{eq:bipolar-gradient-u-component})
through~(\ref{eq:bipolar-viability-function}),
the boundary tracing ODE~(\ref{eq:tracing-ode-coordinate-parametrisation-v})
becomes
\begin{important}{equation}
  \tder{v}{u} =
    \pm
    \frac{A}{v^4}
    \sqrt{
      \roundbr[\bulkysize]{\cosh v - \cos u}^2 - \frac{v^8}{A^2}
    },
  \label{eq:bipolar-tracing-ode-coordinate-parametrisation-v}
\end{important}
which cannot be integrated analytically.
Traced boundaries determined by numerical integration
from starting points in the viable domain
are shown in Figure~\ref{fig:bipolar-traced-boundaries}.

Like the line-source case of Section~\ref{sec:polar.tracing},
the two branches of traced boundaries are segregated
by the sign of~$\td v / {\td u}$,
with the upper branch spiralling inwards
and the lower branch spiralling outwards
as one travels anticlockwise
around the singularity~$v = +\infty$.
Again, any physically sensible radiation--conduction domain
must completely surround the heat-supplying singularity without touching it,
and therefore we seek closed curves surrounding the singularity,
made from patching together the traced boundaries
given by~(\ref{eq:bipolar-tracing-ode-coordinate-parametrisation-v}).
Using similar arguments to those in Section~\ref{sec:polar.tracing},
we conclude that a necessary (but not sufficient) condition
for the convexity of the sought-after closed curve
is for it to have an upper-to-lower branch switch
at an hyperbolic critical terminal point.

\begin{figure}
  \newcommand*{\subfigurewidth}{0.35\textwidth}
  \newcommand*{\subfigureoffsetbottom}{0.08\textwidth}
  \centering
  \hspace*{\fill}
  \begin{subfigure}{\subfigurewidth}
    \centredfigurecontent[
    ]{bipolar-critical-terminal-points-hot}{Hot regime}
  \end{subfigure}
    \hfill
  \begin{subfigure}{\subfigurewidth}
    \centredfigurecontent[
    ]{bipolar-critical-terminal-points-warm_hot}{Hot-to-warm transition}
  \end{subfigure}
  \hspace*{\fill}
  
  \hspace*{\fill}
  \begin{subfigure}{\subfigurewidth}
    \centredfigurecontent[
      trim=0 {\subfigureoffsetbottom} 0 0,
    ]{bipolar-critical-terminal-points-warm}{Warm regime}
  \end{subfigure}
    \hfill
  \begin{subfigure}{\subfigurewidth}
    \centredfigurecontent[
      trim=0 {\subfigureoffsetbottom} 0 0,
    ]{bipolar-critical-terminal-points-cold_warm}{Warm-to-cold transition}
  \end{subfigure}
  \hspace*{\fill}
  
  \includegraphics[
    width=\textwidth,
    trim=0 0 0 -5,
  ]{bipolar-critical-terminal-points-legend}
  \caption{
    Local $T$-contour through each critical terminal point.
  }
  \label{fig:critical-terminal-points}
\end{figure}

The nature of the up to four critical terminal points that exist
can be determined by inspecting Figure~\ref{fig:critical-terminal-points}.
We see that those lying on the segment~$u = \pi$
(to the left of the singularity)
are of elliptic type,
as the local $T$-contour lies on the non-viable side of the terminal curve.
Only the critical terminal points lying on the segment~$u = 0$
(to the right of the singularity)
are of hyperbolic type,
and it is from these points that we might construct convex domains.
Explicitly, we have two hyperbolic critical terminal points~%
$(u, v) = (0, v_{\flat 0})$ and~$(u, v) = (0, v_{\sharp 0})$
for~$0 < A < A_{\nat 0}$ (the hot regime through to the warm regime),
merging to become the single point~$(u, v) = (0, v_{\nat 0})$
at~$A = A_{\nat 0}$ (the warm-to-cold transition)
before disappearing for~$A > A_{\nat 0}$ (the cold regime).

\begin{figure}
  \centering
  \includegraphics[width=\textwidth]{bipolar-candidates}
  \includegraphics[width=\textwidth]{bipolar-candidates-arrow}
  \caption{
    Inner~($v_{\sharp 0}$) and outer~($v_{\flat 0}$) candidate boundaries,
    as $A$~increases.
    The two candidate boundaries become one at~$A = A_{\nat 0}$.
  }
  \label{fig:bipolar-candidates}
\end{figure}

We may therefore construct (up to) two \term{candidate boundaries}
whenever~$0 < A \le A_{\nat 0}$,
as shown in Figure~\ref{fig:bipolar-candidates}.
An \term{outer candidate boundary} is constructed
by performing an upper-to-lower branch switch at~$(0, v_{\flat 0})$,
i.e.~by taking the upper branch for~$-\pi < u \le 0$
and joining it to the lower branch for~$0 \le u < \pi$.
An \term{inner candidate boundary} is produced likewise
from the point~$(0, v_{\sharp 0})$.
The two boundaries merge together
when $v_{\flat 0}$ and~$v_{\sharp 0}$ merge
at the warm-to-cold transition~$A = A_{\nat 0}$.

The inner candidate boundary forms a closed loop
and appears to be convex over the entire interval~$0 < A \le A_{\nat 0}$.
The outer candidate boundary only forms a closed loop
if $A$~is sufficiently large,
but in the next section we show that this is irrelevant
because the outer candidate boundary is always non-convex.

\subsection{Convexity of the candidate boundaries}
\label{sec:bipolar.tracing.convex}

We first consider the inner candidate boundary,
which always forms a closed loop
by way of a corner to the left
(Figure~\ref{fig:bipolar-inner-candidate}).
While this boundary appears to be convex for all~$A \le A_{\nat 0}$,
we cannot be certain until a global curvature analysis is performed
like that of Section~\ref{sec:polar.convex.beyond}.

\begin{figure}
  \centredfigurecontent[width=0.45\textwidth]{%
    bipolar-inner-candidate%
  }{
    Bipolar coordinates of the left-hand and right-hand extremities
    of an inner candidate boundary.
  }
\end{figure}

From the coordinate transformations
we obtain the orthonormal basis vectors
\begin{align}
  \basisvec{u} &= -\scalefac S \basisvec{x} + \scalefac C \basisvec{y},
    \label{eq:u-basis-vector-bipolar} \\
  \basisvec{v} &= -\scalefac C \basisvec{x} - \scalefac S \basisvec{y},
    \label{eq:v-basis-vector-bipolar}
\end{align}
where
\begin{align}
  S &= \sin u \sinh v,
    \label{eq:bipolar-abbreviation-s} \\
  C &= \cos u \cosh v - 1,
    \label{eq:bipolar-abbreviation-c}
\end{align}
and $\scalefac$~is the dimensionless scale factor~%
  (\ref{eq:dimensionless-scale-factor-bipolar}).
We consider a curve parametrised in the form~$v = v (u)$,
and let primes denote $u$-differentiation.
After some algebra, we find that
the basis vectors~(\ref{eq:u-basis-vector-bipolar})
and~(\ref{eq:v-basis-vector-bipolar})
change according to
\begin{align}
  (\basisvec{u})'
  &= -\scalefac^2 C D \basisvec{x} - \scalefac^2 S D \basisvec{y}
  = +\scalefac D \basisvec{v},
    \label{eq:u-basis-vector-u-derivative-bipolar} \\
  (\basisvec{v})'
  &= +\scalefac^2 S D \basisvec{x} - \scalefac^2 C D \basisvec{y}
  = -\scalefac D \basisvec{u},
    \label{eq:v-basis-vector-u-derivative-bipolar}
\end{align}
where
\begin{equation}
  D = \sinh v - v' \sin u.
  \label{eq:bipolar-abbreviation-d}
\end{equation}
Now, from the differential displacement
\begin{equation}
  \td\positionvec =
    \scalefac \td u \basisvec{u} + \scalefac \td v \basisvec{v},
  \label{eq:differential-displacement-bipolar}
\end{equation}
we have the velocity
\begin{equation}
  \positionvec' = \tder{\positionvec}{u} =
    \scalefac \basisvec{u} + \scalefac v' \basisvec{v}.
  \label{eq:velocity-vector-bipolar-by-u}
\end{equation}
Taking another $u$-derivative,
and using~(\ref{eq:u-basis-vector-u-derivative-bipolar})
and~(\ref{eq:v-basis-vector-u-derivative-bipolar})
to simplify,
we obtain the acceleration
\begin{equation}
  \positionvec'' = \tder[2]{\positionvec}{u} =
    - \scalefac^2 (D v' + E) \basisvec{u}
    + \scalefac (v'' - \scalefac E v' + \scalefac D) \basisvec{v},
  \label{eq:acceleration-vector-bipolar-by-u}
\end{equation}
where
\begin{equation}
  E = v' \sinh v + \sin u.
  \label{eq:bipolar-abbreviation-e}
\end{equation}
A quantity having the same sign changes as curvature is therefore
\begin{equation}
  \kappa = \basisvec{z} \dotp (\positionvec' \crossp \positionvec'') =
    \scalefac^3
    \squarebr*{D \roundbr*{1 + {v'}^2} + \frac{v''}{\scalefac}}.
  \label{eq:kappa-bipolar-by-u}
\end{equation}

Inspecting
Figures~\ref{fig:bipolar-candidates} and~\ref{fig:bipolar-inner-candidate}
once more,
we note again that
the inner candidate boundary appears to be always convex.
Certainly the rotund right-hand end is convex.
However, while the tip at the left-hand end looks to be a convex corner,
there remains the possibility
that it could actually be a non-convex spike
similar to Figure~\hyperref[fig:plane-domains]{\ref*{fig:plane-domains}c}
(although much more subtle).
To test this,
we evaluate~(\ref{eq:kappa-bipolar-by-u}),
first using the traced boundary derivative~%
  (\ref{eq:bipolar-tracing-ode-coordinate-parametrisation-v})
for~$v'$,
and then substituting~$u = \mp\pi$
(which is the location of the tip for each branch).
We obtain
\begin{equation}
  \eval*{\kappa}_{u = \mp\pi} =
    \frac{2 A^2}{v^9}
    \roundbr*{-2 + v \tanh\frac{v}{2}},
    \label{eq:bipolar-traced-boundary-kappa-bipolar-by-u-tip}
\end{equation}
which changes sign at the positive solution to the transcendental equation
\begin{equation}
  \frac{v}{2} \tanh\frac{v}{2} = 1,
\end{equation}
numerically
\begin{equation}
  v = v_\infl = 2.39936.
  \label{eq:bipolar-v-inflection-tip}
\end{equation}
The inner candidate boundary will therefore be convex
if and only if its tip at the left-hand end~$(u, v) = (\mp\pi, v_\End)$
either coincides with or lies to the right of~$(u, v) = (\mp\pi, v_\infl)$.
Algebraically, we have convexity if and only if~$v_\End \ge v_\infl$.
We see from Figure~\ref{fig:bipolar-candidates}
that the tip is furthest to the left when~$A = A_{\nat 0}$
(when the inner and outer candidates merge),
and in this worst-case scenario
the tip coordinate computes to~$v_\End = 2.20618$,
which is \emph{less than}
the critical value~(\ref{eq:bipolar-v-inflection-tip}).
It follows that the inner candidate boundary is in fact non-convex for
\begin{equation}
  A_\infl < A \le A_{\nat 0},
  \label{eq:bipolar-inner-candidate-non-convex-a-interval}
\end{equation}
where the lower bound~$A_\infl$ is the value of~$A$ at which
\begin{equation}
  v_\End = v_\infl,
  \label{eq:bipolar-inner-candidate-a-inflection-equation}
\end{equation}
corresponding to inflection occurring exactly at the tip.
Using the bisection algorithm we obtain
\begin{equation}
  A_\infl = 9.76036,
  \label{eq:bipolar-inner-candidate-a-inflection}
\end{equation}
which is extremely close to the upper value~$A_{\nat 0} = 9.76206$:
indeed the width of the interval~%
  (\ref{eq:bipolar-inner-candidate-non-convex-a-interval})
is an exceedingly miniscule~$1.7 \times 10^{-3}$.
It is pure coincidence that
among the entire interval~$0 < A \le A_{\nat 0}$
of inner candidate boundaries,
such a tiny fraction of them should be non-convex.
While this is most remarkable from a theoretical perspective,
we note that there is little significance in practice,
as the amount of self-viewing radiation is exceptionally tiny.

For the outer candidate boundary,
we return again to Figure~\ref{fig:bipolar-candidates}.
First we rule out the candidates with $A$~too small,
as these do not form a closed loop.
Of the remaining outer candidate boundaries
(which do form a closed loop),
we observe that the left-hand tip at best
coincides with the tip of the $A = A_{\nat 0}$~boundary
(when the inner and outer candidates merge).
As we have just seen from the analysis of the inner candidate,
this $A = A_{\nat 0}$~boundary is non-convex.
Hence, the outer candidate boundary
never forms a convex closed loop.

While it is possible to quantify the amount of self-viewing radiation
to determine non-convex yet practical domains
(both for the miniscule continuum~%
  (\ref{eq:bipolar-inner-candidate-non-convex-a-interval})
of inner candidate boundaries
and for the outer candidate boundaries that form a closed loop),
we omit such an analysis here
for the same reasons as given in Section~\ref{sec:polar.convex.self-viewing}.

\subsection{Numerical verification}
\label{sec:bipolar.tracing.verification}

\begin{figure}
  \centering
  \includegraphics[width=0.968\textwidth]{bipolar-verification-domain-mesh}
  \includegraphics[width=\textwidth]{line-verification-domain-mesh-legend}
  \caption{
    Selected domain and finite element mesh for numerical verification.
  }
  \label{fig:bipolar-verification-domain-mesh}
\end{figure}

\begin{figure}
  \centredfigurecontent[width=0.56\textwidth]{%
    bipolar-inner-candidate-with-circle%
  }{
    Circle~$v = v_{\sharp 0}$~\figurestyle{grey},
    which touches the inner candidate boundary~\figurestyle{black}
    at the critical terminal point~$(u, v) = (0, v_{\sharp 0})$.
  }
\end{figure}

For numerical verification using the finite element method,
we consider the domain shown
in Figure~\ref{fig:bipolar-verification-domain-mesh}.
The radiation boundary is the inner candidate boundary for~$A = 9.76$.
The heat-supplying singularity~$v = +\infty$
we have replaced with an equivalent Dirichlet condition~$T = T_\dir$
along a circle~$v = v_\dir$,
with~$T_\dir = v_\dir$ in accordance with
the form~(\ref{eq:bipolar-scaled-laplace-solution})
of the known solution.
In particular we require~$v_\dir > v_{\sharp 0}$
so that the Dirichlet boundary is a strictly interior one,
as the circle~$v = v_{\sharp 0}$ would touch
the critical terminal point~$(0, v_{\sharp 0})$
at the right-hand end of the radiation boundary
(Figure~\ref{fig:bipolar-inner-candidate-with-circle}).
Here we have chosen~$v_\dir = 1.1 v_{\sharp 0} = 4.26$,
so that the constant-temperature boundary
of Figure~\ref{fig:bipolar-verification-domain-mesh}
is a circle of radius~$0.028$.

We again use \software{Mathematica}'s \code{NDSolve\`{}FEM\`{}},
generating a mesh with approximately 500~triangular elements.
The relevant conduction--radiation BVP is solved numerically,
consisting of Laplace's equation in the interior,
the radiation condition~(\ref{eq:bipolar-scaled-radiation-boundary-condition})
on the external boundary,
and the aforementioned Dirichlet condition on the interior boundary.
Comparing the resulting numerical solution
to the known exact solution~(\ref{eq:bipolar-scaled-laplace-solution}),
we find that the maximum relative error throughout the mesh
is of the order~$\SI{0.1}{\percent}$.

\section{Physical range}
\label{sec:bipolar.physical}

In this section, we determine for various quantities
the physical range that can be achieved over the continuum
\begin{equation}
  0 < A \le A_{\nat 0}
  \label{eq:bipolar-inner-candidate-existence-interval}
\end{equation}
of inner candidate boundaries.
For the purposes of this analysis,
we shall ignore the issue of non-convexity
in the minute subinterval~%
  (\ref{eq:bipolar-inner-candidate-non-convex-a-interval}),
with the special value~$A_{\nat 0}$
serving as an algebraically convenient approximation
for the upper limit~$A_\infl$ of convexity.
We return here to unscaled variables,
restoring the \scalingmarks{} which were dropped
from (\ref{eq:bipolar-scaled-radiation-boundary-condition})~onwards.

\subsection{Asymmetry}
\label{sec:bipolar.physical.asymmetry}

Inspecting the inner candidate boundaries
of Figure~\ref{fig:bipolar-candidates},
we see that an increase in the dimensionless group~$A$
results in an elongation of the characteristic teardrop shape.
This makes sense physically.
The bipolar solution~(\ref{eq:bipolar-laplace-solution})
to Laplace's equation
arises from the superposition of equal and opposite line sources
at~$(x, y) = (\pm a, 0)$,
so we may think of it as a perturbation of
the radial line-source solution~(\ref{eq:line-laplace-solution})
caused by the introduction of an opposite line source at distance~$2 a$.
Since the dimensionless group~$A$
is inversely proportional to the length scale~$a$
(see~(\ref{eq:bipolar-dimensionless-group})),
an increase in~$A$ brings the second line source (which is cold)
closer to the first line source (which is hot),
thus increasing the amount of asymmetry.

It is no coincidence that we have a non-viable moat
separating an inner viable island from an outer viable mainland,
both for the line-source solution of Chapter~\ref{ch:polar}
and for the present bipolar solution, when $A$ is sufficiently small.
The limiting case of~$A = 0$
corresponds to the second line source being infinitely far removed,
i.e.~a reduction to the purely radial case.

The inner candidate boundary for the bipolar solution
is in fact a deformed version
of the trivial circular traced boundary~($r = r_\flat$)
that exists on the inner viable island,
in the hot regime of the line-source solution
(Section~\ref{sec:polar.tracing.hot}).
In the limiting $A = 0$~case,
the inner candidate boundary would be a circle
centred on the singularity~$v = +\infty$,
but for~$A > 0$,
the presence of the negative singularity
causes the inner candidate boundary to deviate
from perfectly circularity.
The amount of asymmetry may be quantified
through the ratio of distances from
the ideal centre~$(x, y) = (+a, 0)$
(which is the singularity~$v = +\infty$)
to each of the left-hand and right-hand extremities
(which have bipolar coordinates $(\mp\pi, v_\End)$ and~$(0, v_{\sharp 0})$
respectively, see Figure~\ref{fig:bipolar-inner-candidate}).
Thus
\begin{align*}
  \textq{asymmetry}
  &=
    \frac{a - x_\End}{x_{\sharp 0} - a}
      \\[\tallspace]
  &=
    \frac{
      1 - \sinh v_\End / (\cosh v_\End - \cos(\mp\pi))
    }{
      \sinh v_{\sharp 0} / (\cosh v_{\sharp 0} - \cos 0) - 1
    }
      \\[\tallspace]
  &=
    \frac{\exp v_{\sharp 0} - 1}{\exp v_\End + 1}.
    \yesnumber
    \label{eq:bipolar-inner-candidate-asymmetry}
\end{align*}
The asymmetry will only become significant
if the negative line source has been brought
sufficiently close to the positive one.
From Figure~\ref{fig:bipolar-inner-candidate-asymmetry}
we see that this only occurs towards the colder end of the warm regime.

\begin{figure}
  \centredfigurecontent[width=0.6\textwidth]{%
    bipolar-inner-candidate-asymmetry%
  }{
    Asymmetry~(\ref{eq:bipolar-inner-candidate-asymmetry})
    of the inner candidate boundary,
    as $A$~increases.
    No inner candidate boundary exists
    in the cold regime~($A > A_{\nat 0}$).
  }
\end{figure}

\subsection{Temperature}
\label{sec:bipolar.physical.temperature}

Here we perform a similar analysis
to Section~\ref{sec:polar.physical.temperature},
to determine the physical temperatures that can be realised
by inner candidate boundaries of a prescribed size.
For simplicity,
we take the radius of the circle~$v = v_{\sharp 0}$
(Figure~\ref{fig:bipolar-inner-candidate-with-circle})
as our fixed reference length.
From~(\ref{eq:bipolar-circle-v})
we see that this radius is given by
\begin{equation}
  r_{\sharp 0} = a \csch v_{\sharp 0},
  \label{eq:bipolar-r-sharp-0}
\end{equation}
so that length scale (intrinsic to the bipolar coordinate system) is
\begin{equation}
  a = r_{\sharp 0} \sinh v_{\sharp 0}.
  \label{eq:bipolar-length-scale-in-terms-of-r-sharp-0}
\end{equation}
By construction, the bipolar coordinate~$v_{\sharp 0}$
is a root of the equation~(\ref{eq:bipolar-critical-terminal-point-0})
for critical terminal points along~$u = 0$;
therefore we have
\begin{equation}
  A = \frac{{v_{\sharp 0}}^4}{\cosh v_{\sharp 0} - 1}.
  \label{eq:bipolar-dimensionless-group-in-terms-of-r-sharp-0}
\end{equation}
After rearranging~(\ref{eq:bipolar-dimensionless-group}) for~$T_0$,
we may use~(\ref{eq:bipolar-length-scale-in-terms-of-r-sharp-0})
and~(\ref{eq:bipolar-dimensionless-group-in-terms-of-r-sharp-0})
to obtain
\begin{equation}
  T_0 =
    \roundbr*{
      \frac{
        \cosh v_{\sharp 0} - 1
      }{
        c r_{\sharp 0} {v_{\sharp 0}}^4 \sinh v_{\sharp 0}
      }
    }^{1/3}
  \label{eq:bipolar-temperature-scale-in-terms-of-r-sharp-0}
\end{equation}
for the temperature scale
of the known solution~(\ref{eq:bipolar-laplace-solution}).
It follows that the physical temperature
along the reference circle~$v = v_{\sharp 0}$
is
\begin{align*}
  T_{\sharp 0}
  &= T_0 \cdot v_{\sharp 0} \\
  &=
    \roundbr*{
      \frac{
        \cosh v_{\sharp 0} - 1
      }{
        c r_{\sharp 0} v_{\sharp 0} \sinh v_{\sharp 0}
      }
    }^{1/3}
    \\
  &=
    \roundbr*{
      \frac{\omega (v_{\sharp 0})}{c r_{\sharp 0}}
    }^{1/3},
    \yesnumber
    \label{eq:bipolar-t-sharp-0-in-terms-of-r-sharp-0}
\end{align*}
where $\omega$~is the dimensionless auxiliary function
\begin{equation}
  \omega (v) = \frac{\cosh v - 1}{v \sinh v}
  \label{eq:bipolar-auxiliary-function}
\end{equation}
shown in Figure~\ref{fig:bipolar-auxiliary-function}.

\begin{figure}
  \centredfigurecontent[width=0.55\textwidth]{%
    bipolar-auxiliary-function%
  }{
    Auxiliary function~(\ref{eq:bipolar-auxiliary-function}).
  }
\end{figure}

Now, as the dimensionless group~$A$
runs through the interval~%
  (\ref{eq:bipolar-inner-candidate-existence-interval})
of all inner candidate boundaries,
i.e.~from the limiting extreme~$A = 0$ of the hot regime
up to the warm-to-cold transition~$A = A_{\nat 0} = 9.76206$,
the quantity~$v_{\sharp 0}$ decreases from infinity
down to the transition value~$v_{\nat 0} = 3.83002$;
this can be seen from the left-hand column
of Figure~\ref{fig:bipolar-viable}.
It can be shown that $\omega$~is a decreasing function,
whence we have
\begin{equation}
  \omega (\infty) < \omega (v_{\sharp 0}) \le \omega (v_{\nat 0}).
  \label{eq:bipolar-inner-candidate-omega-interval}
\end{equation}
The upper bound may in fact be simplified to~$1/4$,%
\footnote{
  This would not have happened
  had we chosen $A_\infl$ instead of the special value~$A_{\nat 0}$
  for the upper bound of the interval~%
  (\ref{eq:bipolar-inner-candidate-existence-interval}).
}
so we have
\begin{equation}
  0 < \omega (v_{\sharp 0}) \le 1/4.
  \label{eq:bipolar-inner-candidate-omega-interval-evaluated}
\end{equation}
Therefore,
the unscaled temperature~(\ref{eq:bipolar-t-sharp-0-in-terms-of-r-sharp-0})
spans the interval
\[
  0
    <
  T_{\sharp 0}
    \le
  \roundbr*{\frac{1}{4 c r_{\sharp 0}}}^{1/3},
\]
or
\begin{equation}
  0
    <
  T_{\sharp 0}
    \le
  \roundbr*{\frac{\conduc}{4 \emiss \stefan r_{\sharp 0}}}^{1/3}.
  \label{eq:bipolar-inner-candidate-t-sharp-0-interval}
\end{equation}
It is interesting to note that this interval
perfectly complements the analogous line-source result~%
  (\ref{eq:line-traced-boundary-hot-convex-t-sharp-interval}),
whose lower bound is precisely the upper bound here
(with~$r_\sharp$ in place of~$r_{\sharp 0}$).
Physically, the introduction of a negative image source
has the effect of extending the interval of attainable temperatures
down to absolute zero.
For the PVC parameter values
of Section~\ref{sec:polar.physical.temperature},
the interval~(\ref{eq:bipolar-inner-candidate-t-sharp-0-interval})
evaluates to $\SI{0}{\kelvin} < T_{\sharp 0} \le \SI{277}{\kelvin}$.

\subsection{Power per unit length}
\label{sec:bipolar.physical.power}

% TODO

\section{Summary}
\label{sec:bipolar.summary}

% TODO
