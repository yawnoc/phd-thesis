\chapter{Moderate convex wedges}
\label{ch:moderate}

In this chapter
we apply numerical boundary tracing to analyse corner rounding
in a moderate convex capillary wedge,
for which the wedge half-angle~$\alpha$ and the contact angle~$\gamma$
satisfy the inequality~$\pi/2 - \gamma < \alpha < \pi/2$.
Specifically, we take a numerical solution
to the Laplace--Young equation~(\ref{eq:laplace-young}) in such a wedge
and seek rounded corners along which
the contact condition~(\ref{eq:contact-boundary-condition})
remains satisfied.

At this point an objection will rightfully be raised:
given that a numerical solver is needed
to generate the numerical wedge solution in the first place,
why bother with boundary tracing at all?
Indeed if boundary tracing can only produce
one corner rounding per numerical wedge solution,
we would be better off
directly applying the numerical solver to a rounded wedge!
This objection is addressed
in Section~X, %TODO
where a novel observation is made
which enables multiple corner roundings to be produced by boundary tracing
per numerical wedge solution.

\thematicbreak

For simplicity we take the wedge to be of infinite extent.
The only physical parameter which appears
in the Laplace--Young equation~(\ref{eq:laplace-young})
and in the boundary condition~(\ref{eq:contact-boundary-condition})
is the capillary constant~$\capill$,
which has the dimensions of reciprocal area;
thus the sole length scale of the problem
is the capillary length
\begin{equation}
  L_0 = \frac{1}{\sqrt{\kappa}}.
  \label{eq:capillary-length}
\end{equation}
We therefore put
\begin{align}
  T &\ideq L_0 \scaled{T}, \label{eq:capillary-height-scaling} \\
  x &\ideq L_0 \scaled{x}, \label{eq:capillary-x-scaling} \\
  y &\ideq L_0 \scaled{y}, \label{eq:capillary-y-scaling}
\end{align}
noting that the derivative will transform according to
\begin{equation}
  \del \ideq \scaleddel / L_0.
  \label{eq:capillary-del-scaling}
\end{equation}
Substituting and dropping \scalingmarks,
we obtain the dimensionless Laplace--Young equation
\begin{equation}
  \del \dotp \frac{\del T}{\sqrt{1 + (\del T)^2}} =  T
  \label{eq:scaled-laplace-young}
\end{equation}
within the scaled wedge,
whilst on the boundary we obtain the dimensionless contact condition
\begin{equation}
  \normalvec \dotp \frac{\del T}{\sqrt{1 + (\del T)^2}} = \cos\gamma.
  \label{eq:scaled-contact-boundary-condition}
\end{equation}

\section{Linearised corner rounding revisited}
\label{sec:moderate.linearised}

\section{Rounding of a corner}
\label{sec:moderate.rounding}

\section{Multiple roundings of a corner}
\label{sec:moderate.multiple}
