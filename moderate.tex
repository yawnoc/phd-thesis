\chapter{Moderate convex wedges}
\label{ch:moderate}

In this chapter
we apply numerical boundary tracing to analyse corner rounding
in a moderate convex capillary wedge,
for which the wedge half-angle~$\alpha$ and the contact angle~$\gamma$
satisfy the inequality~$\pi/2 - \gamma < \alpha < \pi/2$.
Specifically, we take a numerical solution
to the Laplace--Young equation~(\ref{eq:laplace-young}) in such a wedge
and seek rounded corners along which
the contact condition~(\ref{eq:contact-boundary-condition})
remains satisfied.

At this point an objection will rightfully be raised:
given that a numerical solver is needed
to generate the numerical wedge solution in the first place,
why bother with boundary tracing at all?
Indeed if boundary tracing can only produce
one corner rounding per numerical wedge solution,
we would be better off
directly applying the numerical solver to a rounded wedge!
This objection is addressed
in Section~\ref{sec:moderate.multiple},
where a novel observation is made
which enables multiple corner roundings to be produced by boundary tracing
per numerical wedge solution.

\thematicbreak

For simplicity we take the wedge to be of infinite extent.
The only physical parameter which appears
in the Laplace--Young equation~(\ref{eq:laplace-young})
and in the boundary condition~(\ref{eq:contact-boundary-condition})
is the capillary constant~$\capill$,
which has the dimensions of reciprocal area;
thus the sole length scale of the problem
is the capillary length
\begin{equation}
  L_0 = \frac{1}{\sqrt{\kappa}}.
  \label{eq:capillary-length}
\end{equation}
We therefore put
\begin{align}
  T &\ideq L_0 \scaled{T}, \label{eq:capillary-height-scaling} \\
  x &\ideq L_0 \scaled{x}, \label{eq:capillary-x-scaling} \\
  y &\ideq L_0 \scaled{y}, \label{eq:capillary-y-scaling}
\end{align}
noting that the derivative will transform according to
\begin{equation}
  \del \ideq \scaleddel / L_0.
  \label{eq:capillary-del-scaling}
\end{equation}
Substituting and dropping \scalingmarks,
we obtain the dimensionless Laplace--Young equation
\begin{equation}
  \del \dotp \frac{\del T}{\sqrt{1 + (\del T)^2}} =  T
  \label{eq:scaled-laplace-young}
\end{equation}
within the scaled wedge,
whilst on the boundary we obtain the dimensionless contact condition
\begin{equation}
  \normalvec \dotp \frac{\del T}{\sqrt{1 + (\del T)^2}} = \cos\gamma.
  \label{eq:scaled-contact-boundary-condition}
\end{equation}

\section{Linearised corner rounding revisited}
\label{sec:moderate.linearised}

In this section we revisit the linearised corner rounding problem
for a right-angled wedge
considered by Anderson~\etal~%
  \cite{anderson-2007-boundary-tracing-ii-applications}.
We do so here to provide ourselves
with an explicit example of corner rounding (via boundary tracing)
in which the wedge solution is known analytically,
\emph{before} proceeding to the fully nonlinear case
where the known wedge solution will be a numerical one.

A similar revisit appeared in Section~2.1 of
the author's Honours thesis~\cite{li-2017-thesis-rounding-capillary-wedge};
there the analysis was conducted purely in polar coordinates,
rather than in the Cartesian coordinates and arc-length parametrisation
used here.

\subsection{Known solution}
\label{sec:moderate.linearised.known}

In the problem under consideration
the wedge is convex and right-angled, i.e.~$\alpha = \pi/4$,
with the contact~$\gamma$ assumed to be close to~$\pi/2$,
so that the capillary problem~(\ref{eq:scaled-laplace-young})
and~(\ref{eq:scaled-contact-boundary-condition})
may be linearised by discarding~$(\del T)^2 \ll 1$.
The subsidiary height scaling
\begin{equation}
  T \ideq U \cos\gamma
  \label{eq:helmholtz-height-scaling}
\end{equation}
removes the contact angle parameter,
leaving us with a linear BVP consisting of
the Helmholtz equation
\begin{equation}
  \del^2 U = U
  \label{eq:scaled-helmholtz}
\end{equation}
in the interior
together with the constant-flux condition
\begin{important}{equation}
  \normalvec \dotp \del U = 1
  \label{eq:scaled-contact-linearised}
\end{important}
on the boundary.

Whereas Anderson~\etal~\cite{anderson-2007-boundary-tracing-ii-applications}
used Cartesian coordinates aligned with the wedge walls,
here we align the positive $x$-axis so that it bisects the wedge
(for a logical extension to arbitrary wedge angles).
Thus the domain is the region~$\abs{y} < x$,
while the boundary consists of the wedge walls~$y = \pm x$ (with~$x \ge 0$),
as shown in Figure~X. % TODO

As observed by Fowkes \&~Hood~\cite[equation~(16)]{
  fowkes-1998-surface-tension-effects-wedge
},
the known solution to the Helmholtz BVP in this wedge is given by
\begin{important}{equation}
  U \ideq \exp \frac{-x+y}{\sqrt{2}} + \exp \frac{-x-y}{\sqrt{2}},
  \label{eq:scaled-helmholtz-solution}
\end{important}
a superposition of capillary layers contributed by each wall.

\subsection{Boundary tracing}
\label{sec:moderate.linearised.tracing}

Comparing the flux condition~(\ref{eq:scaled-contact-linearised})
to the generic one~(\ref{eq:flux-boundary-condition})
(and noting that here we have~$U$ in place of~$T$),
we see that the flux function is simply
\begin{equation}
  F \ideq 1.
  \label{eq:helmholtz-flux-function}
\end{equation}
The derivatives of the known solution are
\begin{align}
  P &\ideq \pder{U}{x} \ideq
    \frac{1}{\sqrt{2}} \roundbr*{
      - \exp \frac{-x+y}{\sqrt{2}} - \exp \frac{-x-y}{\sqrt{2}}
    },
    \label{eq:helmholtz-gradient-u-component} \\[\tallspace]
  Q &\ideq \pder{U}{y} \ideq
    \frac{1}{\sqrt{2}} \roundbr*{
      + \exp \frac{-x+y}{\sqrt{2}} - \exp \frac{-x-y}{\sqrt{2}}
    },
    \label{eq:helmholtz-gradient-v-component}
\end{align}
and so the viability function is
\begin{align*}
  \Phi
  &\ideq (\del U)^2 - F^2 \\
  &\ideq \ee^{-\sqrt{2} (x - y)} + \ee^{-\sqrt{2} (x + y)} - 1.
    \yesnumber
    \label{eq:helmholtz-viability-function}
\end{align*}
Traced boundaries will only exist in the viable domain~$\Phi \ge 0$,
which may be written explicitly as
\begin{equation}
  x \le
    \frac{1}{\sqrt{2}}
    \log \roundbr*{2 \cosh \roundbr[\bulkysize]{\sqrt{2} \cdot y}}.
    \label{eq:helmholtz-viable-domain}
\end{equation}
Its border, the terminal curve~$\Phi = 0$,
is therefore given by
\begin{equation}
  x =
    \frac{1}{\sqrt{2}}
    \log \roundbr*{2 \cosh \roundbr[\bulkysize]{\sqrt{2} \cdot y}},
    \label{eq:helmholtz-terminal-curve}
\end{equation}
with the non-viable domain~$\Phi < 0$ lying strictly to its right
(Figure~X). % TODO
Note how the terminal curve approaches the wedge walls~$y = \pm x$
as one moves away from the line of symmetry~$y = 0$.

The known solution~(\ref{eq:scaled-helmholtz-solution})
is insufficiently simple in form
for the boundary tracing ODE to be analytically solvable
under the coordinate parametrisations~$y = y (x)$ and~$x = x (y)$.
We therefore use the system~(\ref{eq:tracing-ode-arc-length-parametrisation-u})
and~(\ref{eq:tracing-ode-arc-length-parametrisation-v})
for arc-length parametrisation,
which in the present case reduces to
\begin{important}{align}
  \tder{x}{s} &= \frac{-Q \pm P \sqrt{\Phi}}{(\del U)^2},
    \label{eq:helmholtz-tracing-ode-arc-length-parametrisation-x}
    \\[\tallspace]
  \tder{y}{s} &= \frac{+P \pm Q \sqrt{\Phi}}{(\del U)^2}.
    \label{eq:helmholtz-tracing-ode-arc-length-parametrisation-y}
\end{important}
Figure~X % TODO
shows the two branches of traced boundaries obtained
by integrating forward from various points
within the viable domain~(\ref{eq:helmholtz-viable-domain}).
The upper and lower branches have positive and negative slopes respectively.
As expected,
boundary tracing recovers the original boundaries of the Helmholtz BVP\@;
indeed the upper wedge wall~$y = +x$ is a traced boundary of the upper branch,
while the lower wedge wall~$y = -x$ is one of the lower branch.

More interesting of course are the new boundaries obtained;
by construction, these also are curves
along which the boundary condition~(\ref{eq:scaled-contact-linearised}) holds.
More complicated boundaries can be constructed
by patching these curves together in almost arbitrary fashion
(Figure~X); % TODO
the only restriction is that the boundary normal
have consistent orientation.
Since the flux in~(\ref{eq:scaled-contact-linearised}) is positive,
$U$~must be greater on the exterior,
and given the strictly negative $x$-derivative~%
  (\ref{eq:helmholtz-gradient-u-component}),
we may ensure consistent boundary orientation
by always identifying the region to the left as exterior.
Thus, each patching together of boundaries
will mark out a new domain to the right,
which also admits the solution~(\ref{eq:scaled-helmholtz-solution})
to the Helmholtz BVP~(\ref{eq:scaled-helmholtz})
and~(\ref{eq:scaled-contact-linearised}).

\subsection{Corner rounding}
\label{sec:moderate.linearised.rounding}

While the domains produced in Figure~X
are most interesting,
their boundaries contain sharp corners,
incompatible with our goal of constructing a rounded corner.
Recall that at any point \emph{strictly} within the viable domain,
two traced boundaries will cross at a non-zero angle,
forming a sharp corner if they are patched together.
To avoid corners we must therefore only perform patching
along the terminal curve, i.e.~at a terminal point.
Moreover, patching must only be performed at a critical terminal point,
for at an ordinary terminal point,
the two local traced boundaries form a cusp with inconsistent boundary normal
(see Section~\ref{sec:introduction.tracing}).
From Figure~X
we see that there is only one critical terminal point,
located at the intersection between
the terminal curve~(\ref{eq:helmholtz-terminal-curve})
and the line of symmetry~$y = 0$,
\begin{equation}
  (x_0, y_0) = \roundbr*{\frac{\log 2}{\sqrt{2}}, \, 0}.
  \label{eq:helmholtz-critical-terminal-point}
\end{equation}
The local $U$-contour lies on the viable side of the terminal curve;
therefore the critical terminal point is of hyperbolic type,
and two smooth traced boundaries pass through it
(Figure~X). % TODO
The portions to the left of the critical terminal point
eventually collide (at a non-zero angle) with the wedge walls,
so they do not form an acceptable rounding of the corner.
However, the portions to the right
can be shown to asymptotically approach the wedge walls;
patching them together gives the unique corner rounding
for this problem (Figure~X). % TODO
We have therefore constructed a rounded-corner domain
which also admits the known solution~(\ref{eq:scaled-helmholtz-solution})
to the BVP~(\ref{eq:scaled-helmholtz})
and~(\ref{eq:scaled-contact-linearised}).
To assess the effect that such a corner rounding will have
on the height rise,
we simply evaluate the known solution
along the walls of the original sharp-cornered wedge
and along the rounded corner;
the resulting height rise profiles are shown in
Figure~X. % TODO

Of course the discovery of a single rounded corner
hardly completes an analysis of corner rounding.
Anderson~\etal~\cite{anderson-2007-boundary-tracing-ii-applications}
produced different roundings of the corner
by adding Bessel functions
to the known solution~(\ref{eq:scaled-helmholtz-solution})
before applying boundary tracing.
We note that such a superpositioning technique depends on
the linearity of the Helmholtz equation~(\ref{eq:scaled-helmholtz});
in the case of the Laplace--Young equation, which is nonlinear,
a different method is needed to produce multiple roundings of a corner.

\section{Nonlinear corner rounding}
\label{sec:moderate.rounding}

\section{Multiple roundings of a corner}
\label{sec:moderate.multiple}
