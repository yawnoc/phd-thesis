\chapter{Small wedge singularity extraction}
\label{ch:extraction}

Here we apply the change of variable~(\ref{eq:small-wedge-change-of-variable}),
i.e.
\begin{equation}
  \usecontent{eq:small-wedge-change-of-variable},
  \label{eq:small-wedge-change-of-variable-appendix}
\end{equation}
to transform the capillary problem~(\ref{eq:scaled-laplace-young})
\&~(\ref{eq:scaled-contact-boundary-condition})
in a small wedge ($\alpha < \pi/2 - \gamma$)
from a singular BVP in~$T$
to a regular BVP in~$H$.
We recall from Section~\ref{sec:moderate.nonlinear.numerical}
that the capillary problem may be viewed
as the steady-state diffusion problem
\begin{align}
  \del \dotp \squarebr{K \del T} &= T,
    \label{eq:laplace-young-diffusion-appendix}
    \\[\tallspace]
  \normalvec \dotp \squarebr{K \del T} &= \cos\gamma,
  \label{eq:contact-boundary-condition-diffusion-appendix}
\end{align}
where
\begin{equation}
  K = \frac{1}{\sqrt{1 + (\del T)^2}}
  \label{eq:laplace-young-diffusion-coefficient-appendix}
\end{equation}
is the (nonlinear) diffusion coefficient.

Now in Cartesian coordinates,
(\ref{eq:laplace-young-diffusion-appendix})
and~(\ref{eq:contact-boundary-condition-diffusion-appendix})
can simply be represented in matrix form
by the equations
\begin{align}
  \begin{pmatrix}
    \pd / {\pd x} & \pd / {\pd y}
  \end{pmatrix}
  \squarebr*{
    \begin{pmatrix}
      K & 0 \\
      0 & K
    \end{pmatrix}
    \begin{pmatrix}
      \pd / {\pd x} \\
      \pd / {\pd y}
    \end{pmatrix}
    T
  }
    &= T,
    \label{eq:laplace-young-diffusion-matrix-cartesian} \\
  \begin{pmatrix}
    n_x & n_y
  \end{pmatrix}
  \squarebr*{
    \begin{pmatrix}
      K & 0 \\
      0 & K
    \end{pmatrix}
    \begin{pmatrix}
      \pd / {\pd x} \\
      \pd / {\pd y}
    \end{pmatrix}
    T
  }
    &= \cos\gamma.
    \label{eq:contact-boundary-condition-diffusion-matrix-cartesian}
\end{align}
For the purpose of computing~$H$,
we would like to treat~$r$ and~$\phi$ as rectangular coordinates,
which requires rewriting the problem in terms of the formal operator
\begin{equation}
  \matder =
    \begin{pmatrix}
      \pd / {\pd r} \\
      \pd / {\pd\phi}
    \end{pmatrix}.
  \label{eq:small-formally-rectangular-derivative-appendix}
\end{equation}
Since the scale factors of polar coordinates are not both unity,
we cannot simply replace $x$ and~$y$ with $r$ and~$\phi$
in~(\ref{eq:laplace-young-diffusion-matrix-cartesian})
and~(\ref{eq:contact-boundary-condition-diffusion-matrix-cartesian});
the coefficients of the BVP must be adjusted appropriately
to get the equations into an algebraically rectangular form.

\section{Laplace--Young PDE}
\label{sec:extraction.pde}

We first observe that the gradient vector
is given in polar coordinates by
\begin{equation}
  \del T =
    \pder{T}{r} \basisvec{r} + \frac{1}{r} \pder{T}{\phi} \basisvec{\phi}.
  \label{eq:gradient-polar}
\end{equation}
Effecting the transformation~%
  (\ref{eq:small-wedge-change-of-variable-appendix}),
this becomes
\begin{equation}
  \del T =
    \curlybr*{\frac{1}{r} \pder{H}{r} - \frac{H}{r^2}} \basisvec{r}
      +
    \curlybr*{\frac{1}{r^2} \pder{H}{\phi}} \basisvec{\phi}.
  \label{eq:gradient-polar-change-of-variable}
\end{equation}
Multiplying by~$K$ and taking the divergence,
we obtain the left hand side
of the Laplace--Young equation~(\ref{eq:laplace-young-diffusion-appendix}):
\begin{align*}
  \del & \dotp \squarebr{K \del T} \\
  &=
    \frac{1}{r} \squarebr*{
      \pder{}{r} \roundbr*{
        r K \curlybr*{\frac{1}{r} \pder{H}{r} - \frac{H}{r^2}}
      }
        +
      \pder{}{\phi} \roundbr*{
        K \curlybr*{\frac{1}{r^2} \pder{H}{\phi}}
      }
    }
    \\[\tallspace]
  &=
    \frac{1}{r} \squarebr*{
      \pder{}{r} \roundbr*{K \pder{H}{r} - \frac{K}{r} \cdot H}
        +
      \pder{}{\phi} \roundbr*{\frac{K}{r^2} \pder{H}{\phi}}
    }
    \\[\tallspace]
  &=
    \frac{1}{r} \squarebr*{
      \pder{}{r} \roundbr*{K \pder{H}{r}}
        +
      \pder{}{\phi} \roundbr*{\frac{K}{r^2} \pder{H}{\phi}}
        -
      \pder{}{r} \roundbr*{\frac{K}{r} \cdot H}
    }
    \\[\tallspace]
  &=
    \frac{1}{r} \squarebr*{
      \begin{pmatrix}
        \pd / {\pd r} & \pd / {\pd\phi}
      \end{pmatrix}
      \begin{pmatrix}
        K  &  0 \\
        0  &  K / r^2
      \end{pmatrix}
      \begin{pmatrix}
        \pd H / {\pd r} \\
        \pd H / {\pd\phi}
      \end{pmatrix}
        -
      \begin{pmatrix}
        \pd / {\pd r} & \pd / {\pd\phi}
      \end{pmatrix}
      \begin{pmatrix}
        K / r \\
        0
      \end{pmatrix}
      H
    }
    \\[\tallspace]
  &=
    \frac{1}{r}
    \begin{pmatrix}
      \pd / {\pd r} & \pd / {\pd\phi}
    \end{pmatrix}
    \squarebr*{
      \begin{pmatrix}
        K  &  0 \\
        0  &  K / r^2
      \end{pmatrix}
      \begin{pmatrix}
        \pd / {\pd r} \\
        \pd / {\pd\phi}
      \end{pmatrix}
      H
        -
      \begin{pmatrix}
        K / r \\
        0
      \end{pmatrix}
      H
    }
    \\[\tallspace]
  &=
    \frac{1}{r}
    \matder \dotp \squarebr[\bulkysize]{\mat{K} \matder H - \mat{v} H},
\end{align*}
where
\begin{equation}
  \mat{K} =
    \begin{pmatrix}
      K  &  0 \\
      0  &  K / r^2
    \end{pmatrix}
  \label{eq:small-formally-rectangular-diffusion-coefficient-singular}
\end{equation}
and
\begin{equation}
  \mat{v} =
    \begin{pmatrix}
      K / r \\
      0
    \end{pmatrix}.
    \label{eq:small-formally-rectangular-convection-coefficient-singular}
\end{equation}
Since the right hand side
of the Laplace--Young equation~(\ref{eq:laplace-young-diffusion-appendix})
is~$T = H / r$,
we obtain the PDE
\begin{equation}
  \saveequation{
    \matder \dotp \squarebr[\bulkysize]{\mat{K} \matder H - \mat{v} H}
  }{=}{
    H
  }{eq:small-formally-rectangular-laplace-young-appendix}
\end{equation}
in~$H (r, \phi)$.

\section{Contact boundary condition}
\label{sec:extraction.boundary}

First write the normal vector~$\normalvec$ in components,
\begin{equation}
  \normalvec = n_r \basisvec{r} + n_\phi \basisvec{\phi}.
  \label{eq:normal-vector-polar}
\end{equation}
Taking the dot product with
$K$~times the gradient vector~(\ref{eq:gradient-polar-change-of-variable}),
we have
\begin{align*}
  \normalvec & \dotp \squarebr{K \del T} \\
  &=
    n_r K \curlybr*{\frac{1}{r} \pder{H}{r} - \frac{H}{r^2}}
      +
    n_\phi K \curlybr*{\frac{1}{r^2} \pder{H}{\phi}}
    \\[\tallspace]
  &=
    \frac{n_r}{r} \roundbr*{K \pder{H}{r} - \frac{K}{r} \cdot H}
      +
    n_\phi \roundbr*{\frac{K}{r^2} \pder{H}{\phi}}
    \\[\tallspace]
  &=
    \frac{n_r}{r} \roundbr*{K \pder{H}{r}}
      +
    n_\phi \roundbr*{\frac{K}{r^2} \pder{H}{\phi}}
      -
    \frac{n_r}{r} \roundbr*{\frac{K}{r} \cdot H}
    \\[\tallspace]
  &=
    \begin{pmatrix}
      n_r / r  &  n_\phi
    \end{pmatrix}
    \begin{pmatrix}
      K  &  0 \\
      0  &  K / r^2
    \end{pmatrix}
    \begin{pmatrix}
      \pd H / {\pd r} \\
      \pd H / {\pd\phi}
    \end{pmatrix}
      -
    \begin{pmatrix}
      n_r / r  &  n_\phi
    \end{pmatrix}
    \begin{pmatrix}
      K / r \\
      0
    \end{pmatrix}
    H
    \\[\tallspace]
  &=
    \begin{pmatrix}
      n_r / r  &  n_\phi
    \end{pmatrix}
    \squarebr*{
      \begin{pmatrix}
        K  &  0 \\
        0  &  K / r^2
      \end{pmatrix}
      \begin{pmatrix}
        \pd / {\pd r} \\
        \pd / {\pd\phi}
      \end{pmatrix}
      H
        -
      \begin{pmatrix}
        K / r \\
        0
      \end{pmatrix}
      H
    }
    \\[\tallspace]
  &=
    \mat{n} \dotp \squarebr[\bulkysize]{\mat{K} \matder H - \mat{v} H},
\end{align*}
where
\begin{equation}
  \mat{n} =
    \begin{pmatrix}
      n_r / r \\
      n_\phi
    \end{pmatrix}.
  \label{eq:small-formally-rectangular-normal-matrix}
\end{equation}
The contact condition~(\ref{eq:contact-boundary-condition-diffusion-appendix})
therefore becomes
\begin{equation}
  \saveequation{
    \mat{n} \dotp \squarebr[\bulkysize]{\mat{K} \matder H - \mat{v} H}
  }{=}{
    \cos\gamma
  }{eq:small-formally-rectangular-contact-boundary-condition-appendix}.
\end{equation}

\section{Coefficients}
\label{sec:extraction.coefficients}

We have shown that
the transformation~(\ref{eq:small-wedge-change-of-variable-appendix})
converts the capillary problem~(\ref{eq:laplace-young-diffusion-appendix})
\&~(\ref{eq:contact-boundary-condition-diffusion-appendix})
in~$T$
to the BVP~%
  (\ref{eq:small-formally-rectangular-laplace-young-appendix})
\&~%
  (\ref{eq:small-formally-rectangular-contact-boundary-condition-appendix})
in~$H$,
i.e.
\begin{align}
  \useequation{&=}{eq:small-formally-rectangular-laplace-young-appendix},
    \label{eq:small-formally-rectangular-laplace-young-repeat} \\
  \useequation{&=}{%
    eq:small-formally-rectangular-contact-boundary-condition-appendix%
  }.
    \label{eq:small-formally-rectangular-contact-boundary-condition-repeat}
\end{align}
This is indeed of the same form
as the BVP~(\ref{eq:small-formally-rectangular-laplace-young})
\&~(\ref{eq:small-formally-rectangular-contact-boundary-condition});
only it remains to show that the definitions~%
  (\ref{eq:small-formally-rectangular-diffusion-coefficient-singular})
and~%
  (\ref{eq:small-formally-rectangular-convection-coefficient-singular})
given here for the coefficients~$\mat{K}$ and~$\mat{v}$
are indeed equivalent to~%
  (\ref{eq:small-formally-rectangular-diffusion-coefficient})
and~%
  (\ref{eq:small-formally-rectangular-convection-coefficient})
given in Chapter~\ref{ch:small}.

Reading off~(\ref{eq:gradient-polar-change-of-variable}),
the components of the gradient vector in polar coordinates are
\begin{align}
  P &= \frac{1}{r} \pder{H}{r} - \frac{H}{r^2},
    \label{eq:small-gradient-r-component-appendix} \\[\tallspace]
  Q &= \frac{1}{r^2} \pder{H}{\phi}.
    \label{eq:small-gradient-phi-component-appendix}
\end{align}
Noting that these are of order~$1 / r^2$,
we define the de-singularised versions
\begin{align}
  \desing{P} &= r^2 P = r \pder{H}{r} - H,
    \label{eq:small-gradient-de-singularised-r-component-appendix}
    \\[\tallspace]
  \desing{Q} &= r^2 Q = \pder{H}{\phi},
    \label{eq:small-gradient-de-singularised-phi-component-appendix}
\end{align}
as given in~(\ref{eq:small-gradient-de-singularised-r-component})
and~(\ref{eq:small-gradient-de-singularised-phi-component}).
We then have
\begin{equation}
  (\del T)^2
    = P^2 + Q^2
    = \frac{\desing{P}^2 + \desing{Q}^2}{r^4},
  \label{eq:small-gradient-squared-appendix}
\end{equation}
and hence
\begin{align*}
  K
  &=
    \frac{1}{\sqrt{1 + (\desing{P}^2 + \desing{Q}^2) / r^4}}
    \\[\tallspace]
  &=
    \frac{r^2}{\sqrt{r^4 + \desing{P}^2 + \desing{Q}^2}}
    \\[\tallspace]
  &=
    r^2 C,
    \yesnumber
    \label{eq:small-laplace-young-diffusion-coefficient-appendix}
\end{align*}
where
\begin{equation}
  \usecontent{eq:small-formally-rectangular-scalar-coefficient},
  \label{eq:small-formally-rectangular-scalar-coefficient-appendix}
\end{equation}
as in~(\ref{eq:small-formally-rectangular-scalar-coefficient}).
Therefore,
(\ref{eq:small-formally-rectangular-diffusion-coefficient-singular})
and~(\ref{eq:small-formally-rectangular-convection-coefficient-singular})
become
\begin{equation}
  \useequation{=}{eq:small-formally-rectangular-diffusion-coefficient}
  \label{eq:small-formally-rectangular-diffusion-coefficient-appendix}
\end{equation}
and
\begin{equation}
  \mat{v} =
    \begin{pmatrix}
      r C \\
      0
    \end{pmatrix},
  \label{eq:small-formally-rectangular-convection-coefficient-appendix}
\end{equation}
as stated in~(\ref{eq:small-formally-rectangular-diffusion-coefficient})
and~(\ref{eq:small-formally-rectangular-convection-coefficient}).
