\chapter{Small wedge singularity extraction}
\label{ch:extraction}

Here we apply the change of variable~(\ref{eq:small-wedge-change-of-variable}),
i.e.
\begin{equation}
  T (r, \phi) \ideq \frac{H (r, \phi)}{r},
  \label{eq:small-wedge-change-of-variable-appendix}
\end{equation}
to transform the capillary problem~(\ref{eq:scaled-laplace-young})
\&~(\ref{eq:scaled-contact-boundary-condition})
in a small wedge ($\alpha < \pi/2 - \gamma$)
from a singular BVP in~$T$
to a regular BVP in~$H$.
We recall from Section~\ref{sec:moderate.nonlinear.numerical}
that the capillary problem may be viewed
as the steady-state diffusion problem
\begin{align}
  \del \dotp \squarebr{K \del T} &= T,
    \label{eq:laplace-young-diffusion-appendix}
    \\[\tallspace]
  \normalvec \dotp \squarebr{K \del T} &= \cos\gamma,
  \label{eq:contact-boundary-condition-diffusion-appendix}
\end{align}
where
\begin{equation}
  K \ideq \frac{1}{\sqrt{1 + (\del T)^2}}
  \label{eq:laplace-young-diffusion-coefficient-appendix}
\end{equation}
is the (nonlinear) diffusion coefficient.

Now in Cartesian coordinates,
(\ref{eq:laplace-young-diffusion-appendix})
and~(\ref{eq:contact-boundary-condition-diffusion-appendix})
can simply be represented in matrix form
by the equations
\begin{align}
  \begin{pmatrix}
    \pd / {\pd x} & \pd / {\pd y}
  \end{pmatrix}
  \squarebr*{
    \begin{pmatrix}
      K & 0 \\
      0 & K
    \end{pmatrix}
    \begin{pmatrix}
      \pd / {\pd x} \\
      \pd / {\pd y}
    \end{pmatrix}
    T
  }
    &= T,
    \label{eq:laplace-young-diffusion-matrix-cartesian} \\
  \begin{pmatrix}
    n_x & n_y
  \end{pmatrix}
  \squarebr*{
    \begin{pmatrix}
      K & 0 \\
      0 & K
    \end{pmatrix}
    \begin{pmatrix}
      \pd / {\pd x} \\
      \pd / {\pd y}
    \end{pmatrix}
    T
  }
    &= \cos\gamma.
    \label{eq:contact-boundary-condition-diffusion-matrix-cartesian}
\end{align}
For the purpose of computing~$H$,
we would like to treat~$r$ and~$\phi$ as rectangular coordinates,
which requires rewriting the problem in terms of the formal operator
\begin{equation}
  \matder \ideq
    \begin{pmatrix}
      \pd / {\pd r} \\
      \pd / {\pd\phi}
    \end{pmatrix}.
  \label{eq:small-formally-rectangular-derivative-appendix}
\end{equation}
Since the scale factors of polar coordinates are not both unity,
we cannot simply replace $x$ and~$y$ with $r$ and~$\phi$
in~(\ref{eq:laplace-young-diffusion-matrix-cartesian})
and~(\ref{eq:contact-boundary-condition-diffusion-matrix-cartesian});
the coefficients of the BVP must be adjusted appropriately
to get the equations into an algebraically rectangular form.
