\chapter{Small wedge singularity extraction}
\label{ch:extraction}

Here we apply the change of variable~(\ref{eq:small-wedge-change-of-variable}),
i.e.
\begin{equation}
  T (r, \phi) \ideq \frac{H (r, \phi)}{r},
  \label{eq:small-wedge-change-of-variable-appendix}
\end{equation}
to transform the capillary problem~(\ref{eq:scaled-laplace-young})
\&~(\ref{eq:scaled-contact-boundary-condition})
in a small wedge ($\alpha < \pi/2 - \gamma$)
from a singular BVP in~$T$
to a regular BVP in~$H$.
We recall from Section~\ref{sec:moderate.nonlinear.numerical}
that the capillary problem may be viewed
as the steady-state diffusion problem
\begin{align}
  \del \dotp \squarebr{K \del T} &= T,
    \label{eq:laplace-young-diffusion-appendix}
    \\[\tallspace]
  \normalvec \dotp \squarebr{K \del T} &= \cos\gamma,
  \label{eq:contact-boundary-condition-diffusion-appendix}
\end{align}
where
\begin{equation}
  K \ideq \frac{1}{\sqrt{1 + (\del T)^2}}
  \label{eq:laplace-young-diffusion-coefficient-appendix}
\end{equation}
is the (nonlinear) diffusion coefficient.
