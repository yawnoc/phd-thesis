\chapter{Self-viewing radiation calculations}
\label{ch:self-viewing}

\section{Single concave fin}
\label{sec:self-viewing.fin}

\begin{figure}
  \usecontent{fig:self-viewing-radiation-elements}
  \label{fig:self-viewing-radiation-elements-appendix}
\end{figure}

\begin{figure}
  \usecontent{fig:self-viewing-radiation-elements-fin}
  \label{fig:self-viewing-radiation-elements-fin-appendix}
\end{figure}

Here we obtain an explicit integral expression
for the ratio~(\ref{eq:radiation-self-viewing-ratio}),
i.e.
\begin{equation}
  \usecontent{eq:radiation-self-viewing-ratio},
  \label{eq:radiation-self-viewing-ratio-appendix}
\end{equation}
in the case of a single concave fin,
shaped as~$y = y (x)$ on~$x_1 \le x \le x_2$.
This ratio provides a dimensionless measure
of the amount of self-viewing radiation
that has been neglected during boundary tracing.
For convenience, we have included here
repeats of Figures~\ref{fig:self-viewing-radiation-elements}
and~\ref{fig:self-viewing-radiation-elements-fin},
in the form of Figure~\ref{fig:self-viewing-radiation-elements-appendix}
and Figure~\ref{fig:self-viewing-radiation-elements-fin-appendix}.

The displacement between the area elements~$\td A$ and~$\td A^\star$ is
by definition
\begin{equation}
  \vec{d}^\star
    =
      \positionvec - \positionvec^\star \\
    =
      (x - x^\star) \basisvec{x}
        +
      (y - y^\star) \basisvec{y}
        +
      (z - z^\star) \basisvec{z},
  \label{eq:radiation-self-viewing-displacement-appendix}
\end{equation}
while the normal to the fin surface~$y = y (x)$ is given by
\begin{equation}
  \normalvec =
    \frac{
      -y' \basisvec{x} + \basisvec{y}
    }{
      \sqrt{1 + {y'}^2}
    },
  \label{eq:radiation-self-viewing-normal-vector}
\end{equation}
where primes denote $x$-differentiation.

From the geometry of
Figure~\ref{fig:self-viewing-radiation-elements-appendix},
we have
\begin{align*}
  \frac{\cos\theta^\star \cos\theta}{{d^\star}^2}
    &=
      \frac{
        (\normalvec^\star \dotp \vec{d}^\star)
        (-{\normalvec} \dotp \vec{d}^\star)
      }{
        {d^\star}^4
      }
        \\[\tallspace]
    &=
      \smallerfrac{
        \squarebr[\big]{-(x - x^\star) {y'}^\star + (y - y^\star)}
        \squarebr[\big]{(x - x^\star) y' - (y - y^\star)}
      }{
        \squarebr[\big]{
          (x - x^\star)^2 + (y - y^\star)^2 + (z - z^\star)
        }^2
        \sqrt{1 + {{y'}^\star}^2}
        \sqrt{1 + {y'}^2}
      },
        \yesnumber
        \label{eq:radiation-self-viewing-cosine-expression}
\end{align*}
and from Figure~\ref{fig:self-viewing-radiation-elements-fin-appendix},
we see that
\begin{equation}
  \td A^\star
    = \td s^\star \td z^\star
    = \sqrt{1 + {{y'}^\star}^2} \td x^\star \td z^\star,
  \label{eq:radiation-self-viewing-differential-area}
\end{equation}
with $x_1 < x^\star < x_2$ and~$-\infty < z^\star < \infty$
being the appropriate bounds of integration.

Using~(\ref{eq:radiation-self-viewing-cosine-expression})
and~(\ref{eq:radiation-self-viewing-differential-area}),
the ratio~(\ref{eq:radiation-self-viewing-ratio-appendix}) becomes
\begin{equation}
  R =
    \frac{1}{T^4}
    \int_{-\infty}^{\infty}
    \int_{x_1}^{x_2}
      \smallerfrac{
        {T^\star}^4
        \squarebr[\big]{-(x - x^\star) {y'}^\star + (y - y^\star)}
        \squarebr[\big]{(x - x^\star) y' - (y - y^\star)}
        \td x^\star
        \td z^\star
      }{
        \pi
        \squarebr[\big]{
          (x - x^\star)^2 + (y - y^\star)^2 + (z - z^\star)
        }^2
        \sqrt{1 + {y'}^2}
      },
  \label{eq:radiation-self-viewing-ratio-double-integral}
\end{equation}
and after evaluating the outer integral with respect to~$z^\star$,
we obtain
\begin{equation}
  R =
    \frac{1}{T^4}
    \int_{x_1}^{x_2}
      \smallerfrac{
        {T^\star}^4
        \squarebr[\big]{-(x - x^\star) {y'}^\star + (y - y^\star)}
        \squarebr[\big]{(x - x^\star) y' - (y - y^\star)}
        \td x^\star
      }{
        2
        \squarebr[\big]{(x - x^\star)^2 + (y - y^\star)^2}^{3/2}
        \sqrt{1 + {y'}^2}
      },
  \label{eq:plane-self-viewing-ratio-appendix}
\end{equation}
which is~(\ref{eq:plane-self-viewing-ratio}).
