\chapter{Small wedges}
\label{ch:small}

In this chapter
we apply numerical boundary tracing to analyse corner rounding
in small capillary wedges,
for which the wedge half-angle~$\alpha$ and the contact angle~$\gamma$
satisfy the inequality~$\alpha < \pi/2 - \gamma$
and the height rise in the corner is infinite.

The author's Honours thesis~%
  \cite{li-2017-thesis-rounding-capillary-wedge}
considered an asymptotic version of boundary tracing
in which the leading term~(\ref{eq:small-wedge-asymptotic-solution})
was used as the known solution.
Although that analysis was able to produce
a corresponding asymptotic corner rounding,
one could not be certain that the rounding actually lay
within the region of validity
of the asymptotic known solution~(\ref{eq:small-wedge-asymptotic-solution}).
The accuracy of the asymptotic corner rounding
could only be assessed indirectly
by computing a numerical solution in the corresponding rounded-corner domain
(which no longer had a singularity)
and then comparing it
to the asymptotic solution~(\ref{eq:small-wedge-asymptotic-solution}).

It is more preferable to avoid the use of an asymptotic approximation,
so that the results of boundary tracing are applicable throughout the wedge,
rather than only in a near-corner region of unknown size.
Therefore, we determine here a numerical solution
to the scaled capillary BVP~(\ref{eq:scaled-laplace-young})
\&~(\ref{eq:scaled-contact-boundary-condition})
and apply boundary tracing to it,
instead of to an asymptotic solution.
