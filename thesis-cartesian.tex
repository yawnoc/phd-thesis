\chapter{Cartesian coordinates}
\label{ch:cartesian}

In this chapter,
I apply boundary tracing to the thermal radiation problem
by taking known solutions
to Laplace's equation~(\ref{eq:laplace-steady-conduction})
in Cartesian coordinates,
and seeking new boundaries along which
the radiation boundary condition~(\ref{eq:radiation-boundary-condition})
is satisfied.

\section{Slab solution}
\label{sec:cartesian.slab}

In Cartesian coordinates~$(x, y)$,
the simplest non-constant solution
to Laplace's equation~(\ref{eq:laplace-steady-conduction})
is
\begin{equation}
  T = h_0 x,
  \label{eq:laplace-solution-slab}
\end{equation}
corresponding to one-dimensional steady conduction
with constant temperature gradient~$h_0$.
In the context of thermal radiation,
the temperature~$T$ is to be reckoned on an absolute scale;
therefore, the region~$x < 0$ is unphysical and to be ignored,
as the temperature therein is negative.

\subsection{Scaling}
\label{sec:cartesian.slab.scaling}

While the known solution~(\ref{eq:laplace-solution-slab}) is, by itself,
scale-invariant with respect to both temperature and length,
its coupling with
the radiation boundary condition~(\ref{eq:radiation-boundary-condition})
will lead to a characteristic temperature scale~$\tau$
and length scale~$\lambda$.

Defining
\begin{align}
  \scaled{T} = T / \tau, \label{eq:slab-scaled-temperature} \\
  \scaled{x} = x / \lambda, \label{eq:slab-scaled-x} \\
  \scaled{y} = y / \lambda, \label{eq:slab-scaled-y}
\end{align}
and noting that
\begin{equation}
  \scaleddel = \lambda \del,
  \label{eq:slab-scaled-del}
\end{equation}
the radiation boundary condition~(\ref{eq:radiation-boundary-condition})
and the known solution~(\ref{eq:laplace-solution-slab})
become
\begin{align}
  \normalvec \dotp \scaleddel \scaled{T}
    &= -\squarebr*{c \lambda \tau^3} \scaled{T}^4, \\
  \scaled{T}
    &= \squarebr*{\frac{h_0 \lambda}{\tau}} \scaled{x}.
\end{align}
The two dimensionless groups are eliminated by choosing the scales
\begin{align}
  \tau &= \roundbr*{\frac{h_0}{c}}^{1/4},
    \label{eq:slab-temperature-scale} \\
  \lambda &= \roundbr*{\frac{1}{c {h_0}^3}}^{1/4}.
    \label{eq:slab-length-scale}
\end{align}

\section{Others}

\tbd. Both of the following if time permits:
\begin{enumerate}
  \item
    Something of the form
    \[
      T = \const - \textq{trigonometric in~$x$} \textq{exponential in~$y$}
    \]
    such that $T = \const > 0$~along some vertical line~$x = \const$.
    Hopefully this produces convex shapes.
  \item
    The unit square with Dirichlet sides at~$\const, 0, 0, 0$.
    Then extend this to different aspect ratios.
    (Exact solution requires infinite series,
    which is crap near the $T = \const$~edge.
    Computing the derivatives (required for boundary tracing)
    may also be an issue.
\end{enumerate}
