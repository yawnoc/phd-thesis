\chapter{Cartesian coordinates}
\label{ch:cartesian}

In this chapter,
I apply boundary tracing to the thermal radiation problem
by taking known solutions
to Laplace's equation~(\ref{eq:laplace-steady-conduction})
in Cartesian coordinates
and seeking new boundaries along which
the radiation condition~(\ref{eq:radiation-boundary-condition})
is satisfied.

\section{Plane-source solution}
\label{sec:cartesian.plane}

First I consider the simplest non-constant solution
to Laplace's equation~(\ref{eq:laplace-steady-conduction})
in Cartesian coordinates,
\begin{important}{equation}
  T = h_0 x,
  \label{eq:plane-laplace-solution}
\end{important}
corresponding to one-dimensional steady conduction
with constant temperature gradient~$h_0$.
Such would be the equilibrium temperature profile in a slab
with one face held at a fixed temperature
and the other radiating into vacuum,
and the aim of boundary tracing
is to look for other (more interesting) radiation boundaries
which correspond to this same solution.

In the context of thermal radiation,
the temperature~$T$ is to be reckoned on an absolute scale;
therefore, the region~$x < 0$ is unphysical and to be ignored,
as the temperature therein is negative.

\subsection{Scaling}
\label{sec:cartesian.plane.scaling}

While the known solution~(\ref{eq:plane-laplace-solution}) is, by itself,
scale-invariant with respect to both temperature and length,
its coupling with
the radiation boundary condition~(\ref{eq:radiation-boundary-condition})
will determine characteristic temperature and length scales
$\tau$ and~$\lambda$.
Here, scaling is used to remove these parameters
and reduce the problem to its simplest form.

Defining
\begin{align}
  \scaled{T} &= T / \tau, \label{eq:plane-scaled-temperature} \\
  \scaled{x} &= x / \lambda, \label{eq:plane-scaled-x} \\
  \scaled{y} &= y / \lambda, \label{eq:plane-scaled-y}
\end{align}
and noting that
\begin{equation}
  \scaleddel = \lambda \del,
  \label{eq:plane-scaled-del}
\end{equation}
the radiation boundary condition~(\ref{eq:radiation-boundary-condition})
and the known solution~(\ref{eq:plane-laplace-solution})
become
\begin{align}
  \normalvec \dotp \scaleddel \scaled{T}
    &= -\group{c \lambda \tau^3} \scaled{T}^4,
    \label{eq:plane-scaled-radiation-boundary-condition-with-groups}
    \\[\tallspace]
  \scaled{T}
    &= \group{\frac{h_0 \lambda}{\tau}} \scaled{x}.
    \label{eq:plane-scaled-laplace-solution-with-groups}
\end{align}
Setting the two dimensionless groups to unity
results in the temperature and length scales
\begin{align}
  \tau &= \roundbr*{\frac{h_0}{c}}^{1/4},
    \label{eq:plane-temperature-scale} \\[\tallspace]
  \lambda &= \roundbr*{\frac{1}{c {h_0}^3}}^{1/4}.
    \label{eq:plane-length-scale}
\end{align}
These scales may also be arrived at
by seeking the straight-line boundary~$x = \const$
and its corresponding temperature~$T = \const$
along which the radiation condition~(\ref{eq:radiation-boundary-condition})
is satisfied.
Since the temperature gradient
for the known solution~(\ref{eq:plane-laplace-solution})
is $h_0$~everywhere,
there must hold~$h_0 = c T^4$ along this boundary,
and the scales~(\ref{eq:plane-temperature-scale})
and~(\ref{eq:plane-length-scale}) follow immediately.

Of course the straight-line boundary~$x = \lambda$
(or equivalently, $\scaled{x} = 1$)
is rather boring,
and it is through boundary tracing
that more interesting boundaries may be generated,
as will be shown in the next section.

\subsection{Boundary tracing}
\label{sec:cartesian.plane.tracing}

For brevity, \atten{drop all \scalingaccents},
so that the scaled boundary condition~%
  (\ref{eq:plane-scaled-radiation-boundary-condition-with-groups})
and known solution~(\ref{eq:plane-scaled-laplace-solution-with-groups})
become
\begin{important}{align}
  \normalvec \dotp \del T &= -T^4,
    \label{eq:plane-scaled-radiation-boundary-condition} \\
  T &= x.
    \label{eq:plane-scaled-laplace-solution}
\end{important}
The scale factors for Cartesian coordinates are trivial
($\scalefac[x] = \scalefac[y] = 1$),
leading to the following abbreviatory quantities
from Section~\ref{sec:curvilinear.calculus.abbreviations}:
\begin{align}
  P &= \pder{T}{x} = 1,
    \label{eq:plane-gradient-u-component} \\[\tallspace]
  Q &= \pder{T}{y} = 0.
    \label{eq:plane-gradient-v-component}
\end{align}
Comparing the radiation boundary condition~%
  (\ref{eq:plane-scaled-radiation-boundary-condition})
to the generic one~(\ref{eq:flux-boundary-condition}),
the flux function is
\begin{align*}
  F
  &= - T^4 \\
  &= -x^4,
    \yesnumber
    \label{eq:plane-flux-function}
\end{align*}
and it follows that the viability function is
\begin{align*}
  \Phi
  &= (\del T)^2 - F^2 \\
  &= 1 - x^8.
    \yesnumber
    \label{eq:plane-viability-function}
\end{align*}
The viable domain~$\Phi \ge 0$ (excluding the unphysical region~$x < 0$)
is therefore given by the infinite strip
\begin{equation}
  0 \le x \le 1.
  \label{eq:plane-viable-domain}
\end{equation}
The terminal curve~$\Phi = 0$ is also the $T$-contour~$x = 1$
(the boring traced boundary
mentioned in Section~\ref{sec:cartesian.plane.scaling}).
Using the terminology of Section~\ref{sec:introduction.tracing},
$x = 1$~is therefore a critical terminal curve,
and other traced boundaries will attach to it smoothly.

\begin{figure}
  \centredfigurecontent[width=0.5\textwidth]{%
    plane-traced-boundaries%
  }{
    Traced boundaries~(\ref{eq:plane-traced-boundary}).
  }
\end{figure}

The boundary tracing ODE~(\ref{eq:tracing-ode-coordinate-parametrisation-v})
becomes
\begin{important}{equation}
  \tder{y}{x} = \mp \frac{x^4}{\sqrt{1 - x^8}},
  \label{eq:plane-tracing-ode-coordinate-parametrisation-y}
\end{important}
which integrates to give traced boundaries of the form
\begin{important}{equation}
  y =
  \const
    \mp
  \frac{x^5}{5}
    \cdot
  \hypergeo \roundbr*{\frac{1}{2}, \frac{5}{8}; \frac{13}{8}; x^8},
  \label{eq:plane-traced-boundary}
\end{important}
shown in Figure~\ref{fig:plane-traced-boundaries},
where $\hypergeo$~is the hypergeometric function~%
\cite{olver-2010-nist-handbook-mathematical-functions}.
The translational symmetry in the $y$-direction
is a property inherited from the ODE~%
  (\ref{eq:plane-tracing-ode-coordinate-parametrisation-y}).
A local analysis near~$x = 1$ shows that
\begin{equation}
  y = \const \pm \sqrt{\frac{1 - x}{2}} + \order (1 - x)^{3/2},
  \label{eq:plane-traced-boundary-x-near-1}
\end{equation}
so that the traced boundaries do indeed attach smoothly
onto the critical terminal curve~$x = 1$, as expected.
Near~$x = 0$ each pair of traced boundaries forms a thin cusp of the form
\begin{equation}
  y = \const \mp \frac{x^5}{5} + \order \roundbr*{x^{13}}.
  \label{eq:plane-traced-boundary-x-near-0}
\end{equation}
Now each of the traced boundaries is a curve along which
the radiation boundary condition~%
  (\ref{eq:plane-scaled-radiation-boundary-condition})
is satisfied.
More complicated boundaries can be constructed
by patching together several of these curves, or portions thereof,
and the only requirement is that there be consistent orientation.
This requirement is satisfied by indentifying as the interior
the side on which $T$ (which equals~$x$) is greater,
i.e.~the side to the right of each curve.
Figure~\ref{fig:plane-traced-boundaries-patched} shows
a sample of the broad variety of radiation boundaries
which can be produced in this manner.

\begin{figure}
  \centredfigurecontent[width=\textwidth]{%
    plane-traced-boundaries-patched%
  }{
    Radiation boundaries (black solid) patched together
    using the traced boundaries~(\ref{eq:plane-traced-boundary}) (grey).
    The critical terminal curve~$x = 1$ (black dashed)
    is shown for reference.
  }
\end{figure}

\begin{figure}
  \centredfigurecontent[width=\textwidth]{%
    plane-domains%
  }{
    Five domains marked out by radiation boundaries (black solid)
    and constant-temperature boundaries (black dotted).
    The critical terminal curve~$x = 1$ (grey dashed)
    is shown for reference.
  }
\end{figure}

\subsection{Domain construction}
\label{sec:cartesian.plane.domain}

Since the constructed radiation boundaries only dissipate heat,
a domain for the steady conduction--radiation BVP
will not be completely specified
until there is also a boundary to supply it.
The simplest boundary condition which can supply heat
is the Dirichlet condition~$T = \const$,
and given the form of the known solution~%
  (\ref{eq:plane-scaled-laplace-solution}),
these boundaries are simply vertical lines~$x = \const$.

An infinite number of conduction--radiation domains
may therefore be marked out
by joining a constructed radiation boundary
with an appropriate Dirichlet boundary~$x = \const$,
as in Figure~\ref{fig:plane-domains}.
Each of these domains corresponds to steady conduction in the interior,
constant temperature along the right hand side,
and thermal radiation into vacuum to the left.
Most surprising is that \emph{all} of these domains
admit the \emph{same} exact solution~(\ref{eq:plane-scaled-laplace-solution}).

Unfortunately,
the domains produced here by boundary tracing are not convex,
but rather \defin{self-viewing}:
some of the outgoing radiation travels not to infinity,
but strikes another part of the boundary,
where it might be partially or fully absorbed.
Since the simple radiation boundary condition~%
  (\ref{eq:plane-scaled-radiation-boundary-condition})
does not account for this,
the results of this section,
while mathematically sound,
are not physically valid.

\section{Cosinusoidal solution}
\label{sec:cartesian.cosine}

While boundary tracing as performed on
the plane-source solution~(\ref{eq:plane-laplace-solution})
did not yield convex conduction--radiation domains,
one might expect more favourable outcomes
by starting from solutions to Laplace's equation
which are not one-dimensional.
A simple class of such solutions consists of functions which are the product
of a trigonometric function in~$x$ and an hyperbolic function in~$y$.

As seen in Section~\ref{sec:cartesian.plane.domain}
one eventually requires a boundary to supply the heat being radiated away,
and from a practical point of view it is
straight-line, constant-temperature boundaries which are of interest.
Therefore, in this section I consider known solutions of the form
\begin{important}{equation}
  T = T_0 \roundbr*{1 - B \cos\frac{x}{L_0} \cosh\frac{y}{L_0}},
  \label{eq:cosine-laplace-solution}
\end{important}
where $T_0$~is a temperature, $L_0$~is a length scale,
and $B$~is a dimensionless constant.
The temperature is constant at~$T = T_0$
along the straight line~$x = \pi/2 \cdot L_0$,
which shall ultimately serve as the heat-supplying boundary.

\subsection{Scaling}
\label{sec:cartesian.cosine.scaling}

First the physical parameters are scaled out to the extent possible.
Using the same scalings as Section~\ref{sec:cartesian.plane.scaling},
i.e.~(\ref{eq:plane-scaled-temperature}) through~(\ref{eq:plane-scaled-del}),
the radiation boundary condition~(\ref{eq:radiation-boundary-condition})
and the known solution~(\ref{eq:cosine-laplace-solution})
become
\begin{align}
  \normalvec \dotp \scaleddel \scaled{T}
    &= -\group{c \lambda \tau^3} \scaled{T}^4,
    \label{eq:cosine-scaled-radiation-boundary-condition-with-groups}
    \\[\tallspace]
  \scaled{T}
    &=
      \group{\frac{T_0}{\tau}}
      \roundbr*{
        1 -
          B
          \cos \roundbr*{\group{\frac{\lambda}{L_0}} \scaled{x}}
          \cosh \roundbr*{\group{\frac{\lambda}{L_0}} \scaled{y}}
      },
    \label{eq:cosine-scaled-laplace-solution-with-groups}
\end{align}
where there are two free scales, $\tau$~(temperature) and $\lambda$~(length).
Since there are three unique dimensionless groups,
one of the groups cannot be eliminated,
and to keep the cosinusoidal terms as simple as possible,
the obvious scales
\begin{align}
  \tau &= T_0,
    \label{eq:cosine-temperature-scale} \\
  \lambda &= L_0,
    \label{eq:cosine-length-scale}
\end{align}
are chosen.
Defining the dimensionless group
\begin{equation}
  A = \frac{1}{c L_0 {T_0}^3}.
  \label{eq:cosine-dimensionless-group}
\end{equation}
and \atten{dropping \scalingaccents},
the scaled boundary condition~%
  (\ref{eq:cosine-scaled-radiation-boundary-condition-with-groups})
and known solution~(\ref{eq:cosine-scaled-laplace-solution-with-groups})
become
\begin{important}{align}
  \normalvec \dotp \del T &= -\frac{T^4}{A},
    \label{eq:cosine-scaled-radiation-boundary-condition} \\[\tallspace]
  T &= 1 - B \cos x \cosh y.
    \label{eq:cosine-scaled-laplace-solution}
\end{important}

\begin{figure}
  \centredfigurecontent[width=\textwidth]{%
    cosine-physical%
  }{
    Unphysical region (black shaded) and contours (thin grey)
    for the known solution~(\ref{eq:cosine-scaled-laplace-solution}).
    In each plot, the left edge is~$x = 0$
    and the vertical line (grey) is~$x = \pi/2$,
    which is also the contour~$T = 1$.
    The central plot corresponds to~$B = 1$.
  }
\end{figure}

\subsection{Physical region}
\label{sec:cartesian.cosine.physical}

Since it is the straight-line boundary~$x = \pi/2$
which ultimately shall, in the form of a Dirichlet condition~$T = 1$,
supply the heat to be conducted and radiated away,
only the side on which~$T \le 1$ is relevant,
namely the side~$x \le \pi/2$.
Given that the known solution~(\ref{eq:cosine-scaled-laplace-solution})
is an even function of~$x$,
it also suffices to consider~$x \ge 0$ only.

Thus the region of interest
is the vertical strip~$0 \le x \le \pi/2$.
Not all of this strip is physical
as only positive temperatures are admissible
in thermal radiation problems,
and since the known solution~(\ref{eq:cosine-scaled-laplace-solution})
vanishes along~$\cos x \cosh y = 1 / B$,
the geometry of the physical region~$T \ge 0$
depends on the value of the dimensionless constant~$B$
(Figure~\ref{fig:cosine-physical}).

\subsection{Boundary tracing equations}
\label{sec:cartesian.cosine.tracing}

In addition to only considering the physical region~$T \ge 0$,
traced boundaries will only exist within the viable domain.
For the boundary condition~%
  (\ref{eq:cosine-scaled-radiation-boundary-condition})
and known solution~(\ref{eq:cosine-scaled-laplace-solution}) at hand,
one obtains the derivatives
\begin{align}
  P &= \pder{T}{x} = \+B \sin x \cosh y,
    \label{eq:cosine-gradient-u-component} \\[\tallspace]
  Q &= \pder{T}{y} = -B \cos x \sinh y,
    \label{eq:cosine-gradient-v-component}
\end{align}
the flux function
\begin{equation}
  F = -\frac{(1 - B \cos x \cosh y) ^ 4}{A},
  \label{eq:cosine-flux-function}
\end{equation}
and the viability function
\begin{align*}
  \Phi
  &= (\del T)^2 - F^2 \\
  &=
    B^2 \roundbr*{\sin^2 x + \sinh^2 y}
      -
    \frac{(1 - B \cos x \cosh y) ^ 8}{A^2}.
    \yesnumber
    \label{eq:cosine-viability-function}
\end{align*}
The geometry of the viable domain~$\Phi \ge 0$
therefore depends on both of the dimensionless parameters~$A$ and~$B$,
as does the boundary tracing ODE~%
  (\ref{eq:tracing-ode-coordinate-parametrisation-u}),
which becomes
\begin{important}{equation}
  \tder{x}{y} = \frac{P Q \mp F \sqrt{\Phi}}{Q^2 - F^2}.
  \label{eq:cosine-tracing-ode-coordinate-parametrisation-x}
\end{important}
Given the complexity of managing more than one parameter,
I consider~$B = 1$ before examining the general case.

\subsection{Simple case (\texorpdfstring{$B = 1$}{B = 1})}
\label{sec:cartesian.cosine.simple}

\subsubsection{Viable domain}
\label{sec:cartesian.cosine.simple.viable}

In the $B = 1$~case,
the known solution~(\ref{eq:cosine-scaled-laplace-solution}) simplifies to
\begin{equation}
  T = 1 - \cos x \cosh y
  \label{eq:cosine-simple-laplace-solution}
\end{equation}
and the viability function~(\ref{eq:cosine-viability-function}) reduces to
\begin{equation}
  \Phi = \sin^2 x + \sinh^2 y - \frac{(1 - \cos x \cosh y) ^ 8}{A^2}.
  \label{eq:cosine-simple-viability-function}
\end{equation}
Figure~\ref{fig:cosine_simple-physical-viable} shows
the dependence of the non-viable domain%
\footnote{
  There are also non-viable regions within the unphysical region,
  but these are omitted from Figure~\ref{fig:cosine_simple-physical-viable}
  and the remainder of the analysis because they are irrelevant.
}
on the dimensionless group~$A$.
As $A$~increases, the non-viable domain recedes towards the right,
with its boundary the terminal curve
passing through the vertical line~$x = \pi/2$ when~$A = 1$.
Radiation boundaries as produced by boundary tracing will only be found
in the white region of Figure~\ref{fig:cosine_simple-physical-viable},
which is both physical~($T \ge 0$) and viable~($\Phi \ge 0$).

\begin{figure}
  \centredfigurecontent[width=\textwidth]{%
    cosine_simple-physical-viable%
  }{
    Unphysical region (black shaded), contours (thin grey),
    terminal curve (black dashed), and non-viable domain (grey shaded)
    for the known solution~(\ref{eq:cosine-simple-laplace-solution})
    and viability function~(\ref{eq:cosine-simple-viability-function}).
    In each plot, the left edge is~$x = 0$
    and the vertical line (grey) is~$x = \pi/2$,
    which is also the contour~$T = 1$.
    The central plot corresponds to~$A = 1$.
  }
\end{figure}

\subsubsection{Boundary tracing}
\label{sec:cartesian.cosine.simple.tracing}

The tracing ODE~%
  (\ref{eq:cosine-tracing-ode-coordinate-parametrisation-x})
cannot be integrated analytically,
so the traced boundaries are determined numerically.
In doing so, the parametrisation~$x = x (y)$
is unable to handle traced boundaries which are locally horizontal;
such difficulty is avoided by instead using the arc-length parametrisation~%
  (\ref{eq:tracing-ode-arc-length-parametrisation-u})
and~(\ref{eq:tracing-ode-arc-length-parametrisation-v}),
which for Cartesian coordinates reduces to
\begin{align}
  \tder{x}{s} &= \frac{-Q F \pm P \sqrt{\Phi}}{(\del T)^2},
    \label{eq:cosine-tracing-ode-arc-length-parametrisation-x} \\[\tallspace]
  \tder{y}{s} &= \frac{\+P F \pm Q \sqrt{\Phi}}{(\del T)^2}.
    \label{eq:cosine-tracing-ode-arc-length-parametrisation-y}
\end{align}
Two branches of traced boundaries are obtained
by integrating forward
from starting points within the region both physical and viable.
The resulting curves (Figure~\ref{fig:cosine_simple-traced-boundaries})
are boundaries along which the radiation boundary condition~%
  (\ref{eq:cosine-scaled-radiation-boundary-condition})
is satisfied.
For the correct boundary orientation,
note that the temperature~$T$ is an increasing function of~$x$;
it follows that the interior lies on the right of each boundary curve,
with heat being lost by radiation to the left.

\begin{figure}
  \centredfigurecontent[width=0.5\textwidth]{%
    cosine_simple-traced-boundaries%
  }{
    Traced boundaries (black) for~$A = 0.5$, obtained by integrating
    (\ref{eq:cosine-tracing-ode-arc-length-parametrisation-x})
    and~(\ref{eq:cosine-tracing-ode-arc-length-parametrisation-y}).
  }
\end{figure}

By patching together the traced boundaries,
an unlimited number of radiation boundaries may be constructed.
However, almost all such constructions are non-convex
(see Figure~\ref{fig:cosine_simple-traced-boundaries-patched-spiky}).
At any point strictly within the viable domain,
(i.e.~in the viable domain but not along its boundary the terminal curve,)
two traced boundaries cross at a non-zero angle,
and patching these two boundaries together
inevitably results in a spike made of non-convex curves.
As noted already in Section~\ref{sec:cartesian.plane.domain},
self-viewing radiation is not accounted for
by the simple radiation boundary condition~%
  (\ref{eq:cosine-scaled-radiation-boundary-condition}),
and so the non-convex constructions are all invalid.

\begin{figure}
  \newcommand*{\subfigurewidth}{0.36\textwidth}
  \centering
    \hspace*{\fill}
  \begin{subfigure}[t]{\subfigurewidth}
    \centredfigurecontent{cosine_simple-traced-boundaries-patched-spiky}{%
      Spiky non-convex boundary
    }
  \end{subfigure}
    \hfill
  \begin{subfigure}[t]{\subfigurewidth}
    \centredfigurecontent{cosine_simple-traced-boundaries-patched-smooth}{%
      Smooth boundary without spikes
    }
  \end{subfigure}
    \hspace*{\fill}
  \caption{
    Radiation boundaries (black) patched together
    using traced boundaries (grey).
  }
  \label{fig:cosine_simple-traced-boundaries-patched}
\end{figure}

\begin{figure}
  \centredfigurecontent[width=0.36\textwidth]{%
    cosine_simple-terminal-points%
  }{
    Contours (thin grey) which intersect the terminal curve (black dashed).
    (Horizontal scale is exaggerated.)
  }
\end{figure}

To avoid the non-convex spikes,
the join must occur at a point along the terminal curve.
Moreover this point must not be an ordinary terminal point,
at which the two traced boundaries being patched
would form a cusp which has inconsistent boundary orientation
(see Section~\ref{sec:introduction.tracing}).
Instead the join must occur at a critical terminal point,
which, recall, is a point along the terminal curve
for which the local $T$-contour touches the terminal curve tangentially.

Figure~\ref{fig:cosine_simple-terminal-points} shows that in the present case,
all points along the terminal curve are ordinary
except for a single critical terminal point~$(x, y) = (x_0, 0)$
at the intersection between the terminal curve
and the horizontal axis~$y = 0$.
Algebraically, $x_0$~is the positive solution to
\begin{equation}
  \eval[\big]{\Phi}_{y=0}
  = \roundbr*{1 - \cos^2 x} - \frac{(1 - \cos x)^8}{A^2}
  = 0,
  \label{eq:cosine-simple-critical-terminal-point-x}
\end{equation}
a polynomial equation in~$\cos x$.
Furthermore, the local $T$-contour through~$(x_0, 0)$
lies on the viable side of the terminal curve;
therefore the critical terminal point~$(x_0, 0)$ is of hyperbolic type,
with two traced boundaries passing through it smoothly.
By patching together the portions which lie to the right of~$(x_0, 0)$,
a single spike-free radiation boundary can be constructed
as shown in Figure~\ref{fig:cosine_simple-traced-boundaries-patched-smooth};
this boundary shall be referred to as the \defin{candidate boundary}.
It is interesting to note that
the candidate boundary and the terminal curve are virtually indistinguishable;
despite only touching each other at the critical terminal point~$(x_0, 0)$,
the former asymptotically approaches the latter travelling away from~$y = 0$.

\subsubsection{Convexity}
\label{sec:cartesian.cosine.simple.convexity}

The choice of the candidate boundary over all other boundary patchings
is a necessary but not sufficient condition for convexity.
While the candidate boundary is convex at~$(x_0, 0)$,
it eventually inflects
at some~$(x, y) = (x_\infl, \pm y_\infl)$ depending on~$A$,
and becomes concave.

Recalling the goal of constructing domains
corresponding to a conduction--radiation BVP\@,
a \defin{candidate domain} is marked out by using
the candidate boundary (the radiation boundary)
and the straight line~$x = \pi/2$ (the heat-supplying boundary)
for demarcation.
The continuum of candidate domains (for~$0 < A < 1$)\footnote{
  At~$A = 1$ the candidate boundary touches the line~$x = \pi/2$
  and the candidate domain shrinks to a point.
}
is shown in Figure~X; % TODO
each domain is shaped like a thin lens,
corresponding to steady conduction in its interior,
thermal radiation along the curved boundary to the left,
and constant temperature~$T = 1$ along the straight boundary on the right.
The domain will be valid if the curved boundary
(a portion of the corresponding candidate boundary)
is convex, or equivalently,
if the first inflection of the corresponding candidate boundary
occurs at abscissa~$x_\infl \ge \pi/2$.

The critical value~$A = A_\infl$
(where the candidate domain changes from invalid to valid)
is that for which the first inflection occurs at precisely~$x_\infl = \pi/2$.
Given that the candidate boundary is never horizontal,
$A_\infl$~is best sought using the coordinate parametrisation~$x = x (y)$
for the candidate boundary.
In Cartesian coordinates
the boundary curvature is simply the second derivative,
or, using primes for $y$-differentiation, $x''$.
Differentiating the first derivative~%
  (\ref{eq:cosine-tracing-ode-coordinate-parametrisation-x})
gives
\begin{equation}
  x'' = \tder{}{y} \roundbr*{\frac{P Q \mp F \sqrt{\Phi}}{Q^2 - F^2}},
  \label{eq:curvature-cartesian-by-y}
\end{equation}
whose right hand side may be reduced to a function purely of~$x$ and~$y$
by applying~(\ref{eq:cosine-tracing-ode-coordinate-parametrisation-x})
once more to eliminate the first derivatives.
Substituting~$x = x_\infl = \pi/2$, this becomes
\begin{equation}
  \eval[\big]{x''}_{x=\pi/2} =
    % TODO
  \label{eq:cosine-traced-boundary-curvature-cartesian-by-y-x-at-pi-on-2}
\end{equation}

\subsection{General case (\texorpdfstring{$B$~arbitrary}{B arbitrary})}
\label{sec:cartesian.cosine.general}
