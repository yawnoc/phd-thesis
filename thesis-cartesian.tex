\chapter{Cartesian coordinates}
\label{ch:cartesian}

In this chapter,
I apply boundary tracing to the thermal radiation problem
by taking known solutions
to Laplace's equation~(\ref{eq:laplace-steady-conduction})
in Cartesian coordinates,
and seeking new boundaries along which
the radiation boundary condition~(\ref{eq:radiation-boundary-condition})
is satisfied.

\section{Slab solution}
\label{sec:cartesian.slab}

In Cartesian coordinates~$(x, y)$,
the simplest non-constant solution
to Laplace's equation~(\ref{eq:laplace-steady-conduction})
is
\begin{equation}
  T = h_0 x,
  \label{eq:laplace-solution-slab}
\end{equation}
corresponding to one-dimensional steady conduction
with constant temperature gradient~$h_0$.
In the context of thermal radiation,
the temperature~$T$ is to be reckoned on an absolute scale;
therefore, the region~$x < 0$ is unphysical and to be ignored,
as the temperature becomes negative therein.

\section{Others}

\tbd. Both of the following if time permits:
\begin{enumerate}
  \item
    Something of the form
    \[
      T = \const - \textq{trigonometric in~$x$} \textq{exponential in~$y$}
    \]
    such that $T = \const > 0$~along some vertical line~$x = \const$.
    Hopefully this produces convex shapes.
  \item
    The unit square with Dirichlet sides at~$\const, 0, 0, 0$.
    Then extend this to different aspect ratios.
    (Exact solution requires infinite series,
    which is crap near the $T = \const$~edge.
    Computing the derivatives (required for boundary tracing)
    may also be an issue.
\end{enumerate}
