\chapter{Cartesian coordinates}
\label{ch:cartesian}

In this chapter,
I apply boundary tracing to the thermal radiation problem,
by taking known solutions
to Laplace's equation~(\ref{eq:laplace-steady-conduction})
in Cartesian coordinates
and seeking new boundaries along which
the radiation condition~(\ref{eq:radiation-boundary-condition})
is satisfied.

\section{Slab solution}
\label{sec:cartesian.slab}

First I consider the simplest non-constant solution
to Laplace's equation~(\ref{eq:laplace-steady-conduction})
in Cartesian coordinates,
\begin{important}{equation}
  T = h_0 x,
  \label{eq:slab-laplace-solution}
\end{important}
corresponding to one-dimensional steady conduction
with constant temperature gradient~$h_0$.
In the context of thermal radiation,
the temperature~$T$ is to be reckoned on an absolute scale;
therefore, the region~$x < 0$ is unphysical and to be ignored,
as the temperature therein is negative.

\subsection{Scaling}
\label{sec:cartesian.slab.scaling}

While the known solution~(\ref{eq:slab-laplace-solution}) is, by itself,
scale-invariant with respect to both temperature and length,
its coupling with
the radiation boundary condition~(\ref{eq:radiation-boundary-condition})
will lead to a characteristic temperature scale~$\tau$
and length scale~$\lambda$.

Defining
\begin{align}
  \scaled{T} &= T / \tau, \label{eq:slab-scaled-temperature} \\
  \scaled{x} &= x / \lambda, \label{eq:slab-scaled-x} \\
  \scaled{y} &= y / \lambda, \label{eq:slab-scaled-y}
\end{align}
and noting that
\begin{equation}
  \scaleddel = \lambda \del,
  \label{eq:slab-scaled-del}
\end{equation}
the radiation boundary condition~(\ref{eq:radiation-boundary-condition})
and the known solution~(\ref{eq:slab-laplace-solution})
become
\begin{align}
  \normalvec \dotp \scaleddel \scaled{T}
    &= -\squarebr*{c \lambda \tau^3} \scaled{T}^4,
    \label{eq:slab-scaled-radiation-boundary-condition-with-groups}
    \\[\fraclinespace]
  \scaled{T}
    &= \squarebr*{\frac{h_0 \lambda}{\tau}} \scaled{x}.
    \label{eq:slab-scaled-laplace-solution-with-groups}
\end{align}
Setting the two dimensionless groups to unity
results in the temperature and length scales
\begin{align}
  \tau &= \roundbr*{\frac{h_0}{c}}^{1/4},
    \label{eq:slab-temperature-scale} \\[\fraclinespace]
  \lambda &= \roundbr*{\frac{1}{c {h_0}^3}}^{1/4}.
    \label{eq:slab-length-scale}
\end{align}
These scales may also be arrived at
by seeking the straight-line boundary~$x = \const$
and its corresponding temperature~$T = \const$
along which the radiation condition~(\ref{eq:radiation-boundary-condition})
is satisfied.
Since the temperature gradient
for the known solution~(\ref{eq:slab-laplace-solution})
is $h_0$~everywhere,
there must hold~$h_0 = c T^4$ along this boundary,
and the scales~(\ref{eq:slab-temperature-scale})
and~(\ref{eq:slab-length-scale}) follow immediately.

Of course the straight-line boundary~$x = \lambda$
(or equivalently, $\scaled{x} = 1$)
is rather boring,
and it is through boundary tracing
that more interesting boundaries may be generated,
as will be shown in the next section.

\subsection{Boundary tracing}
\label{sec:cartesian.slab.tracing}

For brevity, \strong{drop all \scalingaccents},
so that the scaled boundary condition~%
  (\ref{eq:slab-scaled-radiation-boundary-condition-with-groups})
and known solution~(\ref{eq:slab-scaled-laplace-solution-with-groups})
become
\begin{important}{align}
  \normalvec \dotp \del T &= -T^4,
    \label{eq:slab-scaled-radiation-boundary-condition} \\
  T &= x.
    \label{eq:slab-scaled-laplace-solution}
\end{important}
The scale factors for Cartesian coordinates are trivial
($\scalefac[x] = \scalefac[y] = 1$),
leading to the following abbreviatory quantities
from Section~\ref{sec:curvilinear.calculus.abbreviations}:
\begin{align}
  P &= \pder{T}{x} = 1,
    \label{eq:slab-gradient-u-component} \\[\fraclinespace]
  Q &= \pder{T}{y} = 0.
    \label{eq:slab-gradient-v-component}
\end{align}
Comparing the radiation boundary condition~%
  (\ref{eq:slab-scaled-radiation-boundary-condition})
to the generic one~(\ref{eq:flux-boundary-condition}),
the flux function is
\begin{align*}
  F
  &= - T^4 \\
  &= -x^4,
    \yesnumber
    \label{eq:slab-flux-function}
\end{align*}
and it follows that the viability function is
\begin{align*}
  \Phi
  &= (\del T)^2 - F^2 \\
  &= 1 - x^8.
    \yesnumber
    \label{eq:slab-viability-function}
\end{align*}
The viable domain~$\Phi \ge 0$ (excluding the unphysical region~$x < 0$)
is therefore given by the infinite strip
\begin{equation}
  0 \le x \le 1.
  \label{eq:slab-viable-domain}
\end{equation}
The terminal curve~$\Phi = 0$ is also the $T$-contour~$x = 1$
(the boring traced boundary
mentioned in Section~\ref{sec:cartesian.slab.scaling}).
Using the terminology of Section~\ref{sec:introduction.tracing},
$x = 1$~is a critical terminal curve,
and other traced boundaries will attach to it smoothly.

The boundary tracing ODE~(\ref{eq:tracing-ode-coordinate-parametrisation-v})
becomes
\begin{important}{equation}
  \tder{y}{x} = \mp \frac{x^4}{\sqrt{1 - x^8}},
  \label{eq:slab-tracing-ode-coordinate-parametrisation-y}
\end{important}
which integrates to give traced boundaries of the form
\begin{important}{equation}
  y =
  \const
    \mp
  \frac{x^5}{5}
    \cdot
  \hypergeo \roundbr*{\frac{1}{2}, \frac{5}{8}; \frac{13}{8}; x^8},
  \label{eq:slab-traced-boundary}
\end{important}
where $\hypergeo$~is the hypergeometric function
(see~\tbd).
These curves indeed attach smoothly
onto the critical terminal curve~$x = 1$;
a local analysis near~$x = 1$ reveals the asymptotic relationship
\begin{equation}
  y = \const \pm \sqrt{\frac{1 - x}{2}} + \order (1 - x)^{3/2}.
  \label{eq:slab-traced-boundary-x-near-1}
\end{equation}
Near~$x = 0$ there holds
\begin{equation}
  y = \const \mp \frac{x^5}{5} + \order \roundbr*{x^{13}},
  \label{eq:slab-traced-boundary-x-near-0}
\end{equation}
and for each choice of integration constant
the corresponding pair of traced boundaries
will form a very thin spike (Figure~\tbd),
with half-width
\begin{equation}
  \frac{1}{5}
    \cdot
  \hypergeo \roundbr*{\frac{1}{2}, \frac{5}{8}; \frac{13}{8}; 1}
  = \frac{\sqrt{\pi} \gamm (13/8)}{5 \gamm (9/8)}
  = 0.3375.
  \label{eq:slab-traced-boundary-spike-half-width}
\end{equation}
By piecing together several of these spikes at various separation distances,
a broad variety of boundaries may be produced;
some examples are shown in Figure~\tbd.
Moreover, each and every one of these boundaries marks out a domain
admitting the same exact solution~(\ref{eq:slab-scaled-laplace-solution}),
to the BVP consisting of steady conduction in the interior,
fixed temperature~$T = T_\bath$ along the vertical line~$x = x_\bath$
with~$T_\bath = x_\bath$,
and thermal radiation~(\ref{eq:slab-scaled-radiation-boundary-condition})
into vacuum along the remainder of the boundary.

Unfortunately,
the domains produced here by boundary tracing are not convex,
but rather \emph{self-viewing}:
some of the outgoing radiation travels not to infinity,
but strikes another part of the boundary,
where it might be partially or fully absorbed.
Since the simple radiation boundary condition~%
  (\ref{eq:slab-scaled-radiation-boundary-condition})
does not account for this,
the results of this section,
while mathematically sound,
are not physically valid.

\section{Others}

\tbd. Both of the following if time permits:
\begin{enumerate}
  \item
    Something of the form
    \[
      T = \const - \textq{trigonometric in~$x$} \textq{exponential in~$y$}
    \]
    such that $T = \const > 0$~along some vertical line~$x = \const$.
    Hopefully this produces convex shapes.
  \item
    The unit square with Dirichlet sides at~$(\const, 0, 0, 0)$.
    Then extend this to different aspect ratios.
    (Exact solution requires infinite series,
    which is crap near the $T = \const$~edge.
    Computing the derivatives (required for boundary tracing)
    may also be an issue.)
\end{enumerate}
