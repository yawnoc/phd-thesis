\chapter{Cartesian coordinates}
\label{ch:cartesian}

In this chapter,
I apply boundary tracing to the thermal radiation problem
by taking known solutions
to Laplace's equation~(\ref{eq:laplace-steady-conduction})
in Cartesian coordinates
and seeking new boundaries along which
the radiation condition~(\ref{eq:radiation-boundary-condition})
is satisfied.

\section{Plane-source solution}
\label{sec:cartesian.plane}

First I consider the simplest non-constant solution
to Laplace's equation~(\ref{eq:laplace-steady-conduction})
in Cartesian coordinates,
\begin{important}{equation}
  T = h_0 x,
  \label{eq:plane-laplace-solution}
\end{important}
corresponding to one-dimensional steady conduction
with constant temperature gradient~$h_0$.
In the context of thermal radiation,
the temperature~$T$ is to be reckoned on an absolute scale;
therefore, the region~$x < 0$ is unphysical and to be ignored,
as the temperature therein is negative.

\subsection{Scaling}
\label{sec:cartesian.plane.scaling}

While the known solution~(\ref{eq:plane-laplace-solution}) is, by itself,
scale-invariant with respect to both temperature and length,
its coupling with
the radiation boundary condition~(\ref{eq:radiation-boundary-condition})
will lead to a characteristic temperature scale~$\tau$
and length scale~$\lambda$.

Defining
\begin{align}
  \scaled{T} &= T / \tau, \label{eq:plane-scaled-temperature} \\
  \scaled{x} &= x / \lambda, \label{eq:plane-scaled-x} \\
  \scaled{y} &= y / \lambda, \label{eq:plane-scaled-y}
\end{align}
and noting that
\begin{equation}
  \scaleddel = \lambda \del,
  \label{eq:plane-scaled-del}
\end{equation}
the radiation boundary condition~(\ref{eq:radiation-boundary-condition})
and the known solution~(\ref{eq:plane-laplace-solution})
become
\begin{align}
  \normalvec \dotp \scaleddel \scaled{T}
    &= -\squarebr*{c \lambda \tau^3} \scaled{T}^4,
    \label{eq:plane-scaled-radiation-boundary-condition-with-groups}
    \\[\fraclinespace]
  \scaled{T}
    &= \squarebr*{\frac{h_0 \lambda}{\tau}} \scaled{x}.
    \label{eq:plane-scaled-laplace-solution-with-groups}
\end{align}
Setting the two dimensionless groups to unity
results in the temperature and length scales
\begin{align}
  \tau &= \roundbr*{\frac{h_0}{c}}^{1/4},
    \label{eq:plane-temperature-scale} \\[\fraclinespace]
  \lambda &= \roundbr*{\frac{1}{c {h_0}^3}}^{1/4}.
    \label{eq:plane-length-scale}
\end{align}
These scales may also be arrived at
by seeking the straight-line boundary~$x = \const$
and its corresponding temperature~$T = \const$
along which the radiation condition~(\ref{eq:radiation-boundary-condition})
is satisfied.
Since the temperature gradient
for the known solution~(\ref{eq:plane-laplace-solution})
is $h_0$~everywhere,
there must hold~$h_0 = c T^4$ along this boundary,
and the scales~(\ref{eq:plane-temperature-scale})
and~(\ref{eq:plane-length-scale}) follow immediately.

Of course the straight-line boundary~$x = \lambda$
(or equivalently, $\scaled{x} = 1$)
is rather boring,
and it is through boundary tracing
that more interesting boundaries may be generated,
as will be shown in the next section.

\subsection{Boundary tracing}
\label{sec:cartesian.plane.tracing}

For brevity, \atten{drop all \scalingaccents},
so that the scaled boundary condition~%
  (\ref{eq:plane-scaled-radiation-boundary-condition-with-groups})
and known solution~(\ref{eq:plane-scaled-laplace-solution-with-groups})
become
\begin{important}{align}
  \normalvec \dotp \del T &= -T^4,
    \label{eq:plane-scaled-radiation-boundary-condition} \\
  T &= x.
    \label{eq:plane-scaled-laplace-solution}
\end{important}
The scale factors for Cartesian coordinates are trivial
($\scalefac[x] = \scalefac[y] = 1$),
leading to the following abbreviatory quantities
from Section~\ref{sec:curvilinear.calculus.abbreviations}:
\begin{align}
  P &= \pder{T}{x} = 1,
    \label{eq:plane-gradient-u-component} \\[\fraclinespace]
  Q &= \pder{T}{y} = 0.
    \label{eq:plane-gradient-v-component}
\end{align}
Comparing the radiation boundary condition~%
  (\ref{eq:plane-scaled-radiation-boundary-condition})
to the generic one~(\ref{eq:flux-boundary-condition}),
the flux function is
\begin{align*}
  F
  &= - T^4 \\
  &= -x^4,
    \yesnumber
    \label{eq:plane-flux-function}
\end{align*}
and it follows that the viability function is
\begin{align*}
  \Phi
  &= (\del T)^2 - F^2 \\
  &= 1 - x^8.
    \yesnumber
    \label{eq:plane-viability-function}
\end{align*}
The viable domain~$\Phi \ge 0$ (excluding the unphysical region~$x < 0$)
is therefore given by the infinite strip
\begin{equation}
  0 \le x \le 1.
  \label{eq:plane-viable-domain}
\end{equation}
The terminal curve~$\Phi = 0$ is also the $T$-contour~$x = 1$
(the boring traced boundary
mentioned in Section~\ref{sec:cartesian.plane.scaling}).
Using the terminology of Section~\ref{sec:introduction.tracing},
$x = 1$~is therefore a critical terminal curve,
and other traced boundaries will attach to it smoothly.

\begin{figure}
  \centredfigurecontent[width=0.5\textwidth]{%
    plane-traced-boundaries%
  }{
    Traced boundaries~(\ref{eq:plane-traced-boundary}).
  }
\end{figure}

The boundary tracing ODE~(\ref{eq:tracing-ode-coordinate-parametrisation-v})
becomes
\begin{important}{equation}
  \tder{y}{x} = \mp \frac{x^4}{\sqrt{1 - x^8}},
  \label{eq:plane-tracing-ode-coordinate-parametrisation-y}
\end{important}
which integrates to give traced boundaries of the form
\begin{important}{equation}
  y =
  \const
    \mp
  \frac{x^5}{5}
    \cdot
  \hypergeo \roundbr*{\frac{1}{2}, \frac{5}{8}; \frac{13}{8}; x^8},
  \label{eq:plane-traced-boundary}
\end{important}
shown in Figure~\ref{fig:plane-traced-boundaries},
where $\hypergeo$~is the hypergeometric function~%
\cite{olver-2010-nist-handbook-mathematical-functions}.
A local analysis near~$x = 1$ shows that
\begin{equation}
  y = \const \pm \sqrt{\frac{1 - x}{2}} + \order (1 - x)^{3/2},
  \label{eq:plane-traced-boundary-x-near-1}
\end{equation}
so that the traced boundaries do indeed attach smoothly
onto the critical terminal curve~$x = 1$, as expected.
Near~$x = 0$ each pair of traced boundaries forms a thin cusp of the form
\begin{equation}
  y = \const \mp \frac{x^5}{5} + \order \roundbr*{x^{13}}.
  \label{eq:plane-traced-boundary-x-near-0}
\end{equation}
Now each of the traced boundaries is a curve along which
the radiation boundary condition~%
  (\ref{eq:plane-scaled-radiation-boundary-condition})
is satisfied.
More complicated boundaries can be constructed
by patching together several of these curves, or portions thereof,
and the only requirement is that there be consistent orientation.
This requirement is satisfied by indentifying as the interior
the side on which $T$ (which equals~$x$) is greater,
i.e.~the side to the right of each curve.
Figure~TODO shows a sample of the broad variety of radiation boundaries
which can be produced in this manner.

Since the constructed radiation boundaries only dissipate heat,
a domain for the steady conduction--radiation BVP
will not be completely specified
until there is also a boundary to supply it.
The simplest boundary condition which can supply heat
is the Dirichlet condition~$T = \const$,
and given the form of the known solution~%
  (\ref{eq:plane-scaled-laplace-solution}),
these boundaries are simply vertical lines~$x = \const$.

An infinite number of conduction--radiation domains
may therefore be marked out
by joining a constructed radiation boundary
with an appropriate Dirichlet boundary~$x = \const$,
as in Figure~TODO.
Each of these domains corresponds to steady conduction in the interior,
constant temperature along the right hand side
and thermal radiation into vacuum to the left.
Most surprising is that \emph{all} of these domains
admit the \emph{same} exact solution~(\ref{eq:plane-scaled-laplace-solution}).

Unfortunately,
the domains produced here by boundary tracing are not convex,
but rather \defin{self-viewing}:
some of the outgoing radiation travels not to infinity,
but strikes another part of the boundary,
where it might be partially or fully absorbed.
Since the simple radiation boundary condition~%
  (\ref{eq:plane-scaled-radiation-boundary-condition})
does not account for this,
the results of this section,
while mathematically sound,
are not physically valid.

\section{Others}

TODO: Both of the following if time permits:
\begin{enumerate}
  \item
    Something of the form
    \[
      T = \const - \textq{trigonometric in~$x$} \textq{exponential in~$y$}
    \]
    such that $T = \const > 0$~along some vertical line~$x = \const$.
    Hopefully this produces convex shapes.
  \item
    The unit square with Dirichlet sides at~$(\const, 0, 0, 0)$.
    Then extend this to different aspect ratios.
    (Exact solution requires infinite series,
    which is crap near the $T = \const$~edge.
    Computing the derivatives (required for boundary tracing)
    may also be an issue.)
\end{enumerate}
