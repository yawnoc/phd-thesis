\chapter{Cartesian coordinates}
\label{ch:cartesian}

In this chapter,
I apply boundary tracing to the thermal radiation problem
by taking known solutions
to Laplace's equation~(\ref{eq:laplace-steady-conduction})
in Cartesian coordinates
and seeking new boundaries along which
the radiation condition~(\ref{eq:radiation-boundary-condition})
is satisfied.

\section{Plane-source solution}
\label{sec:cartesian.plane}

First I consider the simplest non-constant solution
to Laplace's equation~(\ref{eq:laplace-steady-conduction})
in Cartesian coordinates,
\begin{important}{equation}
  T \ideq h_0 x,
  \label{eq:plane-laplace-solution}
\end{important}
corresponding to one-dimensional steady conduction
with constant temperature gradient~$h_0$.
Such would be the equilibrium temperature profile in a slab
with one face held at a fixed temperature
and the other radiating into vacuum,
and the aim of boundary tracing
is to look for other (more interesting) radiation boundaries
which correspond to this same solution.

In the context of thermal radiation,
the temperature~$T$ is to be reckoned on an absolute scale;
therefore, the region~$x < 0$ is unphysical and to be ignored,
as the temperature therein is negative.

\subsection{Scaling}
\label{sec:cartesian.plane.scaling}

While the known solution~(\ref{eq:plane-laplace-solution}) is, by itself,
scale-invariant with respect to both temperature and length,
its coupling with
the radiation boundary condition~(\ref{eq:radiation-boundary-condition})
will determine characteristic temperature and length scales
$\tau$ and~$\lambda$.
Here, scaling is used to remove these parameters
and reduce the problem to its simplest form.

Defining
\begin{align}
  \scaled{T} &\defeq T / \tau, \label{eq:plane-scaled-temperature} \\
  \scaled{x} &\defeq x / \lambda, \label{eq:plane-scaled-x} \\
  \scaled{y} &\defeq y / \lambda, \label{eq:plane-scaled-y}
\end{align}
and noting that
\begin{equation}
  \scaleddel \ideq \lambda \del,
  \label{eq:plane-scaled-del}
\end{equation}
the radiation boundary condition~(\ref{eq:radiation-boundary-condition})
and the known solution~(\ref{eq:plane-laplace-solution})
become
\begin{align}
  \normalvec \dotp \scaleddel \scaled{T}
    &= -\group{c \lambda \tau^3} \scaled{T}^4,
    \label{eq:plane-scaled-radiation-boundary-condition-with-groups}
    \\[\tallspace]
  \scaled{T}
    &\ideq \group{\frac{h_0 \lambda}{\tau}} \scaled{x}.
    \label{eq:plane-scaled-laplace-solution-with-groups}
\end{align}
Setting the two dimensionless groups to unity
results in the temperature and length scales
\begin{align}
  \tau &= \roundbr*{\frac{h_0}{c}}^{1/4},
    \label{eq:plane-temperature-scale} \\[\tallspace]
  \lambda &= \roundbr*{\frac{1}{c {h_0}^3}}^{1/4}.
    \label{eq:plane-length-scale}
\end{align}
These scales may also be arrived at
by seeking the straight-line boundary~$x = \const$
and its corresponding temperature~$T = \const$
along which the radiation condition~(\ref{eq:radiation-boundary-condition})
is satisfied.
Since the temperature gradient
for the known solution~(\ref{eq:plane-laplace-solution})
is $h_0$~everywhere,
there must hold~$h_0 = c T^4$ along this boundary,
and the scales~(\ref{eq:plane-temperature-scale})
and~(\ref{eq:plane-length-scale}) follow immediately.

Of course the straight-line boundary~$x = \lambda$
(or equivalently, $\scaled{x} = 1$)
is rather boring,
and it is through boundary tracing
that more interesting boundaries may be generated,
as will be shown in the next section.

\subsection{Boundary tracing}
\label{sec:cartesian.plane.tracing}

For brevity, \atten{drop all \scalingmarks},
so that the scaled boundary condition~%
  (\ref{eq:plane-scaled-radiation-boundary-condition-with-groups})
and known solution~(\ref{eq:plane-scaled-laplace-solution-with-groups})
become
\begin{important}{align}
  \normalvec \dotp \del T &= -T^4,
    \label{eq:plane-scaled-radiation-boundary-condition} \\
  T &\ideq x.
    \label{eq:plane-scaled-laplace-solution}
\end{important}
The scale factors for Cartesian coordinates are trivial
($\scalefac[x] \ideq \scalefac[y] \ideq 1$),
leading to the following abbreviatory quantities
from Section~\ref{sec:curvilinear.calculus.abbreviations}:
\begin{align}
  P &\ideq \pder{T}{x} \ideq 1,
    \label{eq:plane-gradient-u-component} \\[\tallspace]
  Q &\ideq \pder{T}{y} \ideq 0.
    \label{eq:plane-gradient-v-component}
\end{align}
Comparing the radiation boundary condition~%
  (\ref{eq:plane-scaled-radiation-boundary-condition})
to the generic one~(\ref{eq:flux-boundary-condition}),
the flux function is
\begin{align*}
  F
  &\ideq - T^4 \\
  &\ideq -x^4,
    \yesnumber
    \label{eq:plane-flux-function}
\end{align*}
and it follows that the viability function is
\begin{align*}
  \Phi
  &\ideq (\del T)^2 - F^2 \\
  &\ideq 1 - x^8.
    \yesnumber
    \label{eq:plane-viability-function}
\end{align*}
The viable domain~$\Phi \ge 0$ (excluding the unphysical region~$x < 0$)
is therefore given by the infinite strip
\begin{equation}
  0 \le x \le 1.
  \label{eq:plane-viable-domain}
\end{equation}
The terminal curve~$\Phi = 0$ is also the $T$-contour~$x = 1$
(the boring traced boundary
mentioned in Section~\ref{sec:cartesian.plane.scaling}).
Using the terminology of Section~\ref{sec:introduction.tracing},
$x = 1$~is therefore a critical terminal curve,
and other traced boundaries will attach to it smoothly.

\begin{figure}
  \centredfigurecontent[width=0.5\textwidth]{%
    plane-traced-boundaries%
  }{
    Traced boundaries~(\ref{eq:plane-traced-boundary}).
  }
\end{figure}

The boundary tracing ODE~(\ref{eq:tracing-ode-coordinate-parametrisation-v})
becomes
\begin{important}{equation}
  \tder{y}{x} = \mp \frac{x^4}{\sqrt{1 - x^8}},
  \label{eq:plane-tracing-ode-coordinate-parametrisation-y}
\end{important}
which integrates to give traced boundaries of the form
\begin{important}{equation}
  y =
  \const
    \mp
  \frac{x^5}{5}
    \cdot
  \hypergeo \roundbr*{\frac{1}{2}, \frac{5}{8}; \frac{13}{8}; x^8},
  \label{eq:plane-traced-boundary}
\end{important}
shown in Figure~\ref{fig:plane-traced-boundaries},
where $\hypergeo$~is the hypergeometric function~%
\cite{olver-2010-nist-handbook-mathematical-functions}.
The translational symmetry in the $y$-direction
is a property inherited from the ODE~%
  (\ref{eq:plane-tracing-ode-coordinate-parametrisation-y}).
A local analysis near~$x = 1$ shows that
\begin{equation}
  y = \const \pm \sqrt{\frac{1 - x}{2}} + \order (1 - x)^{3/2},
  \label{eq:plane-traced-boundary-x-near-1}
\end{equation}
so that the traced boundaries do indeed attach smoothly
onto the critical terminal curve~$x = 1$, as expected.
Near~$x = 0$ each pair of traced boundaries forms a thin cusp of the form
\begin{equation}
  y = \const \mp \frac{x^5}{5} + \order \roundbr*{x^{13}}.
  \label{eq:plane-traced-boundary-x-near-0}
\end{equation}
Now each of the traced boundaries is a curve along which
the radiation boundary condition~%
  (\ref{eq:plane-scaled-radiation-boundary-condition})
is satisfied.
More complicated boundaries can be constructed
by patching together several of these curves, or portions thereof,
and the only requirement is that there be consistent orientation.
This requirement is satisfied by indentifying as the interior
the side on which $T$ (which equals~$x$) is greater,
i.e.~the side to the right of each curve.
Figure~\ref{fig:plane-traced-boundaries-patched} shows
a sample of the broad variety of radiation boundaries
which can be produced in this manner.

\begin{figure}
  \centredfigurecontent[width=\textwidth]{%
    plane-traced-boundaries-patched%
  }{
    Radiation boundaries~\figurestyle{black} patched together
    using the traced boundaries~%
      (\ref{eq:plane-traced-boundary})~\figurestyle{grey}.
    The critical terminal curve~$x = 1$ is shown for reference.
  }
\end{figure}

\begin{figure}
  \centredfigurecontent[width=\textwidth]{%
    plane-domains%
  }{
    Five domains marked out by a radiation boundary
    and a constant-temperature boundary.
    The critical terminal curve~$x = 1$ is shown for reference.
  }
\end{figure}

\subsection{Domain construction}
\label{sec:cartesian.plane.domain}

Since the constructed radiation boundaries only dissipate heat,
a domain for the steady conduction--radiation BVP
will not be completely specified
until there is also a boundary to supply it.
The simplest boundary condition which can supply heat
is the Dirichlet condition~$T = \const$,
and given the form of the known solution~%
  (\ref{eq:plane-scaled-laplace-solution}),
these boundaries are simply vertical lines~$x = \const$.

An infinite number of conduction--radiation domains
may therefore be marked out
by joining a constructed radiation boundary
with an appropriate Dirichlet boundary~$x = \const$,
as in Figure~\ref{fig:plane-domains}.
Each of these domains corresponds to steady conduction in the interior,
constant temperature along the right hand side,
and thermal radiation into vacuum to the left.
Most surprising is that \emph{all} of these domains
admit the \emph{same} exact solution~(\ref{eq:plane-scaled-laplace-solution}).

Unfortunately,
the domains produced here by boundary tracing are not convex,
but rather \term{self-viewing}:
some of the outgoing radiation travels not to infinity,
but strikes another part of the boundary,
where it might be partially or fully absorbed.
Since the simple radiation boundary condition~%
  (\ref{eq:plane-scaled-radiation-boundary-condition})
does not account for this,
the results of this section,
while mathematically sound,
are not physically valid.

\section{Cosinusoidal solution}
\label{sec:cartesian.cosine}

While boundary tracing as performed on
the plane-source solution~(\ref{eq:plane-laplace-solution})
did not yield convex conduction--radiation domains,
one might expect more favourable outcomes
by starting from solutions to Laplace's equation
which are not one-dimensional.
A simple class of such solutions consists of functions which are the product
of a trigonometric function in~$x$ and an hyperbolic function in~$y$.

As seen in Section~\ref{sec:cartesian.plane.domain}
one eventually requires a boundary to supply the heat being radiated away,
and from a practical point of view it is
straight-line, constant-temperature boundaries which are of interest.
Therefore, in this section I consider known solutions of the form
\begin{important}{equation}
  T \ideq T_0 \roundbr*{1 - B \cos\frac{x}{L_0} \cosh\frac{y}{L_0}},
  \label{eq:cosine-laplace-solution}
\end{important}
where $T_0$~is a temperature, $L_0$~is a length scale,
and $B$~is a dimensionless constant.
The temperature is constant (taking the value~$T = T_0$)
along the straight line~$x = \pi L_0/2$,
which shall ultimately serve as the heat-supplying boundary.

\subsection{Scaling}
\label{sec:cartesian.cosine.scaling}

First the physical parameters are scaled out to the extent possible.
Using the same scalings as Section~\ref{sec:cartesian.plane.scaling},
i.e.~(\ref{eq:plane-scaled-temperature}) through~(\ref{eq:plane-scaled-del}),
the radiation boundary condition~(\ref{eq:radiation-boundary-condition})
and the known solution~(\ref{eq:cosine-laplace-solution})
become
\begin{align}
  \normalvec \dotp \scaleddel \scaled{T}
    &= -\group{c \lambda \tau^3} \scaled{T}^4,
    \label{eq:cosine-scaled-radiation-boundary-condition-with-groups}
    \\[\tallspace]
  \scaled{T}
    &\ideq
      \group{\frac{T_0}{\tau}}
      \roundbr*{
        1 -
          B
          \cos \roundbr*{\group{\frac{\lambda}{L_0}} \scaled{x}}
          \cosh \roundbr*{\group{\frac{\lambda}{L_0}} \scaled{y}}
      },
    \label{eq:cosine-scaled-laplace-solution-with-groups}
\end{align}
where there are two free scales, $\tau$~(temperature) and $\lambda$~(length).
Since there are three unique dimensionless groups,
one of the groups cannot be eliminated,
and to keep the cosinusoidal terms as simple as possible,
the obvious scales
\begin{align}
  \tau &= T_0,
    \label{eq:cosine-temperature-scale} \\
  \lambda &= L_0,
    \label{eq:cosine-length-scale}
\end{align}
are chosen.
Defining the dimensionless group
\begin{equation}
  A \defeq \frac{1}{c L_0 {T_0}^3}.
  \label{eq:cosine-dimensionless-group}
\end{equation}
and \atten{dropping \scalingmarks},
the scaled boundary condition~%
  (\ref{eq:cosine-scaled-radiation-boundary-condition-with-groups})
and known solution~(\ref{eq:cosine-scaled-laplace-solution-with-groups})
become
\begin{important}{align}
  \normalvec \dotp \del T &= -\frac{T^4}{A},
    \label{eq:cosine-scaled-radiation-boundary-condition} \\[\tallspace]
  T &\ideq 1 - B \cos x \cosh y.
    \label{eq:cosine-scaled-laplace-solution}
\end{important}

\begin{figure}
  \centredfigurecontent[width=\textwidth]{%
    cosine-physical%
  }{
    Unphysical region~$T < 0$
    for the known solution~(\ref{eq:cosine-scaled-laplace-solution}),
    as $B$~increases.
    In each plot, the left edge is~$x = 0$
    and the thick vertical line is~$x = \pi/2$,
    which is also the contour~$T = 1$.
  }
\end{figure}

\subsection{Physical region}
\label{sec:cartesian.cosine.physical}

Since it is the straight-line boundary~$x = \pi/2$
which ultimately shall, in the form of a Dirichlet condition~$T = 1$,
supply the heat to be conducted and radiated away,
only the side on which~$T \le 1$ is relevant,
namely the side~$x \le \pi/2$.
Given that the known solution~(\ref{eq:cosine-scaled-laplace-solution})
is an even function of~$x$,
it also suffices to consider~$x \ge 0$ only.

Thus the region of interest
is the vertical strip~$0 \le x \le \pi/2$.
Not all of this strip is physical
as only positive temperatures are admissible
in thermal radiation problems.
Since the known solution~(\ref{eq:cosine-scaled-laplace-solution})
vanishes along~$\cos x \cosh y = 1 / B$,
the geometry of the physical region~$T \ge 0$
shall depend on the value of the dimensionless constant~$B$
(see Figure~\ref{fig:cosine-physical}).
The unphysical region~$T < 0$ is omitted from the remainder of the analysis,
as physically meaningful radiation boundaries cannot be found there.

\subsection{Boundary tracing equations}
\label{sec:cartesian.cosine.tracing}

In addition to only considering the physical region~$T \ge 0$,
traced boundaries will only exist within the viable domain~$\Phi \ge 0$.
For the boundary condition~%
  (\ref{eq:cosine-scaled-radiation-boundary-condition})
and known solution~(\ref{eq:cosine-scaled-laplace-solution}) at hand,
one obtains the derivatives
\begin{align}
  P &\ideq \pder{T}{x} \ideq \+B \sin x \cosh y,
    \label{eq:cosine-gradient-u-component} \\[\tallspace]
  Q &\ideq \pder{T}{y} \ideq -B \cos x \sinh y,
    \label{eq:cosine-gradient-v-component}
\end{align}
the flux function
\begin{equation}
  F \ideq -\frac{(1 - B \cos x \cosh y) ^ 4}{A},
  \label{eq:cosine-flux-function}
\end{equation}
and the viability function
\begin{align*}
  \Phi
  &\ideq (\del T)^2 - F^2 \\
  &\ideq
    B^2 \roundbr*{\sin^2 x + \sinh^2 y}
      -
    \frac{(1 - B \cos x \cosh y) ^ 8}{A^2}.
    \yesnumber
    \label{eq:cosine-viability-function}
\end{align*}
The geometry of the viable domain~$\Phi \ge 0$
therefore depends on both of the dimensionless parameters~$A$ and~$B$,
as does the boundary tracing ODE~%
  (\ref{eq:tracing-ode-coordinate-parametrisation-u}),
which becomes
\begin{important}{equation}
  \tder{x}{y} = \frac{P Q \mp F \sqrt{\Phi}}{Q^2 - F^2}.
  \label{eq:cosine-tracing-ode-coordinate-parametrisation-x}
\end{important}
Given the complexity of managing more than one parameter,
I consider~$B = 1$ before examining the general case.

\subsection{Simple case (\texorpdfstring{$B = 1$}{B = 1})}
\label{sec:cartesian.cosine.simple}

\subsubsection{Viable domain}
\label{sec:cartesian.cosine.simple.viable}

In the $B = 1$~case,
the known solution~(\ref{eq:cosine-scaled-laplace-solution}) simplifies to
\begin{equation}
  T \ideq 1 - \cos x \cosh y
  \label{eq:cosine-simple-laplace-solution}
\end{equation}
and the viability function~(\ref{eq:cosine-viability-function}) reduces to
\begin{equation}
  \Phi \ideq \sin^2 x + \sinh^2 y - \frac{(1 - \cos x \cosh y) ^ 8}{A^2}.
  \label{eq:cosine-simple-viability-function}
\end{equation}
The dependence of the non-viable domain~$\Phi < 0$
on the dimensionless group~$A$
is shown in Figure~\ref{fig:cosine_simple-physical-viable}.
Radiation boundaries as produced by boundary tracing
will only be found in the white region,
which is both physical~($T \ge 0$) and viable~($\Phi \ge 0$).

The non-viable domain consists of a single%
\footnote{
  There also exist non-viable regions
  (not shown in Figure~\ref{fig:cosine_simple-physical-viable})
  which lie within the unphysical region~$T < 0$,
  but recall that the unphysical region has been omitted from the analysis
  due to physical irrelevance.
}
non-viable region which recedes towards the right as $A$~increases.
Not obvious from Figure~\ref{fig:cosine_simple-physical-viable} however
is that the boundary of this single non-viable region,
henceforth called the \term{proper} terminal curve,
only constitutes \emph{almost all} of the terminal curve.
The full terminal curve consists of the proper terminal curve
together with the origin~$(x, y) = (0, 0)$,
which is a terminal point
by virtue of being an isolated root of the equation~$\Phi = 0$.

\begin{figure}
  \centredfigurecontent[width=\textwidth]{%
    cosine_simple-physical-viable%
  }{
    Non-viable domain~$\Phi < 0$
    for the known solution~(\ref{eq:cosine-simple-laplace-solution})
    and viability function~(\ref{eq:cosine-simple-viability-function}),
    as $A$~increases.
    In each plot, the left edge is~$x = 0$
    and the thick vertical line is~$x = \pi/2$,
    which is also the contour~$T = 1$.
  }
\end{figure}

\subsubsection{Boundary tracing}
\label{sec:cartesian.cosine.simple.tracing}

The tracing ODE~%
  (\ref{eq:cosine-tracing-ode-coordinate-parametrisation-x})
cannot be integrated analytically,
so the traced boundaries are determined numerically.
In doing so, the parametrisation~$x = x (y)$
is unable to handle traced boundaries which are locally horizontal;
such difficulty is avoided by instead using the arc-length parametrisation~%
  (\ref{eq:tracing-ode-arc-length-parametrisation-u})
and~(\ref{eq:tracing-ode-arc-length-parametrisation-v}),
which for Cartesian coordinates reduces to
\begin{align}
  \tder{x}{s} &= \frac{-Q F \pm P \sqrt{\Phi}}{(\del T)^2},
    \label{eq:cosine-tracing-ode-arc-length-parametrisation-x} \\[\tallspace]
  \tder{y}{s} &= \frac{\+P F \pm Q \sqrt{\Phi}}{(\del T)^2}.
    \label{eq:cosine-tracing-ode-arc-length-parametrisation-y}
\end{align}
Two branches of traced boundaries are obtained
by integrating forward
from starting points within the region
both physical~($T \ge 0$) and viable~($\Phi \ge 0$).
The resulting curves (Figure~\ref{fig:cosine_simple-traced-boundaries})
are boundaries along which the radiation boundary condition~%
  (\ref{eq:cosine-scaled-radiation-boundary-condition})
is satisfied.
For the correct boundary orientation,
note that the temperature~$T$ is an increasing function of~$x$;
it follows that the interior lies to the right of each boundary curve,
with heat being lost by radiation to the left.

\begin{figure}
  \centredfigurecontent[width=0.5\textwidth]{%
    cosine_simple-traced-boundaries%
  }{
    Traced boundaries for~$A = 0.5$,
    obtained by integrating
    (\ref{eq:cosine-tracing-ode-arc-length-parametrisation-x})
    and~(\ref{eq:cosine-tracing-ode-arc-length-parametrisation-y}).
  }
\end{figure}

By patching together the traced boundaries,
an unlimited number of radiation boundaries may be constructed.
However, almost all such constructions are non-convex
(see Figure~\ref{fig:cosine_simple-traced-boundaries-patched-spiky}).
At any point \emph{strictly} within the viable domain,
i.e.~$\Phi > 0$,
the two traced boundaries through it will cross at a non-zero angle,
and patching these two boundaries together
inevitably results in a spike made of non-convex curves.
As noted already in Section~\ref{sec:cartesian.plane.domain},
self-viewing radiation is not accounted for
by the simple radiation boundary condition~%
  (\ref{eq:cosine-scaled-radiation-boundary-condition}),
and so the non-convex constructions are all invalid.

\begin{figure}
  \newcommand*{\subfigurewidth}{0.36\textwidth}
  \centering
  \hspace*{\fill}
  \begin{subfigure}[t]{\subfigurewidth}
    \centredfigurecontent{cosine_simple-traced-boundaries-patched-spiky}{%
      Spiky non-convex boundary
    }
  \end{subfigure}
    \hfill
  \begin{subfigure}[t]{\subfigurewidth}
    \centredfigurecontent{cosine_simple-traced-boundaries-patched-smooth}{%
      Smooth candidate boundary
    }
  \end{subfigure}
  \hspace*{\fill}
  \caption{
    Radiation boundaries~\figurestyle{black} patched together
    using traced boundaries~\figurestyle{grey}.
  }
  \label{fig:cosine_simple-traced-boundaries-patched}
\end{figure}

\begin{figure}
  \centredfigurecontent[width=0.63\textwidth]{%
    cosine_simple-terminal-points%
  }{
    $T$-contours which intersect the proper terminal curve~$\Phi = 0$,
    which is the border of the non-viable domain~$\Phi < 0$.
    The horizontal scale is exaggerated.
  }
\end{figure}

To avoid the non-convex spikes,
the join must occur at a point along the terminal curve~$\Phi = 0$.
Specifically this point must not be an ordinary terminal point,
at which the two traced boundaries being patched
would form a cusp with inconsistent boundary orientation
(see Section~\ref{sec:introduction.tracing}).
Instead the join must occur at a critical terminal point;
recall that this is a point along the terminal curve
for which the local $T$-contour touches the terminal curve~$\Phi = 0$
tangentially (rather than crossing at a non-zero angle).

Now in the present situation,
the full terminal curve consists of
an isolated terminal point at the origin
in union with the proper terminal curve
(which is the boundary of the non-viable region
visible in Figure~\ref{fig:cosine_simple-physical-viable}).
The terminal point at the origin is already notable due to its isolation,
but even more remarkable is that the local $T$-contour~($T = 0$)
looks like~$y = \pm x$.
One therefore wonders:
  are the local $T$-contour (whose tangent is indeterminate)
  and the local portion of the terminal curve (which is just a point)
  \emph{tangential}, and therefore,
  is the origin a \emph{critical} terminal point?
While the tangency is debatable,
the answer to the criticality question ought to be yes
on account of the actual behaviour:
two traced boundaries pass through the origin
(see Figure~\ref{fig:cosine_simple-traced-boundaries})
as if it were an hyperbolic critical terminal point,
with $y = 0$~as the common tangent.
Unfortunately the pair of traced boundaries forms a non-convex spike,
which is of no use here.

With the origin ruled out,
it remains to consider the proper terminal curve.
Figure~\ref{fig:cosine_simple-terminal-points} shows that in the present case,
all points along the proper terminal curve are ordinary
except for a single critical terminal point~$(x, y) = (x_0, 0)$
at the intersection of the proper terminal curve
and the horizontal axis~$y = 0$.
Algebraically, $x_0$~is the positive%
\footnote{
  The trivial solution~$x = 0$
  corresponds to the isolated terminal point at the origin,
  which has already been dismissed.
}
solution to
\begin{equation}
  \eval[\big]{\Phi}_{y=0}
  \ideq \roundbr*{1 - \cos^2 x} - \frac{(1 - \cos x)^8}{A^2}
  = 0,
  \label{eq:cosine-simple-critical-terminal-point-x}
\end{equation}
a polynomial equation in~$\cos x$.
Furthermore, the local $T$-contour through~$(x_0, 0)$
lies on the viable side of the proper terminal curve;
therefore the critical terminal point~$(x_0, 0)$ is of hyperbolic type,
with two traced boundaries passing through it smoothly.
By patching together the portions which lie to the right of~$(x_0, 0)$,
a single spike-free radiation boundary can be constructed
as shown in Figure~\ref{fig:cosine_simple-traced-boundaries-patched-smooth};
this boundary shall be referred to as the \term{candidate boundary}.
It is interesting to note that
the candidate boundary and the proper terminal curve
are virtually indistinguishable;
despite only touching each other at the critical terminal point~$(x_0, 0)$,
the former asymptotically approaches the latter travelling away from~$y = 0$.

\subsubsection{Convexity}
\label{sec:cartesian.cosine.simple.convexity}

The choice of the candidate boundary over all other boundary patchings
is a necessary but not sufficient condition for convexity.
While the candidate boundary is convex at~$(x_0, 0)$,
it eventually inflects
at some~$(x, y) = (x_\infl, \pm y_\infl)$ depending on~$A$,
and becomes concave.

Recalling the goal of constructing domains
corresponding to a conduction--radiation BVP\@,
a \term{candidate domain} is demarcated by using
the candidate boundary as the radiation boundary
and the straight line~$x = \pi/2$
as a constant-temperature heat-supplying boundary.
The continuum of candidate domains (for~$0 < A < 1$)\footnote{
  At~$A = 1$ the candidate boundary touches the line~$x = \pi/2$,
  and the candidate domain shrinks to a point.
}
is shown in Figure~\ref{fig:cosine_simple-candidate-domains};
each domain is shaped like a thin lens,
corresponding to steady conduction in its interior,
thermal radiation along the curved boundary to the left,
and constant temperature~$T = 1$ along the straight boundary on the right.
The domain will be valid if the curved boundary
(a portion of the corresponding candidate boundary)
is convex, or equivalently,
if the first inflection of the corresponding candidate boundary
occurs at abscissa~$x_\infl \ge \pi/2$.

\begin{figure}
  \centredfigurecontent[width=\textwidth]{%
    cosine_simple-candidate-domains%
  }{
    Candidate domains marked out by a radiation boundary
    and the constant-temperature boundary~$x = \pi/2$,
    as $A$~increases.
    Points of inflection are shown for reference.
  }
\end{figure}

The critical value~$A = A_\infl$
(where the candidate domain changes from invalid to valid)
is that for which the first inflection occurs at precisely~$x_\infl = \pi/2$.
Given that the candidate boundary is never horizontal,
$A_\infl$~is best sought using the coordinate parametrisation~$x = x (y)$
for the candidate boundary.
In Cartesian coordinates, points of inflection coincide with
zero-crossings in the second derivative,
or, using primes for $y$-differentiation, $x''$.
Differentiating the first derivative~%
  (\ref{eq:cosine-tracing-ode-coordinate-parametrisation-x})
gives
\begin{equation}
  x'' = \tder{}{y} \roundbr*{\frac{P Q \mp F \sqrt{\Phi}}{Q^2 - F^2}},
  \label{eq:acceleration-traced-boundary-cartesian-by-y}
\end{equation}
whose right hand side may be reduced to a function purely of~$x$ and~$y$
by applying~(\ref{eq:cosine-tracing-ode-coordinate-parametrisation-x})
once more to eliminate the first derivatives.
Substituting~$x = x_\infl = \pi/2$, this becomes
\begin{equation}
  \eval[\big]{x''}_{x=\pi/2} =
    \frac{A^2 C}{\sqrt{A^2 C^2 - 1}}
    \squarebr*{
      2 S - (1 + S^2) \roundbr*{A^2 S + 4 \sqrt{A^2 (1 + S^2) - 1}}
    },
  \label{eq:%
    cosine-simple-traced-boundary-acceleration-cartesian-by-y-inflection%
  }
\end{equation}
where~$C \defeq \cosh y$ and~$S \defeq \sinh y$.
Only the square-bracketed factor can change sign,
and an analysis shows that for~$0 < A < 1$,
this factor has a unique zero-crossing~$S = S_\infl (A)$.
The would-be $y$-coordinate of inflection for a given~$A$ is therefore
\begin{equation}
  y = y_\infl (A) = \sinh^{-1} (S_\infl (A)).
  \label{eq:cosine-simple-traced-boundary-y-would-be-inflection}
\end{equation}
Since the inflection is supposed to occur at~$x_\infl = \pi/2$,
the critical value~$A = A_\infl$ is the solution to
\begin{equation}
  x (y_\infl (A)) = \pi/2,
  \label{eq:cosine-simple-traced-boundary-a-inflection-equation}
\end{equation}
whose left hand side is the coordinate parametrisation~$x = x (y)$
evaluated at the would-be $y$-coordinate~%
  (\ref{eq:cosine-simple-traced-boundary-y-would-be-inflection}).
The bisection algorithm yields
\begin{equation}
  A_\infl = 0.79718,
  \label{eq:cosine-simple-traced-boundary-a-inflection}
\end{equation}
and therefore the lens-like candidate domain for a given~$A$ is convex
if and only if
\begin{equation}
  A_\infl \le A < 1.
  \label{eq:cosine-simple-traced-boundary-convex-a-interval}
\end{equation}

\subsection{General case (\texorpdfstring{$B$~arbitrary}{B arbitrary})}
\label{sec:cartesian.cosine.general}

\subsubsection{Viable domain}
\label{sec:cartesian.cosine.general.viable}

In the general case the dimensionless group~$B$ may also be varied,
and altogether the parameter space~$(A, B)$ is two-dimensional.
Two degrees of freedom is too many to manage,
and the behaviour of the viable domain is best understood
by fixing~$A$ and varying~$B$.

\begin{figure}
  \centredfigurecontent[width=0.7\textwidth]{%
    cosine_general-critical%
  }{
    Terminal points along the $x$-axis, as $B$~increases with $A$~fixed.
  }
\end{figure}

Of particular interest are the terminal points along the $x$-axis;
these are necessarily critical terminal points
because the local $T$-contour and the terminal curve~$\Phi = 0$
both cross the $x$-axis vertically
(as $T$ and~$\Phi$ are both even functions of~$y$),
and are therefore tangential.
Algebraically these terminal points are given by the roots of
\begin{equation}
  \eval[\big]{\Phi}_{y=0}
  \ideq B^2 \sin^2 x - \frac{(1 - B \cos x)^8}{A^2}
  = 0,
  \label{eq:cosine-general-critical-terminal-point-x}
\end{equation}
which are shown in Figure~\ref{fig:cosine_general-critical}.
For sufficiently small~$B$,
there are no roots and the entire $x$-axis is non-viable,
until a single root~$x = x_\nat (A)$ appears at~$B = B_\nat (A)$.
As $B$~increases past this transitional value,
the root splits into two roots, $x = x_\flat (A, B)$ and~$x = x_\sharp (A, B)$.
At~$B = 1$, the lesser root~$x_\flat$ vanishes,
before becoming positive again but unphysical for~$B > 1$.
In terms of the viable domain, this translates into five cases
(Figure~\ref{fig:cosine_general-physical-viable}):
\begin{enumerate}
  \item
    \label{itm:cartesian.cosine.general.regimes.gentle}
    \term{Gentle regime}, $B < B_\nat (A)$:
    The non-viable domain completely envelopes the $x$-axis,
    dividing the viable domain into two halves.
    Two hyperbolic critical terminal points exist along the $y$-axis
    at~$(x, y) = (0, \pm y_0)$.
  \item
    \label{itm:cartesian.cosine.general.regimes.gentle-to-fair}
    \term{Gentle-to-fair transition}, $B = B_\nat (A)$:
    The two viable regions pinch together at~$(x, y) = (x_\nat, 0)$,
    a critical terminal point of hyperbolic type.%
    \footnote{
      Here the local $T$-contour crosses the $x$-axis vertically
      while the terminal curve~$\Phi = 0$ has indeterminate tangent
      (being the location of a self-intersection).
      While the tangency is again debatable,
      two traced boundaries pass through vertically
      as if the vertical line~$x = x_\nat$ were the common tangent.
    }
    The non-viable domain now consists of two disjoint regions.
  \item
    \label{itm:cartesian.cosine.general.regimes.fair}
    \term{Fair regime}, $B_\nat (A) < B < 1$:
    The two non-viable regions are sundered further apart
    by a growing viable tract along the $x$-axis,
    with $(x_\nat, 0)$~splitting into the two
    hyperbolic critical terminal points~$(x_\flat, 0)$ and~$(x_\sharp, 0)$.
  \item
    \label{itm:cartesian.cosine.general.regimes.fair-to-steep}
    \term{Fair-to-steep transition}, $B = 1$:
    This is the simple case
    of Section~\ref{sec:cartesian.cosine.general.viable};
    the hyperbolic critical terminal point~$(x_\sharp, 0)$
    is what was formerly called~$(x_0, 0)$.
    The smaller non-viable region of the fair regime
    has shrunken to nothingness,
    and the critical terminal points~$(0, \pm y_0)$ and~$(x_\flat, 0)$
    have merged together
    to form the isolated critical terminal point at the origin.
  \item
    \label{itm:cartesian.cosine.general.regimes.steep}
    \term{Steep regime}, $B > 1$:
    The smaller non-viable region
    reappears but is overrun by the growing unphysical region.
    The terminal points~$(0, \pm y_0)$ (if they exist)
    and~$(x_\flat, 0)$ all lie in the unphysical region
    and are therefore irrelevant,
    leaving a single hyperbolic critical terminal point~$(x_\sharp, 0)$.
\end{enumerate}

\subsubsection{Boundary tracing}
\label{sec:cartesian.cosine.general.tracing}

An exhaustive treatment cannot be given here
due to the complex dependence
of the known solution~$T$ and the viability function~$\Phi$
on both~$A$ and~$B$.
Nevertheless, broad statements can be made.

\begin{figure}
  \centredfigurecontent[width=\textwidth]{%
    cosine_general-physical-viable%
  }{
    Non-viable domain~$\Phi < 0$ and critical terminal points
    for the known solution~(\ref{eq:cosine-scaled-laplace-solution})
    and viability function~(\ref{eq:cosine-viability-function}),
    as $B$~increases with $A$~fixed.
    In each plot, the left edge is~$x = 0$ and the right edge is~$x = 2$.
  }
\end{figure}

In Case~\ref{itm:cartesian.cosine.general.regimes.gentle} (gentle regime)
only non-convex spikes can be formed
by patching together traced boundaries,
even at the critical terminal points~$(0, \pm y_0)$ along the $y$-axis.
Case~\ref{itm:cartesian.cosine.general.regimes.fair-to-steep}
  (fair-to-steep transition)
is the simple $B = 1$~case
which has been analysed in Section~\ref{sec:cartesian.cosine.general.viable};
recall that a conduction--radiation domain in the shape of a convex lens
can be produced for each~$A$ in the interval~%
  (\ref{eq:cosine-simple-traced-boundary-convex-a-interval})
by patching together the traced boundaries through~$(x_\sharp, 0)$.
The situation in
Case~\ref{itm:cartesian.cosine.general.regimes.steep} (steep regime)
is very similar,
but the convex lens shapes which can be produced
are even thinner than those of
Case~\ref{itm:cartesian.cosine.general.regimes.fair-to-steep}.

This leaves
Case~\ref{itm:cartesian.cosine.general.regimes.gentle-to-fair}
  (gentle-to-fair transition)
and
Case~\ref{itm:cartesian.cosine.general.regimes.fair} (fair regime),
the former of which need not be treated separately
since it is effectively a degenerate version of the latter
(with~$x_\flat = x_\sharp$).
In the same manner as a spike-free candidate boundary
was constructed through~$(x_0, 0)$
in Section~\ref{sec:cartesian.cosine.general.viable},
so may candidate boundaries be constructed
through the hyperbolic critical terminal points~%
  $(x_\flat, 0)$ and~$(x_\sharp, 0)$.
Again one can form lens-shaped candidate domains,
and these will be convex for sufficiently well-chosen~$A$ and~$B$.

Much more interesting however is the existence
in Case~\ref{itm:cartesian.cosine.general.regimes.fair} (fair regime)
of convex conduction--radiation domains
which are \emph{not} shaped as lenses.
I cannot give a general analysis here
due to the complex dependence of the curvature on~$A$ and~$B$,
so I merely provide the illustrative example~$(A, B) = (12, 0.082506)$.
Figure~\ref{fig:cosine_general-traced-boundaries-convex}
shows (among other features) the frontiers of inflection
for the two branches of traced boundaries,
obtained by computing the zero-contours of the second derivative~%
  (\ref{eq:acceleration-traced-boundary-cartesian-by-y})
using numerical integration.
Only the portions of traced boundary which lie
convex side of the inflection frontiers
will be valid as radiation boundaries.
Since the end is to construct a domain
with the straight line contour~$x = \pi/2$
serving as the constant-temperature boundary,
one may immediately discard any convex portion of traced boundary
which does not reach~$x = \pi/2$.
Of the remaining convex portions,
those which do not reach the $x$-axis~($y = 0$)
are unable to join up with a convex portion of the opposite branch
(which is necessary to form a complete domain boundary);
discarding these also leaves the convex portions of traced boundary
shown in Figure~\ref{fig:cosine_general-traced-boundaries-convex}.

\begin{figure}
  \newcommand*{\subfigurewidth}{0.45\textwidth}
  \centering
  \includegraphics[width=\textwidth]%
    {cosine_general-asymmetric-construction-legend}
  \hspace*{\fill}
  \begin{subfigure}[t]{\subfigurewidth}
    \centredfigurecontent{cosine_general-traced-boundaries-convex}{%
      Convex portions of the two branches of traced boundaries
      which reach both the constant-temperature boundary~$x = \pi/2$
      and the $x$-axis.
    }
  \end{subfigure}
  \hfill
  \begin{subfigure}[t]{\subfigurewidth}
    \centredfigurecontent{cosine_general-asymmetric_domain}{%
      An asymmetric domain marked out by a radiation boundary
      and the constant-temperature boundary~$x = \pi/2$.
    }
  \end{subfigure}
  \hspace*{\fill}
  \caption{
    Constructing an asymmetric conduction--radiation domain
    for~$(A, B) = (12, 0.082506)$ (fair regime).
  }
  \label{fig:cosine_general-asymmetric-construction}
\end{figure}

The smooth traced boundary through the critical terminal point~$(x_\sharp, 0)$
may be used to construct the familiar lens-shaped domain.
The remaining convex portions of traced boundary
shown in Figure~\ref{fig:cosine_general-traced-boundaries-convex}
can be used to construct conduction--radiation domains
which are not lens-shaped, including asymmetric domains
(Figure~\ref{fig:cosine_general-asymmetric_domain});
simply take two intersecting traced boundaries
(one from the lower branch and one from the upper)
and trim them where they intersect with each other
and with the constant-temperature boundary~$x = \pi/2$.
It is most remarkable that
\emph{any} conduction--radiation domain constructed in this manner
will admit the very same exact solution~%
  (\ref{eq:cosine-scaled-laplace-solution}).

\subsection{Numerical verification}
\label{sec:cartesian.cosine.verification}

The validity of any convex domain constructed
in Sections~\ref{sec:cartesian.cosine.simple}
and~\ref{sec:cartesian.cosine.general}
may be tested by numerically solving the conduction--radiation BVP
in that domain and comparing the result
to the exact solution~(\ref{eq:cosine-scaled-laplace-solution}).
To do this I use \lang{Mathematica~12},
whose finite elements package~\code{NDSolve\`{}FEM\`{}}
has built-in capabilities for mesh generation and for solving nonlinear BVPs.

\begin{figure}
  \centredfigurecontent[width=\textwidth]{%
    cosine-verification-domain-meshes%
  }{
    Selected domains and finite element meshes for numerical verification.
  }
\end{figure}

Figure~\ref{fig:cosine-verification-domain-meshes}
shows the two selected test domains:
for the simple case~($B = 1$),
this is the lens-shaped domain with~$A = 0.8$,
and for the general case~($B$~arbitrary),
the asymmetric domain constructed
in Figure~\ref{fig:cosine_general-asymmetric_domain}.
Each domain is discretised into an unstructured triangular mesh
with around 500~elements.

Laplace's equation~(\ref{eq:laplace-steady-conduction}) is then solved,
with the radiation condition~%
  (\ref{eq:cosine-scaled-radiation-boundary-condition})
applied along the radiation boundary
and the Dirichlet condition~$T = 1$ applied
along the constant-temperature boundary~$x = \pi/2$.
The agreement between the resulting numerical solutions
and the exact solution~(\ref{eq:cosine-scaled-laplace-solution})
is excellent;
in both cases the relative error is strictly less than~$\SI{0.06}{\percent}$
throughout all vertices of the mesh.

\section{Summary}
\label{sec:cartesian.summary}

In this chapter I have applied the method of boundary tracing
for the thermal radiation problem in Cartesian coordinates.
Starting from the simplest possible solution,
namely the one-dimensional solution~(\ref{eq:plane-laplace-solution}),
a family of hypergeometric traced boundaries
with translational symmetry in~$y$ is produced.
By construction
the radiation condition~(\ref{eq:radiation-boundary-condition})
holds along these boundaries,
and they may be patched together
to form a broad variety of radiation boundaries.
Unfortunately these boundaries are physically invalid
because they are not convex;
the boundary condition~(\ref{eq:radiation-boundary-condition})
does not account for self-viewing radiation.

A more favourable outcome might be expected
by starting from a solution to Laplace's equation
which is not one-dimensional.
The conduction--radiation problem requires a heat source,
and since it is straight-line, constant-temperature boundaries
which are of the most physical interest,
I have chosen the cosinusoidal solution~(\ref{eq:cosine-laplace-solution}),
which has the straight-line contour~$x = \pi L_0/2$.
In the simple case~($B = 1$)
I have shown that a lens-shaped domain can be produced
which is convex for certain values of the dimensionless group~$A$,
while for the more complicated general case~($B$~arbitrary)
I have demonstrated the construction of an asymmetric domain
from convex portions of traced boundary.
Finally I have confirmed the validity of the domains produced
by comparing numerical solutions to the conduction--radiation BVP
with the exact cosinusoidal solution~(\ref{eq:cosine-laplace-solution}).
