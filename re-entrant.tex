\chapter{Re-entrant wedges}
\label{ch:re-entrant}

In this chapter
boundary tracing is applied to re-entrant capillary wedges,
i.e.~wedges with half-angle~$\alpha > \pi/2$.
For simplicity we only consider infinite wedge domains;
this rules out the asymmetric situations observed by
Korevaar~\cite{korevaar-1980-capillary-re-entrant-corner}
and by King~\etal~\cite{king-1999-laplace-young-near-corner}.
We take a numerical solution to
the scaled capillary BVP~(\ref{eq:scaled-laplace-young})
\&~(\ref{eq:scaled-contact-boundary-condition})
in a wedge
and seek rounding curves along which
the contact condition~(\ref{eq:scaled-contact-boundary-condition})
remains satisfied.
Finally, practical results are obtained for the dip-coating problem
of Figure~X, % TODO
by embracing the sharp corners
that have heretofore been avoided
for the sake of constructing smooth corner roundings.
% TODO
% Perhaps for the body:
% 'Our avoidance of sharp corners has been most restrictive',
% 'Here we change tack and use them to our advantage', etc.

\section{Numerical wedge solutions}
\label{sec:re-entrant.numerical}

In this section we obtain numerical wedge solutions
to the capillary BVP~(\ref{eq:scaled-laplace-young})
\&~(\ref{eq:scaled-contact-boundary-condition}).
Finite element meshes are constructed in an identical manner
to Section~\ref{sec:moderate.nonlinear.numerical.wedge};
we take the domain to be the sector $0 < r < 10$, $-\alpha < \phi < \alpha$,
we require all mesh elements to have no more area
than an equilateral triangle of side length~$0.2$,
and we apply the minimalistic refinement strategy
of Section~\ref{sec:moderate.nonlinear.numerical.half-plane}
along the two wedge walls,
using a fine length scale of~$0.01$.
Figure~X % TODO
shows the resultant mesh of X~elements
  % TODO (and perhaps a footnote about elements increasing with alpha)
for an $\alpha = \SI{135}{\degree}$~wedge.

As before, we obtain numerical solutions in \software{Mathematica}
by specifying the capillary BVP
as the steady-state diffusion problem~(\ref{eq:laplace-young-diffusion})
\&~(\ref{eq:contact-boundary-condition-diffusion}),
with the non-wetting contact condition~%
  (\ref{eq:natural-boundary-condition-diffusion})
applied on the outer arc~$r = 10$
to approximate a vanishing height rise at infinity.

Having precluded asymmetry,
the solution in the moderate re-entrant case~$\pi/2 < \alpha < \pi/2 + \gamma$
has the locally planar form~%
  (\ref{eq:moderate-wedge-asymptotic-solution}),
and we can compare the theoretical and computed values
for the slope in the corner,
as in Table~X. % TODO
% TODO
% For \alpha = \pi/2 + \gamma, check against King et al. (46).
% For \alpha > \pi/2 + \gamma, (43) (but note that A is undetermined).
% Then a sentence about excellent agreement.

\section{Corner rounding}
\label{sec:re-entrant.rounding}

In this section,
we take the numerical wedge solutions that have just been obtained
and perform boundary tracing
to seek roundings of the wedge corner.

\subsection{Boundary tracing}
\label{sec:re-entrant.rounding.tracing}

First we consider the default case,
where the contact angle for boundary tracing
is the same as the contact angle~$\gamma$
for the known numerical solution~$T$.

Proceeding as in Section~\ref{sec:moderate.nonlinear.tracing},
the contact condition~(\ref{eq:scaled-contact-boundary-condition})
is rewritten as
\begin{important}{equation}
  \normalvec \dotp \del T = \cos\gamma \sqrt{1 + (\del T)^2},
  \label{eq:scaled-contact-flux-boundary-condition-repeat_re-entrant}
\end{important}
and we again have flux function
\begin{equation}
  F \ideq \cos\gamma \sqrt{1 + (\del T)^2},
  \label{eq:contact-flux-function-repeat_re-entrant}
\end{equation}
viability function
\begin{equation}
  \Phi \ideq \sin^2\gamma \, (\del T)^2 - \cos^2\gamma,
  \label{eq:contact-viability-function-repeat_re-entrant}
\end{equation}
and viable domain
\begin{equation}
  \norm{\del T} \ge \cot\gamma.
  \label{eq:contact-viable-domain-repeat_re-entrant}
\end{equation}
The non-viable domain is shown
in Figure~X. % TODO
Away from the corner but near the walls,
the wedge solution may be approximated by the half-plane solution
with contact angle~$\gamma$
(and hence slope~$\cot\gamma$)
along the wall;
therefore the terminal curve
\begin{equation}
  \norm{\del T} = \cot\gamma
  \label{eq:contact-terminal-curve-repeat_re-entrant}
\end{equation}
asymptotes to the wedges walls as one travels away from the wedge corner.
The viable domain lies to the left of the terminal curve
and hugs the wedge walls.

Working in Cartesian coordinates as before,
we have the derivatives
\begin{align}
  P &\ideq \pder{T}{x},
    \label{eq:re-entrant-gradient-x-component} \\[\tallspace]
  Q &\ideq \pder{T}{y}.
    \label{eq:re-entrant-gradient-y-component}
\end{align}
Again the boundary tracing system of ODEs for arc-length parametrisation
reduces to
\begin{important}{align}
  \tder{x}{s} &= \frac{-Q F \pm P \sqrt{\Phi}}{(\del T)^2},
    \label{eq:re-entrant-tracing-ode-arc-length-parametrisation-x}
    \\[\tallspace]
  \tder{y}{s} &= \frac{+P F \pm Q \sqrt{\Phi}}{(\del T)^2},
    \label{eq:re-entrant-tracing-ode-arc-length-parametrisation-y}
\end{important}
and we obtain traced boundaries
as in Figure~X % TODO
by performing numerical integration
from starting points within the viable domain.

As in the convex-wedge case,
the original wedge walls~$\phi = \pm\alpha$
are themselves traced boundaries,
but here a swapping of the two branches has occurred;
the top wall~$\phi = +\alpha$ now belongs to the lower branch
of~(\ref{eq:re-entrant-tracing-ode-arc-length-parametrisation-x})
\&~(\ref{eq:re-entrant-tracing-ode-arc-length-parametrisation-y}),
rather than the upper branch.
The swapping of branches at~$\alpha = \pi/2$ may be seen
by substituting the locally planar form~%
  (\ref{eq:moderate-asymptotic-solution});
without assuming either a convex or a re-entrant wedge,
the boundary tracing system of ODEs reduces to
\begin{align}
  \tder{x}{s} &\asy \mp \abs{\cos\alpha},
    \label{eq:planar-asymptotic-tracing-ode-arc-length-parametrisation-x}
    \\[\tallspace]
  \tder{y}{s} &= -\sin\alpha,
    \label{eq:planar-asymptotic-tracing-ode-arc-length-parametrisation-y}
\end{align}
so that
\begin{equation}
  \tder{y}{x} \asy \frac{\pm\sin\alpha}{\abs{\cos\alpha}} =
    \begin{cases}
      \pm\tan\alpha, &\text{$\alpha < \pi/2$ (convex)} \\
      \mp\tan\alpha, &\text{$\alpha > \pi/2$ (re-entrant)}
    \end{cases}
  \label{eq:planar-asymptotic-tracing-ode-coordinate-parametrisation-y}
\end{equation}
for traced boundaries near the corner.

We return now to
Figure~X, % TODO
and we attempt to construct a rounding of the corner
by patching together the traced boundaries.
As in Section~\ref{sec:moderate.nonlinear.tracing},
corners are inevitably formed \emph{strictly} within the viable domain;
therefore any smooth patching together must necessarily occur
along the terminal curve~(\ref{eq:contact-terminal-curve-repeat_re-entrant}).
Inspecting Figure~X, % TODO
we see that almost all points along the terminal curve are ordinary,
for which the two local traced boundaries form a cusp
with inconsistent boundary orientation.
As before, the only exception is
a single critical terminal point
along the line of symmetry~$y = 0$,
explicitly~$(x, y) = (x_0, 0)$,
where $x = x_0$~is the unique solution to
\begin{equation}
  \eval*{-\pder{T}{x}}_{y=0} = \cot\gamma.
  \label{eq:re-entrant-critical-terminal-point}
\end{equation}
Unlike the convex wedge case,
the local $T$-contour lies toward
the non-viable side of the the terminal curve;
the critical terminal point~$(x_0, 0)$ is therefore of elliptic type,
and there are no smooth traced boundaries passing through it.
Thus we cannot construct a rounding of the corner.

In fact there are no new domains at all,
even with sharp corners,
that can be constructed from the traced boundaries
of Figure~X.
Let us briefly view the traced boundaries
as trajectories of a dynamical system,
with direction of travel as shown in
Figure~X.
In the case of a convex wedge,
the original wedge walls are stable manifolds of their respective branches,
as confirmed by the perturbation analysis
of Section~\ref{sec:moderate.nonlinear.tracing}
(which culminated in the estimate~%
  (\ref{eq:moderate-perturbation-near-wall-approach})
for the rate at which traced boundaries approach the wall).
For a re-entrant wedge we find the opposite;
indeed a similar perturbation analysis
(again using wall coordinates~$\xi$ and~$\eta$)
yields
\begin{equation}
  \tder{\vd\xi}{\eta} \asy
    \squarebr*{
      +\frac{h}{\sin\gamma} \cdot \roundbr*{\pder{T}{\eta}}^{-1}
    }_{\xi=0}
      \cdot \vd\xi,
  \label{eq:re-entrant-perturbation-near-wall-repulsion}
\end{equation}
with the numerical evidence indicating that
$\pd T / {\pd\eta}$~is positive along the wall
(i.e.~the height rise along the wall increases
as one moves away from the corner).
The wedge walls are therefore unstable;
the traced boundaries are repelled from the walls,
eventually colliding with the terminal curve and terminating.
Thus, the only traced boundaries which extend to infinity
are the original wedge walls themselves,
and no new domains can be constructed.

\subsection{Different contact angle}
\label{sec:re-entrant.rounding.different}

The bleak outlook persists
even when a different tracing contact angle~$\tr{\gamma}$ is used
to the contact angle~$\gamma$ of the known solution.
As in Section~\ref{sec:moderate.multiple.different},
we replace the contact condition~%
  (\ref{eq:scaled-contact-flux-boundary-condition-repeat_re-entrant})
with the more general~%
  (\ref{eq:scaled-contact-flux-boundary-condition-different-angle}),
i.e.
\begin{important}{equation}
  \normalvec \dotp \del T (x, y; \alpha, \gamma) =
    \cos\tr{\gamma}
    \sqrt{1 + \squarebr[\bulkysize]{\del T (x, y; \alpha, \gamma)}^2},
  \label{eq:scaled-contact-flux-boundary-condition-different-angle%
    -repeat_re-entrant
  }
\end{important}
leading to flux function~(\ref{eq:contact-flux-function-different-angle}),
viability function~(\ref{eq:contact-viability-function-different-angle}),
and viable domain~(\ref{eq:contact-viable-domain-different-angle}).

For~$\tr{\gamma} < \gamma$
(Figure~X), % TODO
we again have a viable domain
that does not extend to infinity,
and a rounding of the corner cannot be produced
from the traced boundaries.

For~$\tr{\gamma} > \gamma$
(Figure~X), % TODO
as in Section~\ref{sec:moderate.multiple.different},
the avoidance of corners leads us to
a lone critical terminal point~$(x_0, 0)$
at the intersection between the terminal curve
and the line of symmetry~$y = 0$,
with $x = x_0$~being the unique solution to
\begin{equation}
  \eval*{-\pder{T}{x}}_{y=0} = \cot\tr{\gamma}.
  \label{eq:re-entrant-critical-terminal-point-different-angle}
\end{equation}
Unfortunately $(x_0, 0)$~is of elliptic type,
as the local $T$-contour lies toward the non-viable side
of the terminal curve (Figure~X),
so we cannot construct a rounding of the corner.

\section{Dip-coating}
\label{sec:re-entrant.dip-coating}

\subsection{Roughness}
\label{sec:re-entrant.dip-coating.roughness}

\subsection{Indentations}
\label{sec:re-entrant.dip-coating.indentations}

\section{Summary}
\label{sec:re-entrant.summary}

% TODO
\{TO BE WRITTEN\}
