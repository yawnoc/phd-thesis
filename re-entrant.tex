\chapter{Re-entrant wedges}
\label{ch:re-entrant}

In this chapter
boundary tracing is applied to re-entrant capillary wedges,
i.e.~wedges with half-angle~$\alpha > \pi/2$.
We take a numerical solution to
the scaled capillary BVP~(\ref{eq:scaled-laplace-young})
\&~(\ref{eq:scaled-contact-boundary-condition})
in a wedge
and seek rounding curves along which
the contact condition~(\ref{eq:scaled-contact-boundary-condition})
remains satisfied.
Finally, practical results are obtained for the dip-coating problem
of Figure~X, % TODO
by embracing the sharp corners
that have heretofore been avoided
for the sake of constructing smooth corner roundings.
% TODO
% Perhaps for the body:
% 'Our avoidance of sharp corners has been most restrictive',
% 'Here we change tack and use them to our advantage', etc.

\section{Numerical wedge solutions}
\label{sec:re-entrant.numerical}

\section{Corner rounding}
\label{sec:re-entrant.rounding}

\subsection{Boundary tracing}
\label{sec:re-entrant.rounding.tracing}

\subsection{Different contact angle}
\label{sec:re-entrant.rounding.angle}

\section{Dip-coating}
\label{sec:re-entrant.dip-coating}

\subsection{Roughness}
\label{sec:re-entrant.dip-coating.roughness}

\subsection{Indentations}
\label{sec:re-entrant.dip-coating.indentations}

\section{Summary}
\label{sec:re-entrant.summary}

% TODO
\{TO BE WRITTEN\}
