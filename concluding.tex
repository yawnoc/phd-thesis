\chapter{Concluding remarks}
\label{ch:concluding}

In this thesis,
we have applied the theory of boundary tracing
in the physically significant contexts
of thermal radiation and capillarity.
After deriving general versions of the boundary tracing ODE
(for both coordinate parametrisation
and the numerically robust arc-length parametrisation)
applicable in any orthogonal coordinate system,
we have thoroughly analysed
the traced boundaries that can be produced in various scenarios:

\section{Part~\ref*{pt:radiation}: Thermal radiation}
\label{sec:concluding.radiation}

In the first part we have examined the use of boundary tracing
in a simple conduction--radiation problem,
consisting of Laplace's equation in the interior
and thermal radiation along the boundary
in accordance with the Stefan--Boltzmann law.
While exact solutions to Laplace's equation are plenty,
the quartic dependence of the boundary flux on the temperature
is difficult to handle analytically;
thus boundary tracing is a well-suited approach to tackling this problem.

Starting from the one-dimensional solution to Laplace's equation,
boundary tracing produces a non-trivial family of exact curves
which yet satisfy the radiation boundary condition.
By patching these together
we have formed a vast array of new domains,
all of which satisfy the original conduction--radiation problem,
but whose non-convexity is problematic
because the boundary condition does not account for self-viewing radiation.
Even so, we have demonstrated a means of identifying cases
where the amount of self-viewing radiation is negligible.
We have also constructed an explicit example of a radiating fin
whose physical length and temperature are of a practical size.

Better results on the convexity front have been achieved
by beginning instead from a cosinusoidal solution to Laplace's equation.
We have obtained a continuum of lens-shaped domains,
as well as an asymmetric domain
which nevertheless admits a symmetric solution.

% TODO
% + Continuum of domains for line and bipolar
% + Future work

\section{Part~\ref*{pt:capillarity}: Capillarity}
\label{sec:concluding.capillarity}

\section{Final remarks}
\label{sec:concluding.final}
