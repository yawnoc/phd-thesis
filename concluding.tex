\chapter{Concluding remarks}
\label{ch:concluding}

In this thesis,
we have applied the theory of boundary tracing
in the physically significant contexts
of thermal radiation and capillarity.
After deriving general versions of the boundary tracing ODE
(for both coordinate parametrisation
and the numerically robust arc-length parametrisation)
applicable in any orthogonal coordinate system,
we have thoroughly analysed
the traced boundaries that can be produced in various scenarios:

\section{Part~\ref*{pt:radiation}: Thermal radiation}
\label{sec:concluding.radiation}

In the first part we have examined the use of boundary tracing
in a simple conduction--radiation problem,
consisting of Laplace's equation in the interior
and thermal radiation along the boundary
in accordance with the Stefan--Boltzmann law.
While exact solutions to Laplace's equation are plenty,
the quartic dependence of the boundary flux on the temperature
is difficult to handle analytically using conventional approaches;
thus boundary tracing is a well-suited approach to tackling this problem.

For each of several analytical solutions to Laplace's equation,
we have constructed a vast array of new domains
which admit exactly that solution
to the conduction--radiation problem.
These results have been solidly backed
by numerical verification using finite elements.
Although certain constructed domains are inadmissible
due to non-convexity,
we have demonstrated how to salvage useful results from among them
by identifying cases
where the amount of self-viewing radiation is negligibly small.
In determining the physically realisable lengths and temperatures
in various instances,
we have shown that the results of boundary tracing
are indeed applicable in practical contexts,
and not just mathematical curiosities of theoretical interest.

% TODO
% + Future work

\section{Part~\ref*{pt:capillarity}: Capillarity}
\label{sec:concluding.capillarity}

\section{Final remarks}
\label{sec:concluding.final}
