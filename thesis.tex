\documentclass[a4paper,11pt]{book}
\title{Applications of boundary tracing in thermal radiation and capillarity}
\author{Conway}

%%%%%%%%%%%%%%%%%%%%%%%%%%%%%%%%%%%%%%%%%%%%%%%%%%%%%%%%%%%%%%%%
% Customisation
%%%%%%%%%%%%%%%%%%%%%%%%%%%%%%%%%%%%%%%%%%%%%%%%%%%%%%%%%%%%%%%%

% Margins
%   NOTE: a 4cm inner margin is required for binding
\usepackage[
  bindingoffset=2cm,
  inner=2cm,
  outer=2.5cm,
  top=3cm,
  bottom=2.7cm,
]{geometry}

% Slightly-more-than-one spacing
% --------
% NOTE: for 11pt font, \onehalfspacing is defined as \setstretch{1.213}.
% See <https://github.com/rf-latex/setspace/blob/02cf9e0/setspace.sty#L350>.
% The general consensus is that
% (1) nobody knows what one-and-a-half and double spacing mean, and
% (2) large line spacing is ugly.
% See <https://tex.stackexchange.com/a/13803>.
\usepackage{setspace}
\setstretch{1.13}

% Heading styles
% --------
% Larger part headings label
% Smaller chapter headings
\usepackage{titlesec}
\titleformat{\part}[display]%
  {\setstretch{1.5}\normalfont\Huge\bfseries\filcenter}%
  {\partname~\thepart}%
  {0em}%
  {}
\titleformat{\chapter}[display]%
  {\setstretch{1}\normalfont\huge\bfseries\filright}%
  {\chaptertitlename~\thechapter}%
  {0em}%
  {}
\titlespacing*{\chapter}{0em}{0em}{3.3em}

% Citations ordered by number with whitespace stripped for keys
\usepackage{cite}

% Symbols for footnotes
\renewcommand*{\thefootnote}{\fnsymbol{footnote}}

%%%%%%%%%%%%%%%%%%%%%%%%%%%%%%%%%%%%%%%%%%%%%%%%%%%%%%%%%%%%%%%%
% Mathematical notation
%%%%%%%%%%%%%%%%%%%%%%%%%%%%%%%%%%%%%%%%%%%%%%%%%%%%%%%%%%%%%%%%

% General symbols ("\;" etc.)
\usepackage{amsmath}

% Paired delimiters ("\abs" etc.)
\usepackage{mathtools}

% Boxed equation emphasis
\usepackage{empheq}

% Physical units
\usepackage{siunitx}

%%%%%%%%%%%%%%%%%%%%%%%%%%%%%%%%%%%%%%%%%%%%%%%%%%%%%%%%%%%%%%%%
% Figures
%%%%%%%%%%%%%%%%%%%%%%%%%%%%%%%%%%%%%%%%%%%%%%%%%%%%%%%%%%%%%%%%

% External graphics
\usepackage{graphicx}

% Folder of figures
\graphicspath{{figures/}}

% Subfigures with captions
\usepackage{subcaption}

% Centred figure content (with same label as file name)
%   #1  [\includegraphics options] default "width=\textwidth"
%   #2  {\includegraphics file name}
%   #3  {caption}
\newcommand{\centredfigurecontent}[3][width=\textwidth]{%
  \centering
  \includegraphics[#1]{#2}
  \caption{#3}
  \label{fig:#2}
}

%%%%%%%%%%%%%%%%%%%%%%%%%%%%%%%%%%%%%%%%%%%%%%%%%%%%%%%%%%%%%%%%
% Hyperlinks
%%%%%%%%%%%%%%%%%%%%%%%%%%%%%%%%%%%%%%%%%%%%%%%%%%%%%%%%%%%%%%%%

\usepackage[hidelinks,bookmarksopen=true]{hyperref}

%%%%%%%%%%%%%%%%%%%%%%%%%%%%%%%%%%%%%%%%%%%%%%%%%%%%%%%%%%%%%%%%
% Math macros
%%%%%%%%%%%%%%%%%%%%%%%%%%%%%%%%%%%%%%%%%%%%%%%%%%%%%%%%%%%%%%%%

% Optional superscript
\newcommand*{\optionalsup}[1]{%
  \ifx \relax #1\relax
  \else
    ^{#1}%
  \fi
}

% Optional subscript
\newcommand*{\optionalsub}[1]{%
  \ifx \relax #1\relax
  \else
    _{#1}%
  \fi
}

% Opposite of \nonumber
\newcommand*{\yesnumber}{\addtocounter{equation}{1}\tag{\theequation}}

% Mathop spacing
\newcommand*{\mathopspace}{\mathop{}\!}

% More vertical space for lines containing fractions etc.
\newcommand*{\tallspace}{0.3em}

% Asymptotically
\newcommand*{\asy}{\sim}

% Defined equal to
\newcommand*{\defeq}{:=}

% Identically equal to
\newcommand*{\ideq}{\equiv}

% Bold vectors
\renewcommand*{\vec}[1]{\boldsymbol{\mathbf{#1}}}

% Bracketing
% --------
% Style guide for optional sizing argument:
%   *       for fraction-height, radical-height, or superscript-height content
%   [\big]  for text-height content which is bulky
%   [\Big]  for text-height content which a derivative (\pder) is acting upon
\DeclarePairedDelimiter{\curlybr}{\{}{\}}
\DeclarePairedDelimiter{\roundbr}{(}{)}
\DeclarePairedDelimiter{\squarebr}{[}{]}

% Evaluation
% --------
% Style guide for sizing argument (do not omit since left delimiter is "."):
%   *       for fraction-height or radical-height content
%   [\big]  for text-height or superscript-height content
\DeclarePairedDelimiter{\eval}{.}{\rvert}

% Del operator
\newcommand*{\del}{\mathopspace\vec{\nabla}}

% Partial differential
\newcommand*{\pd}{\mathopspace\partial}

% Partial derivative
% --------
%   #1  [order] default ""
%   #2  {dependent variable}
%   #3  {independent variable}
\newcommand*{\pder}[3][]{
  \frac{\pd\optionalsup{#1}#2}{{\pd #3}\optionalsup{#1}}
}

% Total differential
% --------
% Style guide:
%   {\td x}   for the denominator of a derivative after "/"
%             (see <https://tex.stackexchange.com/a/68303>)
%   {\td s}^2 for the square of a differential
%             with thin space "\," before or after as necessary
\newcommand*{\td}{\mathopspace\mathrm{d}}

% Total derivative
% --------
%   #1  [order] default ""
%   #2  {dependent variable}
%   #3  {independent variable}
\newcommand*{\tder}[3][]{
  \frac{\td\optionalsup{#1}#2}{{\td #3}\optionalsup{#1}}
}

% Component of a vector
\newcommand*{\comp}[2]{#1_{#2}}

% Cross product
\newcommand*{\crossp}{\boldsymbol{\times}}

% Dot product
\newcommand*{\dotp}{\boldsymbol{\cdot}}

% Absolute value
\DeclarePairedDelimiter{\abs}{\lvert}{\rvert}

% Norm
\DeclarePairedDelimiter{\norm}{\lVert}{\rVert}

% Position vector
\newcommand*{\positionvec}{\vec{r}}

% Local (orthogonal) basis vector
\newcommand*{\localvec}[1]{\mathopspace\vec{h}_{#1}}

% Scale factor
\newcommand*{\scalefac}[1][]{h\optionalsub{#1}}

% Orthonormal basis vector
\newcommand*{\basisvec}[1]{\mathopspace\vec{a}_{#1}}

% Normal vector
\newcommand*{\normalvec}{\mathopspace\vec{n}}

% Tangent vector
\newcommand*{\tangentvec}{\mathopspace\vec{t}}

% Scaled quantity
\newcommand*{\scaled}[1]{\hat{#1}}
\newcommand*{\scaleddel}{\mathopspace\scaled{\del}}
\newcommand*{\scalingmarks}{hats}

% Text quantity
\newcommand*{\textq}[1]{\curlybr*{\text{#1}}}

% Dimensionless group
\newcommand*{\group}[1]{\squarebr*{#1}}

% Generic constant
\newcommand*{\const}{\mathrm{const}}

% Natural exponential base
\newcommand*{\ee}{\mathrm{e}}

% Gamma function
\newcommand*{\gamm}{\mathopspace\Gamma}

% Hypergeometric function
\newcommand*{\hypergeo}{\mathopspace{_2F_1}}

% Order (Big-O)
\newcommand*{\order}{\mathopspace O}

% Conductivity
\newcommand*{\conduc}{k}

% Emissivity
\newcommand*{\emiss}{\epsilon}

% Stefan--Boltzmann constant
\newcommand*{\stefan}{\sigma}

% Bath subscript
\newcommand*{\bath}{\bullet}

% Inflection subscript
\newcommand*{\infl}{\mathrm{i}}

% Abbreviation for natural
\newcommand*{\nat}{\natural}

% Important equation box
\newcommand*{\importantpadding}{\hspace{0.7em}}
\newcommand{\importantbox}[1]{%
  \fbox{%
    \importantpadding
      #1%
    \importantpadding
  }%
}
\newenvironment{important}[1]{%
  \setkeys{EmphEqEnv}{#1}%
  \setkeys{EmphEqOpt}{box=\importantbox}%
  \EmphEqMainEnv
}{%
  \endEmphEqMainEnv
}

% Phantom plus sign
\newcommand*{\+}{\phantom{+}}

%%%%%%%%%%%%%%%%%%%%%%%%%%%%%%%%%%%%%%%%%%%%%%%%%%%%%%%%%%%%%%%%
% Text macros
%%%%%%%%%%%%%%%%%%%%%%%%%%%%%%%%%%%%%%%%%%%%%%%%%%%%%%%%%%%%%%%%

% Italicised term
\newcommand*{\term}[1]{\textit{#1}}

% Bring to attention
\newcommand*{\atten}[1]{\textbf{#1}}

% Code
\newcommand*{\code}[1]{\texttt{#1}}

% Programming language
\newcommand*{\lang}[1]{\textsf{#1}}

% Figure style (in caption)
\newcommand*{\figurestyle}[1]{[#1]}

% Names containing non-ASCII characters
\newcommand*{\lame}{Lam\'e}
\newcommand*{\laplaceyoung}{Laplace--Young}
\newcommand*{\stefanboltz}{Stefan--Boltzmann}

% Italicised foreign phrases
\newcommand*{\foreign}[1]{\textit{#1}}
\newcommand*{\foreignspace}{\;}
\makeatletter
  \newcommand*{\adhoc}{\foreign{ad\foreignspace hoc}}
  \newcommand*{\etalraw}{et\foreignspace al}
  \newcommand*{\etal}{%
    \@ifnextchar.{%
      \foreign{\etalraw}%
    }{%
      \foreign{\etalraw.\@}%
    }%
  }
\makeatother

%%%%%%%%%%%%%%%%%%%%%%%%%%%%%%%%%%%%%%%%%%%%%%%%%%%%%%%%%%%%%%%%
% TODO and temporary
%%%%%%%%%%%%%%%%%%%%%%%%%%%%%%%%%%%%%%%%%%%%%%%%%%%%%%%%%%%%%%%%

% Line numbers
\usepackage{lineno}
\linenumbers

% STRUCTURE
% + Boundary tracing ODE (integrate this)
%   -> Traced boundaries (patch these curves)
%   -> Radiation boundaries (join with T = const)
%   -> Domains

% TODO list
% + Append ($+$) or ($-$) to upper and lower for clarity.
% + Check acronym/abbreviation spacings: abbr.\@ ABC\@.
% + Check display equation punctuation and enforce consistency
% + Reduce spacing between figure content and caption
% + Coerce figures to be on the same page turn if possible
%   i.e. [even-page | odd-page]
% + Perhaps put {fig:plane-traced-boundaries-patched} and {fig:plane-domains}
%   one above the other on the same page,
%   so that there is exact horizontal alignment.
% + Show DOIs in bibliography
% + Show NIST database URL in bibliography
% + Title page
% + Bureaucratic stuff (declarations etc.)
% + Acknowledgements
% + Fix "Underfull \vbox (badness ...)" warnings
% + Prefer "Background" over "BackgroundDarker" style (see FigureStyles.wl)
%   where possible
% + Fix unwanted lines in RegionPlot exported to PDF
%   See <https://mathematica.stackexchange.com/q/2629> etc.
% + Remove line numbers
% + Make the PDF file size as small as possible
% + Remove this TODO section

%%%%%%%%%%%%%%%%%%%%%%%%%%%%%%%%%%%%%%%%%%%%%%%%%%%%%%%%%%%%%%%%
% Document
%%%%%%%%%%%%%%%%%%%%%%%%%%%%%%%%%%%%%%%%%%%%%%%%%%%%%%%%%%%%%%%%

\begin{document}

\frontmatter

\tableofcontents

\mainmatter

\chapter{Introduction and Literature Review}
\label{ch:introduction}

The usual approach
to investigating the effect of domain shape
on solutions to a boundary value problem (BVP)
is simply to solve the BVP for various domains.
For any given domain shape, one seeks solutions
to the associated partial differential equation (PDE) in the interior
which also satisfy the prescribed boundary conditions on the boundary.
In linear problems possessing sufficient symmetry,
the superposition principle
together with separation of variables or integral transforms
typically allows one to complete this task,
but in nonlinear problems,
exact solutions are rare and often only available for simple geometries.
From a numerical viewpoint
it can also be computationally expensive
to consider many different domain shapes,
since the BVP must be solved from scratch for each new domain.

An alternative strategy is to
consider the BVP in reverse:
given a solution to the PDE\@, are there new boundaries
which also satisfy the prescribed boundary conditions?
If so, it is conceivable that these new boundaries may be used
to construct new domains
which admit the same solution to the given BVP\@.
Such an approach has been employed in an \adhoc{} fashion
by Anderssen~\etal~\cite{anderssen-1969-ion-uptake-growing-roots}
to numerically determine the profile of a growing root
and by McNabb~\etal~\cite{mcnabb-1991-theoretical-derivation-freezing-times}
to estimate finite measures of cooling time in ellipsoids.
The first systematic investigation of this
reverse strategy for flux boundary conditions,
known as \emph{boundary tracing},
was carried out by
Anderson~\etal~\cite{anderson-2007-boundary-tracing-i-theory}
with applications to the \laplaceyoung{} equation
of capillarity~\cite{anderson-2006-exact-solutions-laplace-young}
and other PDEs, including the Helmholtz,
constant mean curvature and
Poisson equations~\cite{anderson-2007-boundary-tracing-ii-applications}.

Over a decade has since passed
with little further work in the area,
and the aim of this thesis is to
continue the application of boundary tracing
to physically significant contexts.
I explore its use in thermal radiation problems
and extend the work on capillarity.
I begin by giving a brief introduction to boundary tracing
and reviewing the relevant known results
in thermal radiation and capillarity.

\section{Boundary tracing}
\label{sec:introduction.tracing}

In this section, I provide a quick overview of
the theory of \emph{boundary tracing},
which is well-described by
Anderson~\etal~\cite{anderson-2007-boundary-tracing-i-theory}
with further details given in
Anderson's thesis~\cite{anderson-2002-thesis-boundary-tracing-pdes}.
Consider a BVP\@, consisting of
a PDE in some two-dimensional domain~$\Omega$
along with the boundary condition
\begin{important}{equation}
  \normalvec \dotp \del T = F \roundbr[\big]{x, y, T, \norm{\del T}}
  \label{eq:flux-boundary-condition}
\end{important}
on its boundary~$\pd\Omega$,
where $\normalvec$~is the outward-pointing unit normal
and $F$~is a given \emph{flux function}.
Whereas the usual goal is
to determine the solution~$T$ for a given domain~$\Omega$,
the aim of boundary tracing is
to seek~$\Omega$ for a given~$T$.
Specifically, one takes any known solution~$T = T (x, y)$ to the PDE
and seeks \emph{traced boundaries}, which are curves
along which the flux condition~(\ref{eq:flux-boundary-condition}) holds.
Once the traced boundaries have been found,
they may be used to construct new domains
for which the solution to the BVP is also~$T$.
Surprisingly, this reverse problem
of determining boundaries from a known solution
often has infinitely many answers,
leading to far more new and interesting domain shapes
than expected~\cite{anderson-2007-boundary-tracing-ii-applications}.

The flux condition~(\ref{eq:flux-boundary-condition})
is best understood as a geometric constraint:
at any point~$(x, y)$ in the domain of~$T$,
it determines the possible local orientations for~$\normalvec$,
and therefore, the possible local orientations
of the sought-after traced boundaries
(which have normal~$\normalvec$).
With $\theta$ denoting the angle between~$\normalvec$ and~$\del T$,
the boundary condition~(\ref{eq:flux-boundary-condition}) may be rewritten as
\begin{equation}
  \cos\theta = \frac{F}{\norm{\del T}},
  \label{eq:flux-boundary-condition-cosine}
\end{equation}
and there are three cases at any given point:
\begin{enumerate}
  \item
    If~$\norm{\del T} > \abs{F}$,
    then~(\ref{eq:flux-boundary-condition-cosine})
    has a conjugate pair of solutions in~$\theta$,
    and there are two choices for~$\normalvec$,
    symmetric about~$\del T$,
    corresponding to two possible local orientations
    for the traced boundaries,
    which cross the local $T$-contour at an angle.
  \item
    If~$\norm{\del T} = \abs{F}$,
    then $\normalvec$~is either parallel~($\theta = 0$)
    or antiparallel~($\theta = \pi$) to~$\del T$,
    corresponding to traced boundaries which are tangential
    to the local $T$-contour.
  \item
    If~$\norm{\del T} < \abs{F}$,
    then the right hand side of~(\ref{eq:flux-boundary-condition-cosine})
    exceeds unity in magnitude,
    and there are no solutions in~$\theta$,
    i.e.~traced boundaries do not exist.
\end{enumerate}
With these observations in mind,
the domain of the known solution~$T$
is partitioned into
\begin{itemize}
  \item
    the \emph{viable domain},
    wherein~$\norm{\del T} \ge \abs{F}$
    and there are two possible branches of traced boundaries,
    and
  \item
    the \emph{non-viable domain},
    wherein~$\norm{\del T} < \abs{F}$
    and traced boundaries cannot exist.
\end{itemize}
The border between these two regions is
\begin{itemize}
  \item
    the \emph{terminal curve},
    along which~$\norm{\del T} = \abs{F}$
    and traced boundaries are tangential to the local $T$-contour.
\end{itemize}
A point along the terminal curve is called a \emph{terminal point},
and is one of the following:
\begin{enumerate}
  \item
    An \emph{ordinary} terminal point (Figure~\tbd):
    the local $T$-contour crosses the terminal curve at a non-zero angle.
    The traced boundaries (which are tangential to the local $T$-contour)
    terminate in a cusp at the terminal point,
    for they cannot enter the non-viable domain.
  \item
    A \emph{critical} terminal point:
    the local $T$-contour touches the terminal curve tangentially.
    This results in one of the following:
    \begin{enumerate}
      \item
        The \emph{hyperbolic} case (Figure~\tbd):
        the $T$-contour lies toward the viable side of the terminal curve.
        Two smooth traced boundaries pass through the terminal point,
        at which the traced boundaries,
        the local $T$-contour and the terminal curve
        all touch.
      \item
        The \emph{elliptic} case (Figure~\tbd):
        the $T$-contour lies toward the non-viable side of the terminal curve,
        and no smooth traced boundaries pass through the terminal point.
      \item
        The \emph{degenerate} case:
        the $T$-contour and the entire terminal curve
        are in fact the same curve.
        Thus the terminal curve consists solely of critical terminal points,
        and is therefore called a \emph{critical terminal curve}.
        This curve is itself a traced boundary,
        unto which other traced boundaries attach smoothly.
    \end{enumerate}
    The degenerate case occurs
    when the known solution~$T$ possesses sufficient symmetry.
    Anderson~\etal~\cite{anderson-2007-boundary-tracing-i-theory}
    note that they have not encountered any case
    where a non-discrete proper subset of the terminal curve
    coincides with a $T$-contour;
    neither have I encountered such a case.
\end{enumerate}

To determine the sought-after traced boundaries,
simply choose an appropriate coordinate system and parametrisation,
rewrite the flux condition~(\ref{eq:flux-boundary-condition})
as an ordinary differential equation (ODE) for the traced boundaries,
and integrate.
By patching together the traced boundaries which result,
new domains may be constructed
which also admit the solution~$T$ to the given BVP\@.
The only restriction on the manner of patching
is that the boundary normal~$\normalvec$ have consistent orientation
with respect to the flux condition~(\ref{eq:flux-boundary-condition}),
and it is here that
the distinction between ordinary and critical terminal points
becomes important.
Generally speaking one must avoid ordinary terminal points;
the cusp formed by the two traced boundaries
will not have consistent boundary orientation,
except possibly for vanishing or discontinuous~$F$.

Anderson~\etal~\cite{anderson-2007-boundary-tracing-i-theory}
have also derived results for the curvature of traced boundaries
and given a very neat analysis of boundary tracing
by mapping the curves onto the manifold~$z^2 = (\del T)^2 - F^2$.
While both of these are of utmost theoretical importance
(indeed a proper understanding of and classification system for
critical terminal points was a result of the manifold analysis),
they are not required for the boundary tracing work in this thesis.

To summarise, boundary tracing is a method in which
a known solution to a PDE is used
to generate new domains which admit the same solution
to the associated BVP
with flux boundary condition~(\ref{eq:flux-boundary-condition}).
It is fitting to conclude this overview with the observation that
boundary tracing is not a perturbative method.
No approximation is required
to convert the flux condition~(\ref{eq:flux-boundary-condition})
into an ODE for the traced boundaries,
and although numerical procedures may be required
to do the subsequent integration,
the underlying theory is exact.

\section{Thermal radiation}
\label{sec:introduction.radiation}

In outer space,
thermal radiation is the primary means of disposing of waste heat.
An archetypal steady-state problem consists of
determining the temperature profile~$T$ of a conducting object~$\Omega$,
with heat generated internally or towards one end
and expelled into vacuum via radiation from its surface.

Assuming that the object is homogeneous and isotropic,
the steady conduction within~$\Omega$
is simply described by Laplace's equation
\begin{equation}
  \del^2 T = 0,
  \label{eq:laplace-steady-conduction}
\end{equation}
but on the portion of its surface~$\pd \Omega$ where radiation occurs,
the \stefanboltz{} law implies that
\begin{equation}
  \normalvec \dotp \del T = - c T^4,
  \label{eq:radiation-boundary-condition}
\end{equation}
where
\begin{equation}
  c = \frac{\emiss \stefan}{\conduc},
  \label{eq:radiation-constant}
\end{equation}
with $\conduc$~being the conductivity of the object,
$\emiss$~the emissivity of its surface
and $\stefan$~the \stefanboltz{} constant.

This boundary value problem is not straightforward
due to the nonlinearity of
the radiation condition~(\ref{eq:radiation-boundary-condition}).
The usual treatment in the literature has been to consider thin geometries
for which the problem is effectively one-dimensional,
so that conduction~(\ref{eq:laplace-steady-conduction})
and radiation~(\ref{eq:radiation-boundary-condition})
may be lumped into a single ODE of the form
\begin{equation}
  \textq{derivatives of~$T$} - c T^4 = 0.
  \label{eq:conduction-radiation-lumped}
\end{equation}
Indeed this has been the approach taken in the analytical investigations of
Liu~\cite{liu-1960-minimum-rectangular-radiating-fins},
Wilkins~\cite{
  wilkins-1960-minumum-mass-fins-radiation,
  wilkins-1961-minimum-mass-fins-thickness,
  wilkins-1962-minimum-mass-fins-gradients,
  wilkins-1974-optimum-shapes-convection-radiation
},
and
Shouman~\cite{shouman-1968-exact-radiation-convection-fin},
and in numerical work by
Chambers \&~Somers~\cite{chambers-1959-radiation-fin-efficiency-circular},
Lieblein~\cite{lieblein-1959-radiant-fin-constant-thickness},
Bartas \&~Sellers~\cite{bartas-1960-radiation-fin-effectiveness},
and
Keller \&~Holdredge~\cite{keller-1970-radiation-annular-fins-trapezoidal}.
While such a simplification is an appropriate choice
for the study of thin, heat-rejecting fins on spacecraft
(where weight minimisation and thinness are desirable),
it has arguably been a necessity
for making the BVP~(\ref{eq:laplace-steady-conduction})
and~(\ref{eq:radiation-boundary-condition}) analytically tractable;
I have not come across any analytical treatment
of a conduction--radiation problem
in which the nonlinearity appears as a proper boundary term
in the flux condition~(\ref{eq:radiation-boundary-condition}),
rather than as a volumetric term in an ODE
of the form~(\ref{eq:conduction-radiation-lumped}).

Given the relative abundance of known solutions
to Laplace's equation~(\ref{eq:laplace-steady-conduction}),
and the limited amount of progress which can be made
using conventional techniques
for the nonlinear boundary condition~(\ref{eq:radiation-boundary-condition}),
boundary tracing is a most suitable method
for tackling the BVP~(\ref{eq:laplace-steady-conduction})
and~(\ref{eq:radiation-boundary-condition}),
being an analytical approach
which does not require reducing the problem to one dimension.

\section{Capillary wedges}

\section{Thesis overview}

\chapter{Curvilinear boundary tracing}
\label{ch:curvilinear}

The actual determination of the sought-after traced boundaries involves
recasting the flux boundary condition~(\ref{eq:flux-boundary-condition})
as a first-order ODE (or system of ODEs) for the traced boundaries.
In practice, the appropriate coordinate system and parametrisation
will depend on the geometry of the specific problem at hand.

Whilst Anderson~\etal~\cite{%
  anderson-2007-boundary-tracing-i-theory,%
  anderson-2007-boundary-tracing-ii-applications%
}
did not restrict themselves to Cartesian coordinates in analytical work,
the boundary tracing ODE was separately derived
for each new coordinate system encountered;
no generalised version was given.
As for numerical boundary tracing using arc-length parametrisation,
Anderson~\cite{anderson-2002-thesis-boundary-tracing-pdes}
only considered Cartesian coordinates.

In preparation for the various coordinate systems
which will be used in this thesis,
I derive in this chapter the boundary tracing ODE
for coordinate parametrisation
and the corresponding system of ODEs for arc-length parametrisation,
with both applicable in any two-dimensional orthogonal coordinate system.

\emph{In the interest of keeping this thesis wholly self-contained,}
I first go through the foundational definitions of
orthogonal curvilinear coordinates and vector calculus
in Sections~\ref{sec:curvilinear.orthogonal}
and~\ref{sec:curvilinear.calculus},
before proceeding to the derivations of the boundary tracing ODEs
in Section~\ref{sec:curvilinear.tracing}.

\section{Orthogonal curvilinear coordinates}
\label{sec:curvilinear.orthogonal}

First I run through some standard definitions and terminology.

Let~$(\basisvec{x}, \basisvec{y})$ be the standard basis vectors
of the usual Cartesian coordinates~$(x, y)$.
The \defin{position vector} is
\begin{equation}
  \positionvec \defeq x \basisvec{x} + y \basisvec{y},
  \label{eq:position-vector-cartesian}
\end{equation}
and since $\basisvec{x}$ and~$\basisvec{y}$ are globally constant,
observe that
\begin{align}
  \basisvec{x} &\ideq \pder{\positionvec}{x},
    \label{eq:x-basis-vector} \\[\tallspace]
  \basisvec{y} &\ideq \pder{\positionvec}{y}.
    \label{eq:y-basis-vector}
\end{align}
Now consider the \defin{curvilinear coordinates}~$(u, v)$
given by the transformation
\begin{align}
  x &= x (u, v), \label{eq:x-transformation-curvilinear} \\
  y &= y (u, v). \label{eq:y-transformation-curvilinear}
\end{align}
A \defin{local basis}~$(\localvec{u}, \localvec{v})$
arises from the derivatives of position
in analogy to~(\ref{eq:x-basis-vector}) and~(\ref{eq:y-basis-vector}),
\begin{align}
  \localvec{u} &
    \defeq \pder{\positionvec}{u}
    \ideq \pder{x}{u} \basisvec{x} + \pder{y}{u} \basisvec{y},
      \label{eq:u-local-basis-vector} \\[\tallspace]
  \localvec{v} &
    \defeq \pder{\positionvec}{v}
    \ideq \pder{x}{v} \basisvec{x} + \pder{y}{v} \basisvec{y}.
      \label{eq:v-local-basis-vector}
\end{align}
It is assumed that the curvilinear coordinates~$(u, v)$ are \defin{orthogonal},
meaning the local basis vectors are orthogonal to each other:
\begin{equation}
  \localvec{u} \dotp \localvec{v} \ideq 0.
  \label{eq:local-basis-orthogonal}
\end{equation}
The magnitudes of the local basis vectors may not be unity;
in fact, should either~$u$ or~$v$ not have the dimensions of length,
the corresponding local basis vector will not even be dimensionless.
Therefore, one defines the \defin{scale factors}
(or \defin{\lame{} coefficients})
\begin{align}
  \scalefac[u] &\defeq \norm{\localvec{u}}, \label{eq:u-scale-factor} \\
  \scalefac[v] &\defeq \norm{\localvec{v}}, \label{eq:v-scale-factor}
\end{align}
and then normalises the local basis
to obtain the \defin{local orthonormal basis},
\begin{align}
  \basisvec{u} &\defeq \frac{\localvec{u}}{\scalefac[u]},
      \label{eq:u-basis-vector} \\[\tallspace]
  \basisvec{v} &\defeq \frac{\localvec{v}}{\scalefac[v]},
      \label{eq:v-basis-vector}
\end{align}
for which
\begin{equation}
  \basisvec{u} \dotp \basisvec{v} \ideq 0
  \label{eq:orthonormal-basis-orthogonal}
\end{equation}
and
\begin{equation}
  \norm{\basisvec{u}} \ideq \norm{\basisvec{v}} \ideq 1.
  \label{eq:orthonormal-basis-normalised}
\end{equation}

\begin{figure}
  \centredfigurecontent[width=0.4\textwidth]{%
    orthogonal-curvilinear-coordinates%
  }{
    Infinitesimal rectangle formed by differential displacements
    in the orthogonal curvilinear coordinates~$(u, v)$.
  }
\end{figure}

\section{Vector calculus}
\label{sec:curvilinear.calculus}

Next, I go through the standard mathematical machinery required
to do calculus in an orthogonal coordinate system.

\subsection{Differential displacement}
\label{sec:curvilinear.calculus.displacement}

Consider the $u$-contour and the $v$-contour through a point~$(u, v)$.
Independently incrementing~$u$ by~$\td u$ and~$v$ by~$\td v$
results in two new contours,
corresponding to each of the new coordinates.
Since the coordinate system is orthogonal,
the four contours (two old and two new)
shall mark out an infinitesimal rectangle,
aligned with the local orthonormal basis
and with sides of length~$\scalefac[u] \td u$ and~$\scalefac[v] \td v$
(Figure~\ref{fig:orthogonal-curvilinear-coordinates}).
Indeed the \defin{differential displacement} is
\begin{align*}
  \td \positionvec
  &\ideq
    \pder{\positionvec}{u} \td u + \pder{\positionvec}{v} \td v
    \\[\tallspace]
  &\ideq \localvec{u} \td u + \localvec{v} \td v \\
  &\ideq \scalefac[u] \td u \basisvec{u} + \scalefac[v] \td v \basisvec{v}.
    \yesnumber
    \label{eq:differential-displacement}
\end{align*}
The area of the infinitesimal rectangle is
the \defin{differential area element}
\begin{equation}
  \td V \ideq \scalefac[u] \scalefac[v] \td u \td v.
  \label{eq:differential-area-element}
\end{equation}

\subsection{Gradient}
\label{sec:curvilinear.calculus.gradient}

Consider a scalar field~$T$
(which, in the context of boundary tracing, shall be the known solution).
By the chain rule, the independent increments~$\td u$ and~$\td v$
effect the differential change
\begin{align*}
  \td T
  &\ideq \pder{T}{u} \td u + \pder{T}{v} \td v \\[\tallspace]
  &\ideq
    \roundbr*{\frac{1}{\scalefac[u]} \pder{T}{u}}
    \roundbr[\big]{\scalefac[u] \td u}
      +
    \roundbr*{\frac{1}{\scalefac[v]} \pder{T}{v}}
    \roundbr[\big]{\scalefac[v] \td v}.
    \yesnumber
    \label{eq:scalar-differential-change}
\end{align*}
By the definition of gradient,
and using~(\ref{eq:differential-displacement}),
\begin{align*}
  \td T
  &\ideq \del T \dotp \td \positionvec \\
  &\ideq
    \del T
      \dotp
    \roundbr[\big]{
      \scalefac[u] \td u \basisvec{u}
        +
      \scalefac[v] \td v \basisvec{v}
    }.
    \yesnumber
    \label{eq:scalar-differential-change-gradient}
\end{align*}
Comparing~(\ref{eq:scalar-differential-change})
and~(\ref{eq:scalar-differential-change-gradient}),
it follows that the \defin{gradient} of~$T$ is given by
\begin{equation}
  \del T \ideq
    \frac{1}{\scalefac[u]} \pder{T}{u} \basisvec{u}
      +
    \frac{1}{\scalefac[v]} \pder{T}{v} \basisvec{v}.
  \label{eq:gradient}
\end{equation}

\subsection{Divergence}
\label{sec:curvilinear.calculus.divergence}

Consider a vector field~$%
\vec{E} \defeq \comp{E}{u} \basisvec{u} + \comp{E}{v} \basisvec{v}$,
and the infinitesimal rectangle
of Figure~\ref{fig:orthogonal-curvilinear-coordinates},
whose sides are of length~$\scalefac[u] \td u$ and~$\scalefac[v] \td v$.

To first order,
the net flux across the two edges normal to~$\basisvec{u}$
(which have length~$\scalefac[v] \td v$) is
\[
  \pder{}{u} \roundbr[\Big]{
    \comp{E}{u} \cdot \scalefac[v] \td v
  } \cdot \td u,
\]
and the net flux across the two edges normal to~$\basisvec{v}$
(which have length~$\scalefac[u] \td u$) is
\[
  \pder{}{v} \roundbr[\Big]{
    \comp{E}{v} \cdot \scalefac[u] \td u
  } \cdot \td v.
\]
Summing these results in the flux across the entire perimeter of the rectangle,
which, divided by its area,
the area element~(\ref{eq:differential-area-element}),
yields the \defin{divergence} of~$\vec{E}$,
\begin{equation}
  \del \dotp \vec{E} \ideq
    \frac{1}{\scalefac[u] \scalefac[v]}
    \squarebr*{
      \pder{}{u} \roundbr[\Big]{\scalefac[v] \comp{E}{u}}
        +
      \pder{}{v} \roundbr[\Big]{\scalefac[u] \comp{E}{v}}
    }.
  \label{eq:divergence}
\end{equation}

\subsection{Laplacian}
\label{sec:curvilinear.calculus.laplacian}

Composing the divergence~(\ref{eq:divergence})
and the gradient~(\ref{eq:gradient}) (with~$\vec{E} = \del T$),
it follows that the \defin{Laplacian} of~$T$ is given by
\begin{equation}
  \del^2 T \defeq \del \dotp \del T \ideq
    \frac{1}{\scalefac[u] \scalefac[v]}
    \squarebr*{
      \pder{}{u} \roundbr*{\frac{\scalefac[v]}{\scalefac[u]} \pder{T}{u}}
        +
      \pder{}{v} \roundbr*{\frac{\scalefac[u]}{\scalefac[v]} \pder{T}{v}}
    }.
  \label{eq:laplacian}
\end{equation}
In particular,
if the two scale factors~$\scalefac[u]$ and~$\scalefac[v]$ are equal,
say~$\scalefac[u] \ideq \scalefac[v] \ideq \scalefac$,
this simplifies to
\begin{equation}
  \del^2 T \ideq
    \frac{1}{\scalefac^2}
    \squarebr*{\pder[2]{T}{u} + \pder[2]{T}{v}}.
  \label{eq:laplacian-scale-factors-equal}
\end{equation}

\subsection{Boundary normal}
\label{sec:curvilinear.calculus.normal}

The \defin{tangent vector}~$\tangentvec$ to a curve
is given by the unit vector of
the differential displacement~(\ref{eq:differential-displacement}),
i.e.\@
\begin{equation}
  \tangentvec \ideq
    \frac{
      \scalefac[u] \td u \basisvec{u}
        +
      \scalefac[v] \td v \basisvec{v}
    }{
      \td s
    },
  \label{eq:tangent-vector}
\end{equation}
where $\td s$~is the \defin{differential arc length}
\begin{equation}
  \td s \defeq \norm{\td \positionvec} \ideq
  \sqrt{
    \roundbr[\big]{\scalefac[u] \td u}^2
      +
    \roundbr[\big]{\scalefac[v] \td v}^2
  }.
  \label{eq:differential-arc-length}
\end{equation}
The normal vector~$\normalvec$ to a curve
is perpendicular to the tangent~$\tangentvec$;
therefore
\begin{equation}
  \normalvec \ideq
    \frac{
      \scalefac[v] \td v \basisvec{u}
        -
      \scalefac[u] \td u \basisvec{v}
    }{
      \td s
    },
  \label{eq:normal-vector}
\end{equation}
up to sign.

\subsection{Abbreviations}
\label{sec:curvilinear.calculus.abbreviations}

Since the scale factors~$\scalefac[u]$ and~$\scalefac[v]$ appear regularly,
it is helpful to define abbreviations
before proceeding any further.

For the components of the differential displacement
vector~(\ref{eq:differential-displacement}),
which are the side lengths of the infinitesimal rectangle
of Figure~\ref{fig:orthogonal-curvilinear-coordinates},
let
\begin{align}
  \td \mu &\defeq \scalefac[u] \td u,
    \label{eq:differential-displacement-u-component} \\
  \td \nu &\defeq \scalefac[v] \td v,
    \label{eq:differential-displacement-v-component}
\end{align}
so that
\begin{equation}
  \td \positionvec \ideq \td \mu \basisvec{u} + \td \nu \basisvec{v}
  \label{eq:differential-displacement-abbreviated}
\end{equation}
and
\begin{equation}
  \td s \ideq \sqrt{{\td \mu}^2 + {\td \nu}^2}.
  \label{eq:differential-arc-length-abbreviated}
\end{equation}
For the components of the gradient vector~(\ref{eq:gradient}),
write
\begin{align}
  P &\defeq \frac{1}{\scalefac[u]} \pder{T}{u},
    \label{eq:gradient-u-component} \\[\tallspace]
  Q &\defeq \frac{1}{\scalefac[v]} \pder{T}{v},
    \label{eq:gradient-v-component}
\end{align}
so that
\begin{equation}
  \del T \ideq P \basisvec{u} + Q \basisvec{v}.
  \label{eq:gradient-abbreviated}
\end{equation}
Finally, for the components of
the tangent vector~(\ref{eq:tangent-vector}), define
\begin{align}
  \alpha &\defeq \tder{\mu}{s} \ideq \frac{\scalefac[u] \td u}{\td s},
    \label{eq:tangent-u-component} \\[\tallspace]
  \beta &\defeq \tder{\nu}{s} \ideq \frac{\scalefac[v] \td v}{\td s},
    \label{eq:tangent-v-component}
\end{align}
so that
\begin{align}
  \tangentvec
  &\ideq \frac{\td \mu \basisvec{u} + \td \nu \basisvec{v}}{\td s}
  \ideq \alpha \basisvec{u} + \beta \basisvec{v},
    \label{eq:tangent-vector-abbreviated} \\[\tallspace]
  \normalvec
  &\ideq \frac{\td \nu \basisvec{u} - \td \mu \basisvec{v}}{\td s}
  \ideq \beta \basisvec{u} - \alpha \basisvec{v}.
    \label{eq:normal-vector-abbreviated}
\end{align}
Note that
\begin{align}
  P^2 + Q^2 &\ideq (\del T)^2, \label{eq:gradient-pythagoras} \\
  \alpha^2 + \beta^2 &\ideq 1. \label{eq:tangent-pythagoras}
\end{align}

\section{Boundary tracing ODE}
\label{sec:curvilinear.tracing}

With the workings of orthogonal curvilinear coordinates now established,
I derive in this section
the boundary tracing ODE for coordinate parametrisation
and the corresponding system of ODEs for arc-length parametrisation.
Recall that the sought-after traced boundaries are to satisfy
the boundary condition~(\ref{eq:flux-boundary-condition}),
in which $T$~is the chosen known solution (to the PDE under consideration)
and $F$~is the prescribed flux function.

It is convenient to define the \defin{viability function}
\begin{important}{equation}
  \Phi \defeq (\del T)^2 - F^2.
  \label{eq:viability-function}
\end{important}
This is not only algebriacally useful,
but also physically meaningful;
the viable domain is the region~$\Phi \ge 0$,
the non-viable domain, $\Phi < 0$,
and the terminal curve, $\Phi = 0$.

\subsection{Coordinate parametrisation}
\label{sec:curvilinear.tracing.coordinate}

Suppose the traced boundaries are to be parametrised
in the form~$v = v (u)$.
Using~(\ref{eq:normal-vector-abbreviated}),
(\ref{eq:differential-arc-length-abbreviated}),
and~(\ref{eq:gradient-abbreviated}),
the boundary condition~(\ref{eq:flux-boundary-condition}) becomes
\[
  \frac{
    \td \nu \basisvec{u} - \td \mu \basisvec{v}
  }{
    \sqrt{{\td \mu}^2 + {\td \nu}^2}
  }
    \dotp
  \roundbr[\big]{P \basisvec{u} + Q \basisvec{v}}
    =
  F,
\]
which expands into the quadratic equation
\begin{equation}
  \roundbr*{P^2 - F^2} \, {\td \nu}^2
  - 2 P Q \td \nu \td \mu
  + \roundbr*{Q^2 - F^2} \, {\td \mu}^2
    =
  0.
  \label{eq:tracing-coordinate-parametrisation-quadratic}
\end{equation}
Solving this for~$\td \nu / {\td \mu}$ results in the boundary tracing ODE
\begin{equation}
  \tder{\nu}{\mu} = \frac{P Q \pm F \sqrt{\Phi}}{P^2 - F^2},
  \label{eq:tracing-ode-coordinate-parametrisation-nu}
\end{equation}
or
\begin{important}{equation}
  \tder{v}{u} =
    \frac{\scalefac[u]}{\scalefac[v]}
      \cdot
    \frac{P Q \pm F \sqrt{\Phi}}{P^2 - F^2}.
  \label{eq:tracing-ode-coordinate-parametrisation-v}
\end{important}
Alternatively, for the parametrisation~$u = u (v)$,
one solves for~$\td \mu / {\td \nu}$, giving
\begin{equation}
  \tder{\mu}{\nu} = \frac{P Q \mp F \sqrt{\Phi}}{Q^2 - F^2},
  \label{eq:tracing-ode-coordinate-parametrisation-mu}
\end{equation}
or
\begin{important}{equation}
  \tder{u}{v} =
    \frac{\scalefac[v]}{\scalefac[u]}
      \cdot
    \frac{P Q \mp F \sqrt{\Phi}}{Q^2 - F^2}.
  \label{eq:tracing-ode-coordinate-parametrisation-u}
\end{important}
In both cases, the two possible choices of sign
correspond to the two branches of traced boundaries
which exist in the viable domain.
Note also the occurrence of~$\sqrt{\Phi}$,
corresponding to the absence of traced boundaries
in the non-viable domain~$\Phi < 0$.

Whichever of the symbols~$\pm$ and~$\mp$ occurs,
the \defin{upper} branch shall refer to
the traced boundaries for the upper sign in that symbol,
and the \defin{lower} branch to those for the lower sign
(e.g.~in~(\ref{eq:tracing-ode-coordinate-parametrisation-u}),
the upper branch corresponds to choosing the minus sign,
while the lower branch corresponds to choosing the plus sign).

\subsection{Arc-length parametrisation}
\label{sec:curvilinear.tracing.arc-length}

In numerical boundary tracing,
the coordinate parametrisations~$v = v (u)$ and~$u = u (v)$
will be problematic
if~$\td v / {\td u}$ or~$\td u / {\td v}$ ever become infinite,
corresponding to traced boundaries
which are parallel to~$\basisvec{v}$ or~$\basisvec{u}$.

To avoid such singularities in numerical work, one should instead use
the arc-length parametrisation~$u = u (s)$, $v = v(s)$
to put the two coordinates~$u$ and~$v$ on an equal footing.
Using~(\ref{eq:normal-vector-abbreviated})
and~(\ref{eq:gradient-abbreviated}),
the boundary condition~(\ref{eq:flux-boundary-condition}) becomes
\[
  \roundbr[\big]{\beta \basisvec{u} - \alpha \basisvec{v}}
    \dotp
  \roundbr[\big]{P \basisvec{u} + Q \basisvec{v}}
    =
  F,
\]
which, together with the unit-speed condition~(\ref{eq:tangent-pythagoras}),
forms the system%
\footnote{
  This system is also given
  in Anderson's thesis~\cite{anderson-2002-thesis-boundary-tracing-pdes},
  but only in Cartesian coordinates.
  Moreover, the system is passed directly to a numerical integrator
  without first solving for~$\alpha$ and~$\beta$,
  so that at every step of the integration,
  a numerical comparison with the previous root
  is required to select the correct root among the two possible.
  A better approach is to separate the two branches analytically
  before undertaking any numerical procedures,
  as I have done per~(\ref{eq:tracing-ode-arc-length-parametrisation-mu})
  and~(\ref{eq:tracing-ode-arc-length-parametrisation-nu}).
}
\begin{align}
  P \beta - Q \alpha &= F,
    \label{eq:tracing-arc-length-parametrisation-flux-boundary-condition} \\
  \alpha^2 + \beta^2 &= 1.
    \label{eq:tracing-arc-length-parametrisation-unit-speed}
\end{align}
Solving~(\ref{eq:tracing-arc-length-parametrisation-flux-boundary-condition})
and~(\ref{eq:tracing-arc-length-parametrisation-unit-speed})
for~$\alpha$ and~$\beta$, one obtains
the boundary tracing system of ODEs under arc-length parametrisation,
\begin{align}
  \tder{\mu}{s} &\ideq \alpha = \frac{-Q F \pm P \sqrt{\Phi}}{(\del T)^2},
    \label{eq:tracing-ode-arc-length-parametrisation-mu} \\[\tallspace]
  \tder{\nu}{s} &\ideq \beta = \frac{\+P F \pm Q \sqrt{\Phi}}{(\del T)^2},
    \label{eq:tracing-ode-arc-length-parametrisation-nu}
\end{align}
or
\begin{important}{align}
  \tder{u}{s} &= \frac{-Q F \pm P \sqrt{\Phi}}{\scalefac[u] (\del T)^2},
    \label{eq:tracing-ode-arc-length-parametrisation-u} \\[\tallspace]
  \tder{v}{s} &= \frac{\+P F \pm Q \sqrt{\Phi}}{\scalefac[v] (\del T)^2}.
    \label{eq:tracing-ode-arc-length-parametrisation-v}
\end{important}
As was the case for coordinate parametrisation
(Section~\ref{sec:curvilinear.tracing.coordinate}),
notice the two possible choices of sign
(which correspond to the two branches of traced boundaries)
and the occurrences of~$\sqrt{\Phi}$
(which correspond to the absence of traced boundaries
in the non-viable domain~$\Phi < 0$).

\subsection{Arc-length parametrisation for contours}
\label{sec:curvilinear.tracing.arc-length-contours}

When classifying terminal points as either ordinary or critical
(and the critical cases thereof as hyperbolic, elliptic, or degenerate),
it is useful to visualise the terminal curve (which is the contour~$\Phi = 0$)
and the contours of the known solution~$T$,
because it is the manner of their intersection,
i.e.~whether they cross or touch,
which determines this.

For numerical computation of contours,
arc-length parametrisation is again more robust
than coordinate parametrisation
(which is unable to handle singularities
in~$\td v / {\td u}$ and~$\td u / {\td v}$).
Suppose a $T$-contour is to be parametrised by arc length,
i.e.~$u = u (s)$, $v = v(s)$.
By definition, there holds
\[
  \td T \ideq \del T \dotp \td \positionvec = 0
\]
along the contour,
which, using~(\ref{eq:gradient-abbreviated})
and~(\ref{eq:differential-displacement-abbreviated}),
becomes
\[
  P \td \mu + Q \td \nu = 0,
\]
or
\begin{equation}
  P \alpha + Q \beta = 0.
  \label{eq:contour-arc-length-parametrisation-contour}
\end{equation}
Combining this with the unit-speed condition~(\ref{eq:tangent-pythagoras}),
the resulting system in~$\alpha$ and~$\beta$
solves to give the system of ODEs
\begin{align}
  \tder{\mu}{s} &\ideq \alpha = \frac{\pm Q}{\norm{\del T}},
    \label{eq:contour-ode-arc-length-parametrisation-mu} \\[\tallspace]
  \tder{\nu}{s} &\ideq \beta = \frac{\mp P}{\norm{\del T}}
    \label{eq:contour-ode-arc-length-parametrisation-nu},
\end{align}
or
\begin{important}{align}
  \tder{u}{s} &= \frac{\pm Q}{\scalefac[u] \norm{\del T}},
    \label{eq:contour-ode-arc-length-parametrisation-u} \\[\tallspace]
  \tder{v}{s} &= \frac{\mp P}{\scalefac[v] \norm{\del T}}
    \label{eq:contour-ode-arc-length-parametrisation-v},
\end{important}
for a $T$-contour.
If a $\Phi$-contour is instead sought
(such as for the terminal curve~$\Phi = 0$),
simply replace~$T$ with~$\Phi$
(including in the definitions~(\ref{eq:gradient-u-component})
and~(\ref{eq:gradient-v-component})
for the gradient components~$P$ and~$Q$).


\part{Thermal radiation}
\label{pt:radiation}

\chapter{Cartesian coordinates}
\label{ch:cartesian}

In this chapter,
I apply boundary tracing to the thermal radiation problem
by taking known solutions
to Laplace's equation~(\ref{eq:laplace-steady-conduction})
in Cartesian coordinates
and seeking new boundaries along which
the radiation condition~(\ref{eq:radiation-boundary-condition})
is satisfied.

\section{Plane-source solution}
\label{sec:cartesian.plane}

First I consider the simplest non-constant solution
to Laplace's equation~(\ref{eq:laplace-steady-conduction})
in Cartesian coordinates,
\begin{important}{equation}
  T \ideq h_0 x,
  \label{eq:plane-laplace-solution}
\end{important}
corresponding to one-dimensional steady conduction
with constant temperature gradient~$h_0$.
Such would be the equilibrium temperature profile in a slab
with one face held at a fixed temperature
and the other radiating into vacuum,
and the aim of boundary tracing
is to look for other (more interesting) radiation boundaries
which correspond to this same solution.

In the context of thermal radiation,
the temperature~$T$ is to be reckoned on an absolute scale;
therefore, the region~$x < 0$ is unphysical and to be ignored,
as the temperature therein is negative.

\subsection{Scaling}
\label{sec:cartesian.plane.scaling}

While the known solution~(\ref{eq:plane-laplace-solution}) is, by itself,
scale-invariant with respect to both temperature and length,
its coupling with
the radiation boundary condition~(\ref{eq:radiation-boundary-condition})
will determine characteristic temperature and length scales
$\tau$ and~$\lambda$.
Here, scaling is used to remove these parameters
and reduce the problem to its simplest form.

Defining
\begin{align}
  \scaled{T} &\defeq T / \tau, \label{eq:plane-scaled-temperature} \\
  \scaled{x} &\defeq x / \lambda, \label{eq:plane-scaled-x} \\
  \scaled{y} &\defeq y / \lambda, \label{eq:plane-scaled-y}
\end{align}
and noting that
\begin{equation}
  \scaleddel \ideq \lambda \del,
  \label{eq:plane-scaled-del}
\end{equation}
the radiation boundary condition~(\ref{eq:radiation-boundary-condition})
and the known solution~(\ref{eq:plane-laplace-solution})
become
\begin{align}
  \normalvec \dotp \scaleddel \scaled{T}
    &= -\group{c \lambda \tau^3} \scaled{T}^4,
    \label{eq:plane-scaled-radiation-boundary-condition-with-groups}
    \\[\tallspace]
  \scaled{T}
    &\ideq \group{\frac{h_0 \lambda}{\tau}} \scaled{x}.
    \label{eq:plane-scaled-laplace-solution-with-groups}
\end{align}
Setting the two dimensionless groups to unity
results in the temperature and length scales
\begin{align}
  \tau &= \roundbr*{\frac{h_0}{c}}^{1/4},
    \label{eq:plane-temperature-scale} \\[\tallspace]
  \lambda &= \roundbr*{\frac{1}{c {h_0}^3}}^{1/4}.
    \label{eq:plane-length-scale}
\end{align}
These scales may also be arrived at
by seeking the straight-line boundary~$x = \const$
and its corresponding temperature~$T = \const$
along which the radiation condition~(\ref{eq:radiation-boundary-condition})
is satisfied.
Since the temperature gradient
for the known solution~(\ref{eq:plane-laplace-solution})
is $h_0$~everywhere,
there must hold~$h_0 = c T^4$ along this boundary,
and the scales~(\ref{eq:plane-temperature-scale})
and~(\ref{eq:plane-length-scale}) follow immediately.

Of course the straight-line boundary~$x = \lambda$
(or equivalently, $\scaled{x} = 1$)
is rather boring,
and it is through boundary tracing
that more interesting boundaries may be generated,
as will be shown in the next section.

\subsection{Boundary tracing}
\label{sec:cartesian.plane.tracing}

For brevity, \atten{drop all \scalingmarks},
so that the scaled boundary condition~%
  (\ref{eq:plane-scaled-radiation-boundary-condition-with-groups})
and known solution~(\ref{eq:plane-scaled-laplace-solution-with-groups})
become
\begin{important}{align}
  \normalvec \dotp \del T &= -T^4,
    \label{eq:plane-scaled-radiation-boundary-condition} \\
  T &\ideq x.
    \label{eq:plane-scaled-laplace-solution}
\end{important}
The scale factors for Cartesian coordinates are trivial
($\scalefac[x] \ideq \scalefac[y] \ideq 1$),
leading to the following abbreviatory quantities
from Section~\ref{sec:curvilinear.calculus.abbreviations}:
\begin{align}
  P &\ideq \pder{T}{x} \ideq 1,
    \label{eq:plane-gradient-u-component} \\[\tallspace]
  Q &\ideq \pder{T}{y} \ideq 0.
    \label{eq:plane-gradient-v-component}
\end{align}
Comparing the radiation boundary condition~%
  (\ref{eq:plane-scaled-radiation-boundary-condition})
to the generic one~(\ref{eq:flux-boundary-condition}),
the flux function is
\begin{align*}
  F
  &\ideq - T^4 \\
  &\ideq -x^4,
    \yesnumber
    \label{eq:plane-flux-function}
\end{align*}
and it follows that the viability function is
\begin{align*}
  \Phi
  &\ideq (\del T)^2 - F^2 \\
  &\ideq 1 - x^8.
    \yesnumber
    \label{eq:plane-viability-function}
\end{align*}
The viable domain~$\Phi \ge 0$ (excluding the unphysical region~$x < 0$)
is therefore given by the infinite strip
\begin{equation}
  0 \le x \le 1.
  \label{eq:plane-viable-domain}
\end{equation}
The terminal curve~$\Phi = 0$ is also the $T$-contour~$x = 1$
(the boring traced boundary
mentioned in Section~\ref{sec:cartesian.plane.scaling}).
Using the terminology of Section~\ref{sec:introduction.tracing},
$x = 1$~is therefore a critical terminal curve,
and other traced boundaries will attach to it smoothly.

\begin{figure}
  \centredfigurecontent[width=0.5\textwidth]{%
    plane-traced-boundaries%
  }{
    Traced boundaries~(\ref{eq:plane-traced-boundary}).
  }
\end{figure}

The boundary tracing ODE~(\ref{eq:tracing-ode-coordinate-parametrisation-v})
becomes
\begin{important}{equation}
  \tder{y}{x} = \mp \frac{x^4}{\sqrt{1 - x^8}},
  \label{eq:plane-tracing-ode-coordinate-parametrisation-y}
\end{important}
which integrates to give traced boundaries of the form
\begin{important}{equation}
  y =
  \const
    \mp
  \frac{x^5}{5}
    \cdot
  \hypergeo \roundbr*{\frac{1}{2}, \frac{5}{8}; \frac{13}{8}; x^8},
  \label{eq:plane-traced-boundary}
\end{important}
shown in Figure~\ref{fig:plane-traced-boundaries},
where $\hypergeo$~is the hypergeometric function~%
\cite{olver-2010-nist-handbook-mathematical-functions}.
The translational symmetry in the $y$-direction
is a property inherited from the ODE~%
  (\ref{eq:plane-tracing-ode-coordinate-parametrisation-y}).
A local analysis near~$x = 1$ shows that
\begin{equation}
  y = \const \pm \sqrt{\frac{1 - x}{2}} + \order (1 - x)^{3/2},
  \label{eq:plane-traced-boundary-x-near-1}
\end{equation}
so that the traced boundaries do indeed attach smoothly
onto the critical terminal curve~$x = 1$, as expected.
Near~$x = 0$ each pair of traced boundaries forms a thin cusp of the form
\begin{equation}
  y = \const \mp \frac{x^5}{5} + \order \roundbr*{x^{13}}.
  \label{eq:plane-traced-boundary-x-near-0}
\end{equation}
Now each of the traced boundaries is a curve along which
the radiation boundary condition~%
  (\ref{eq:plane-scaled-radiation-boundary-condition})
is satisfied.
More complicated boundaries can be constructed
by patching together several of these curves, or portions thereof,
and the only requirement is that there be consistent orientation.
This requirement is satisfied by indentifying as the interior
the side on which $T$ (which equals~$x$) is greater,
i.e.~the side to the right of each curve.
Figure~\ref{fig:plane-traced-boundaries-patched} shows
a sample of the broad variety of radiation boundaries
which can be produced in this manner.

\begin{figure}
  \centredfigurecontent[width=\textwidth]{%
    plane-traced-boundaries-patched%
  }{
    Radiation boundaries~\figurestyle{black solid} patched together
    using the traced boundaries~%
      (\ref{eq:plane-traced-boundary})~\figurestyle{grey}.
    The critical terminal curve~$x = 1$~\figurestyle{black dashed}
    is shown for reference.
  }
\end{figure}

\begin{figure}
  \centredfigurecontent[width=\textwidth]{%
    plane-domains%
  }{
    Five domains marked out by a radiation boundary~\figurestyle{black solid}
    and a constant-temperature boundary~\figurestyle{black dotted}.
    The critical terminal curve~$x = 1$~\figurestyle{grey dashed}
    is shown for reference.
  }
\end{figure}

\subsection{Domain construction}
\label{sec:cartesian.plane.domain}

Since the constructed radiation boundaries only dissipate heat,
a domain for the steady conduction--radiation BVP
will not be completely specified
until there is also a boundary to supply it.
The simplest boundary condition which can supply heat
is the Dirichlet condition~$T = \const$,
and given the form of the known solution~%
  (\ref{eq:plane-scaled-laplace-solution}),
these boundaries are simply vertical lines~$x = \const$.

An infinite number of conduction--radiation domains
may therefore be marked out
by joining a constructed radiation boundary
with an appropriate Dirichlet boundary~$x = \const$,
as in Figure~\ref{fig:plane-domains}.
Each of these domains corresponds to steady conduction in the interior,
constant temperature along the right hand side,
and thermal radiation into vacuum to the left.
Most surprising is that \emph{all} of these domains
admit the \emph{same} exact solution~(\ref{eq:plane-scaled-laplace-solution}).

Unfortunately,
the domains produced here by boundary tracing are not convex,
but rather \term{self-viewing}:
some of the outgoing radiation travels not to infinity,
but strikes another part of the boundary,
where it might be partially or fully absorbed.
Since the simple radiation boundary condition~%
  (\ref{eq:plane-scaled-radiation-boundary-condition})
does not account for this,
the results of this section,
while mathematically sound,
are not physically valid.

\section{Cosinusoidal solution}
\label{sec:cartesian.cosine}

While boundary tracing as performed on
the plane-source solution~(\ref{eq:plane-laplace-solution})
did not yield convex conduction--radiation domains,
one might expect more favourable outcomes
by starting from solutions to Laplace's equation
which are not one-dimensional.
A simple class of such solutions consists of functions which are the product
of a trigonometric function in~$x$ and an hyperbolic function in~$y$.

As seen in Section~\ref{sec:cartesian.plane.domain}
one eventually requires a boundary to supply the heat being radiated away,
and from a practical point of view it is
straight-line, constant-temperature boundaries which are of interest.
Therefore, in this section I consider known solutions of the form
\begin{important}{equation}
  T \ideq T_0 \roundbr*{1 - B \cos\frac{x}{L_0} \cosh\frac{y}{L_0}},
  \label{eq:cosine-laplace-solution}
\end{important}
where $T_0$~is a temperature, $L_0$~is a length scale,
and $B$~is a dimensionless constant.
The temperature is constant (taking the value~$T = T_0$)
along the straight line~$x = \pi/2 \cdot L_0$,
which shall ultimately serve as the heat-supplying boundary.

\subsection{Scaling}
\label{sec:cartesian.cosine.scaling}

First the physical parameters are scaled out to the extent possible.
Using the same scalings as Section~\ref{sec:cartesian.plane.scaling},
i.e.~(\ref{eq:plane-scaled-temperature}) through~(\ref{eq:plane-scaled-del}),
the radiation boundary condition~(\ref{eq:radiation-boundary-condition})
and the known solution~(\ref{eq:cosine-laplace-solution})
become
\begin{align}
  \normalvec \dotp \scaleddel \scaled{T}
    &= -\group{c \lambda \tau^3} \scaled{T}^4,
    \label{eq:cosine-scaled-radiation-boundary-condition-with-groups}
    \\[\tallspace]
  \scaled{T}
    &\ideq
      \group{\frac{T_0}{\tau}}
      \roundbr*{
        1 -
          B
          \cos \roundbr*{\group{\frac{\lambda}{L_0}} \scaled{x}}
          \cosh \roundbr*{\group{\frac{\lambda}{L_0}} \scaled{y}}
      },
    \label{eq:cosine-scaled-laplace-solution-with-groups}
\end{align}
where there are two free scales, $\tau$~(temperature) and $\lambda$~(length).
Since there are three unique dimensionless groups,
one of the groups cannot be eliminated,
and to keep the cosinusoidal terms as simple as possible,
the obvious scales
\begin{align}
  \tau &= T_0,
    \label{eq:cosine-temperature-scale} \\
  \lambda &= L_0,
    \label{eq:cosine-length-scale}
\end{align}
are chosen.
Defining the dimensionless group
\begin{equation}
  A \defeq \frac{1}{c L_0 {T_0}^3}.
  \label{eq:cosine-dimensionless-group}
\end{equation}
and \atten{dropping \scalingmarks},
the scaled boundary condition~%
  (\ref{eq:cosine-scaled-radiation-boundary-condition-with-groups})
and known solution~(\ref{eq:cosine-scaled-laplace-solution-with-groups})
become
\begin{important}{align}
  \normalvec \dotp \del T &= -\frac{T^4}{A},
    \label{eq:cosine-scaled-radiation-boundary-condition} \\[\tallspace]
  T &\ideq 1 - B \cos x \cosh y.
    \label{eq:cosine-scaled-laplace-solution}
\end{important}

\begin{figure}
  \centredfigurecontent[width=\textwidth]{%
    cosine-physical%
  }{
    Unphysical region~$T < 0$~\figurestyle{black shaded}
    for the known solution~(\ref{eq:cosine-scaled-laplace-solution}),
    as $B$~increases.
    Also shown are the $T$-contours~\figurestyle{thin grey}.
    In each plot, the left edge is~$x = 0$
    and the vertical line~\figurestyle{thicker grey} is~$x = \pi/2$,
    which is also the contour~$T = 1$.
  }
\end{figure}

\subsection{Physical region}
\label{sec:cartesian.cosine.physical}

Since it is the straight-line boundary~$x = \pi/2$
which ultimately shall, in the form of a Dirichlet condition~$T = 1$,
supply the heat to be conducted and radiated away,
only the side on which~$T \le 1$ is relevant,
namely the side~$x \le \pi/2$.
Given that the known solution~(\ref{eq:cosine-scaled-laplace-solution})
is an even function of~$x$,
it also suffices to consider~$x \ge 0$ only.

Thus the region of interest
is the vertical strip~$0 \le x \le \pi/2$.
Not all of this strip is physical
as only positive temperatures are admissible
in thermal radiation problems.
Since the known solution~(\ref{eq:cosine-scaled-laplace-solution})
vanishes along~$\cos x \cosh y = 1 / B$,
the geometry of the physical region~$T \ge 0$
shall depend on the value of the dimensionless constant~$B$
(see Figure~\ref{fig:cosine-physical}).
The unphysical region~$T < 0$ is omitted from the remainder of the analysis,
as physically meaningful radiation boundaries cannot be found there.

\subsection{Boundary tracing equations}
\label{sec:cartesian.cosine.tracing}

In addition to only considering the physical region~$T \ge 0$,
traced boundaries will only exist within the viable domain~$\Phi \ge 0$.
For the boundary condition~%
  (\ref{eq:cosine-scaled-radiation-boundary-condition})
and known solution~(\ref{eq:cosine-scaled-laplace-solution}) at hand,
one obtains the derivatives
\begin{align}
  P &\ideq \pder{T}{x} \ideq \+B \sin x \cosh y,
    \label{eq:cosine-gradient-u-component} \\[\tallspace]
  Q &\ideq \pder{T}{y} \ideq -B \cos x \sinh y,
    \label{eq:cosine-gradient-v-component}
\end{align}
the flux function
\begin{equation}
  F \ideq -\frac{(1 - B \cos x \cosh y) ^ 4}{A},
  \label{eq:cosine-flux-function}
\end{equation}
and the viability function
\begin{align*}
  \Phi
  &\ideq (\del T)^2 - F^2 \\
  &\ideq
    B^2 \roundbr*{\sin^2 x + \sinh^2 y}
      -
    \frac{(1 - B \cos x \cosh y) ^ 8}{A^2}.
    \yesnumber
    \label{eq:cosine-viability-function}
\end{align*}
The geometry of the viable domain~$\Phi \ge 0$
therefore depends on both of the dimensionless parameters~$A$ and~$B$,
as does the boundary tracing ODE~%
  (\ref{eq:tracing-ode-coordinate-parametrisation-u}),
which becomes
\begin{important}{equation}
  \tder{x}{y} = \frac{P Q \mp F \sqrt{\Phi}}{Q^2 - F^2}.
  \label{eq:cosine-tracing-ode-coordinate-parametrisation-x}
\end{important}
Given the complexity of managing more than one parameter,
I consider~$B = 1$ before examining the general case.

\subsection{Simple case (\texorpdfstring{$B = 1$}{B = 1})}
\label{sec:cartesian.cosine.simple}

\subsubsection{Viable domain}
\label{sec:cartesian.cosine.simple.viable}

In the $B = 1$~case,
the known solution~(\ref{eq:cosine-scaled-laplace-solution}) simplifies to
\begin{equation}
  T \ideq 1 - \cos x \cosh y
  \label{eq:cosine-simple-laplace-solution}
\end{equation}
and the viability function~(\ref{eq:cosine-viability-function}) reduces to
\begin{equation}
  \Phi \ideq \sin^2 x + \sinh^2 y - \frac{(1 - \cos x \cosh y) ^ 8}{A^2}.
  \label{eq:cosine-simple-viability-function}
\end{equation}
The dependence of the non-viable domain~$\Phi < 0$
on the dimensionless group~$A$
is shown in Figure~\ref{fig:cosine_simple-physical-viable}.
Radiation boundaries as produced by boundary tracing
will only be found in the white region,
which is both physical~($T \ge 0$) and viable~($\Phi \ge 0$).

The non-viable domain consists of a single%
\footnote{
  There also exist non-viable regions
  (not shown in Figure~\ref{fig:cosine_simple-physical-viable})
  which lie within the unphysical region~$T < 0$,
  but recall that the unphysical region has been omitted from the analysis
  due to physical irrelevance.
}
non-viable region which recedes towards the right as $A$~increases.
Not obvious from Figure~\ref{fig:cosine_simple-physical-viable} however
is that the boundary of this single non-viable region,
henceforth called the \term{proper} terminal curve,
only constitutes \emph{almost all} of the terminal curve.
The full terminal curve consists of the proper terminal curve
together with the origin~$(x, y) = (0, 0)$,
which is a terminal point
by virtue of being an isolated root of the equation~$\Phi = 0$.

\begin{figure}
  \centredfigurecontent[width=\textwidth]{%
    cosine_simple-physical-viable%
  }{
    Non-viable domain~$\Phi < 0$~\figurestyle{grey shaded}
    for the known solution~(\ref{eq:cosine-simple-laplace-solution})
    and viability function~(\ref{eq:cosine-simple-viability-function}),
    as $A$~increases.
    Also shown are the unphysical region~$T < 0$~\figurestyle{black shaded},
    the $T$-contours~\figurestyle{thin grey},
    and the (proper) terminal curve~$\Phi = 0$~\figurestyle{black dashed}.
    In each plot, the left edge is~$x = 0$
    and the vertical line~\figurestyle{thicker grey} is~$x = \pi/2$,
    which is also the contour~$T = 1$.
  }
\end{figure}

\subsubsection{Boundary tracing}
\label{sec:cartesian.cosine.simple.tracing}

The tracing ODE~%
  (\ref{eq:cosine-tracing-ode-coordinate-parametrisation-x})
cannot be integrated analytically,
so the traced boundaries are determined numerically.
In doing so, the parametrisation~$x = x (y)$
is unable to handle traced boundaries which are locally horizontal;
such difficulty is avoided by instead using the arc-length parametrisation~%
  (\ref{eq:tracing-ode-arc-length-parametrisation-u})
and~(\ref{eq:tracing-ode-arc-length-parametrisation-v}),
which for Cartesian coordinates reduces to
\begin{align}
  \tder{x}{s} &= \frac{-Q F \pm P \sqrt{\Phi}}{(\del T)^2},
    \label{eq:cosine-tracing-ode-arc-length-parametrisation-x} \\[\tallspace]
  \tder{y}{s} &= \frac{\+P F \pm Q \sqrt{\Phi}}{(\del T)^2}.
    \label{eq:cosine-tracing-ode-arc-length-parametrisation-y}
\end{align}
Two branches of traced boundaries are obtained
by integrating forward
from starting points within the region
both physical~($T \ge 0$) and viable~($\Phi \ge 0$).
The resulting curves (Figure~\ref{fig:cosine_simple-traced-boundaries})
are boundaries along which the radiation boundary condition~%
  (\ref{eq:cosine-scaled-radiation-boundary-condition})
is satisfied.
For the correct boundary orientation,
note that the temperature~$T$ is an increasing function of~$x$;
it follows that the interior lies to the right of each boundary curve,
with heat being lost by radiation to the left.

\begin{figure}
  \centredfigurecontent[width=0.5\textwidth]{%
    cosine_simple-traced-boundaries%
  }{
    Traced boundaries~\figurestyle{black} for~$A = 0.5$,
    obtained by integrating
    (\ref{eq:cosine-tracing-ode-arc-length-parametrisation-x})
    and~(\ref{eq:cosine-tracing-ode-arc-length-parametrisation-y}).
    Also shown are the unphysical region~$T < 0$~\figurestyle{black shaded}
    and the non-viable domain~$\Phi < 0$~\figurestyle{grey shaded}.
  }
\end{figure}

By patching together the traced boundaries,
an unlimited number of radiation boundaries may be constructed.
However, almost all such constructions are non-convex
(see Figure~\ref{fig:cosine_simple-traced-boundaries-patched-spiky}).
At any point \emph{strictly} within the viable domain,
i.e.~$\Phi > 0$,
the two traced boundaries through it will cross at a non-zero angle,
and patching these two boundaries together
inevitably results in a spike made of non-convex curves.
As noted already in Section~\ref{sec:cartesian.plane.domain},
self-viewing radiation is not accounted for
by the simple radiation boundary condition~%
  (\ref{eq:cosine-scaled-radiation-boundary-condition}),
and so the non-convex constructions are all invalid.

\begin{figure}
  \newcommand*{\subfigurewidth}{0.36\textwidth}
  \centering
    \hspace*{\fill}
  \begin{subfigure}[t]{\subfigurewidth}
    \centredfigurecontent{cosine_simple-traced-boundaries-patched-spiky}{%
      Spiky non-convex boundary
    }
  \end{subfigure}
    \hfill
  \begin{subfigure}[t]{\subfigurewidth}
    \centredfigurecontent{cosine_simple-traced-boundaries-patched-smooth}{%
      Smooth candidate boundary
    }
  \end{subfigure}
    \hspace*{\fill}
  \caption{
    Radiation boundaries~\figurestyle{black} patched together
    using traced boundaries~\figurestyle{grey}.
    Also shown are the unphysical region~$T < 0$~\figurestyle{black shaded}
    and the non-viable domain~$\Phi < 0$~\figurestyle{grey shaded}.
  }
  \label{fig:cosine_simple-traced-boundaries-patched}
\end{figure}

\begin{figure}
  \centredfigurecontent[width=0.35\textwidth]{%
    cosine_simple-terminal-points%
  }{
    $T$-contours~\figurestyle{thin grey} which intersect
    the proper terminal curve~$\Phi = 0$~\figurestyle{black dashed},
    which is the border of
    the non-viable domain~$\Phi < 0$~\figurestyle{grey shaded}.
    The horizontal scale is exaggerated.
  }
\end{figure}

To avoid the non-convex spikes,
the join must occur at a point along the terminal curve~$\Phi = 0$.
Specifically this point must not be an ordinary terminal point,
at which the two traced boundaries being patched
would form a cusp with inconsistent boundary orientation
(see Section~\ref{sec:introduction.tracing}).
Instead the join must occur at a critical terminal point;
recall that this is a point along the terminal curve
for which the local $T$-contour touches the terminal curve~$\Phi = 0$
tangentially (rather than crossing at a non-zero angle).

Now in the present situation,
the full terminal curve consists of
an isolated terminal point at the origin
in union with the proper terminal curve
(which is the boundary of the non-viable region
visible in Figure~\ref{fig:cosine_simple-physical-viable}).
The terminal point at the origin is already notable due to its isolation,
but even more remarkable is that the local $T$-contour~($T = 0$)
looks like~$y = \pm x$.
One therefore wonders:
  are the local $T$-contour (whose tangent is indeterminate)
  and the local portion of the terminal curve (which is just a point)
  \emph{tangential}, and therefore,
  is the origin a \emph{critical} terminal point?
While the tangentiality is debatable,
the answer to the criticality question ought to be yes
on account of the actual behaviour:
two traced boundaries pass through the origin
(see Figure~\ref{fig:cosine_simple-traced-boundaries})
as if it were an hyperbolic critical terminal point,
with $y = 0$~as the common tangent.
Unfortunately the pair of traced boundaries forms a non-convex spike,
which is of no use here.

With the origin ruled out,
it remains to consider the proper terminal curve.
Figure~\ref{fig:cosine_simple-terminal-points} shows that in the present case,
all points along the proper terminal curve are ordinary
except for a single critical terminal point~$(x, y) = (x_0, 0)$
at the intersection of the proper terminal curve
and the horizontal axis~$y = 0$.
Algebraically, $x_0$~is the positive%
\footnote{
  The trivial solution~$x = 0$
  corresponds to the isolated terminal point at the origin,
  which has already been dismissed.
}
solution to
\begin{equation}
  \eval[\big]{\Phi}_{y=0}
  \ideq \roundbr*{1 - \cos^2 x} - \frac{(1 - \cos x)^8}{A^2}
  = 0,
  \label{eq:cosine-simple-critical-terminal-point-x}
\end{equation}
a polynomial equation in~$\cos x$.
Furthermore, the local $T$-contour through~$(x_0, 0)$
lies on the viable side of the proper terminal curve;
therefore the critical terminal point~$(x_0, 0)$ is of hyperbolic type,
with two traced boundaries passing through it smoothly.
By patching together the portions which lie to the right of~$(x_0, 0)$,
a single spike-free radiation boundary can be constructed
as shown in Figure~\ref{fig:cosine_simple-traced-boundaries-patched-smooth};
this boundary shall be referred to as the \term{candidate boundary}.
It is interesting to note that
the candidate boundary and the proper terminal curve
are virtually indistinguishable;
despite only touching each other at the critical terminal point~$(x_0, 0)$,
the former asymptotically approaches the latter travelling away from~$y = 0$.

\subsubsection{Convexity}
\label{sec:cartesian.cosine.simple.convexity}

The choice of the candidate boundary over all other boundary patchings
is a necessary but not sufficient condition for convexity.
While the candidate boundary is convex at~$(x_0, 0)$,
it eventually inflects
at some~$(x, y) = (x_\infl, \pm y_\infl)$ depending on~$A$,
and becomes concave.

Recalling the goal of constructing domains
corresponding to a conduction--radiation BVP\@,
a \term{candidate domain} is demarcated by using
the candidate boundary as the radiation boundary
and the straight line~$x = \pi/2$
as a constant-temperature heat-supplying boundary.
The continuum of candidate domains (for~$0 < A < 1$)\footnote{
  At~$A = 1$ the candidate boundary touches the line~$x = \pi/2$,
  and the candidate domain shrinks to a point.
}
is shown in Figure~\ref{fig:cosine_simple-candidate-domains};
each domain is shaped like a thin lens,
corresponding to steady conduction in its interior,
thermal radiation along the curved boundary to the left,
and constant temperature~$T = 1$ along the straight boundary on the right.
The domain will be valid if the curved boundary
(a portion of the corresponding candidate boundary)
is convex, or equivalently,
if the first inflection of the corresponding candidate boundary
occurs at abscissa~$x_\infl \ge \pi/2$.

\begin{figure}
  \centredfigurecontent[width=0.65\textwidth]{%
    cosine_simple-candidate-domains%
  }{
    Candidate domains marked out
    by a radiation boundary~\figurestyle{black solid}
    and a constant-temperature boundary~\figurestyle{black dotted},
    as $A$~increases.
    Points of inflection are shown for reference.
  }
\end{figure}

The critical value~$A = A_\infl$
(where the candidate domain changes from invalid to valid)
is that for which the first inflection occurs at precisely~$x_\infl = \pi/2$.
Given that the candidate boundary is never horizontal,
$A_\infl$~is best sought using the coordinate parametrisation~$x = x (y)$
for the candidate boundary.
In Cartesian coordinates, points of inflection coincide with
zero-crossings in the second derivative,
or, using primes for $y$-differentiation, $x''$.
Differentiating the first derivative~%
  (\ref{eq:cosine-tracing-ode-coordinate-parametrisation-x})
gives
\begin{equation}
  x'' = \tder{}{y} \roundbr*{\frac{P Q \mp F \sqrt{\Phi}}{Q^2 - F^2}},
  \label{eq:acceleration-traced-boundary-cartesian-by-y}
\end{equation}
whose right hand side may be reduced to a function purely of~$x$ and~$y$
by applying~(\ref{eq:cosine-tracing-ode-coordinate-parametrisation-x})
once more to eliminate the first derivatives.
Substituting~$x = x_\infl = \pi/2$, this becomes
\begin{equation}
  \eval[\big]{x''}_{x=\pi/2} =
    \frac{A^2 C}{\sqrt{A^2 C^2 - 1}}
    \squarebr*{
      2 S - (1 + S^2) \roundbr*{A^2 S + 4 \sqrt{A^2 (1 + S^2) - 1}}
    },
  \label{eq:%
    cosine-simple-traced-boundary-acceleration-cartesian-by-y-inflection%
  }
\end{equation}
where~$C \defeq \cosh y$ and~$S \defeq \sinh y$.
Only the square-bracketed factor can change sign,
and an analysis shows that for~$0 < A < 1$,
this factor has a unique zero-crossing~$S = S_\infl (A)$.
The would-be $y$-coordinate of inflection for a given~$A$ is therefore
\begin{equation}
  y = y_\infl (A) = \sinh^{-1} (S_\infl (A)).
  \label{eq:cosine-simple-traced-boundary-y-would-be-inflection}
\end{equation}
Since the inflection is supposed to occur at~$x_\infl = \pi/2$,
the critical value~$A = A_\infl$ is the solution to
\begin{equation}
  x (y_\infl (A)) = \pi/2,
  \label{eq:cosine-simple-traced-boundary-a-inflection-equation}
\end{equation}
whose left hand side is the coordinate parametrisation~$x = x (y)$
evaluated at the would-be $y$-coordinate~%
  (\ref{eq:cosine-simple-traced-boundary-y-would-be-inflection}).
The bisection algorithm yields
\begin{equation}
  A_\infl = 0.79718,
  \label{eq:cosine-simple-traced-boundary-a-inflection}
\end{equation}
and therefore the lens-like candidate domain for a given~$A$ is convex
if and only if
\begin{equation}
  A_\infl \le A < 1.
  \label{eq:cosine-simple-traced-boundary-convex-a-interval}
\end{equation}

% TODO: rewrite from here onward using "proper" terminal curve terminology

\subsection{General case (\texorpdfstring{$B$~arbitrary}{B arbitrary})}
\label{sec:cartesian.cosine.general}

\subsubsection{Viable domain}
\label{sec:cartesian.cosine.general.viable}

In the general case, the dimensionless group~$B$ may also be varied,
and altogether the parameter space~$(A, B)$ is two-dimensional.
Two degrees of freedom is too many to manage,
and the behaviour of the viable domain is best understood
by fixing~$A$ and varying~$B$
(which is the slope~$\pd T / \pd x$ evaluated at~$(x, y) = (\pi/2, 0)$).

Of particular interest are the terminal points along the (positive) $x$-axis
for they locate the critical terminal points.%
\footnote{
  At a terminal point along the $x$-axis,
  the local $T$-contour and the terminal curve~$\Phi = 0$
  will be parallel (unless either has a corner)
  since $T$ and~$\Phi$ are both even functions of~$y$.
}
Algebraically they are given by the roots of
\begin{equation}
  \eval[\big]{\Phi}_{y=0}
  \ideq B^2 \sin^2 x - \frac{(1 - B \cos x)^8}{A^2}
  = 0,
  \label{eq:cosine-general-critical-terminal-point-x}
\end{equation}
which are shown in Figure~\ref{fig:cosine_general-critical}.
For sufficiently small~$B$,
there are no roots and the entire $x$-axis is non-viable,
until a single root~$x = x_\nat (A)$ appears at~$B = B_\nat (A)$.
As $B$~increases past this transitional value,
the root splits into two roots, $x_\flat$ and~$x_\sharp$
(both dependent on both~$A$ and $B$).
At~$B = 1$, the smaller root~$x_\flat$ vanishes,
before becoming positive again for~$B > 1$.
In terms of the viable domain, this translates to five cases
(Figure~\ref{fig:cosine_general-physical-viable}):
\begin{enumerate}
  \item
    \term{Gentle regime}, $B < B_\nat (A)$:
    The non-viable domain completely envelopes the $x$-axis,
    dividing the viable domain into two halves.
    There are no critical terminal points.
  \item
    \term{Gentle-to-fair transition}, $B = B_\nat (A)$:
    The two viable regions pinch together at~$(x, y) = (x_\nat, 0)$,
    a critical terminal point of hyperbolic type
    (with $x_\nat$~dependent on~$A$).
    The non-viable domain now consists of two disjoint regions.
  \item
    \term{Fair regime}, $B_\nat (A) < B < 1$:
    The two non-viable regions are sundered further apart
    by a growing viable tract along the $x$-axis,
    with $(x_\nat, 0)$~splitting into the two
    hyperbolic critical terminal points~$(x_\flat, 0)$ and~$(x_\sharp, 0)$.
  \item
    \term{Fair-to-steep transition}, $B = 1$:
    This is the simple case
    of Section~\ref{sec:cartesian.cosine.general.viable};
    the hyperbolic critical terminal point~$(x_\sharp, 0)$
    is what was formerly called~$(x_0, 0)$.
    The other terminal point~$(x_\flat, 0)$ has receded to the origin,
    which is a single-point terminal ``curve''
    (a peculiarity caused by the smaller non-viable region
    having shrunk to nothingness).
  \item
    \term{Steep regime}, $B > 1$:
    The smaller non-viable region
    (which shrunk to nothingness at the fair-to-steep transition)
    reappears but is overtaken by the growing unphysical region.
    The terminal point~$(x_\flat, 0)$ thus lies in the unphysical region
    and is therefore irrelevant,
    leaving a single hyperbolic critical terminal point~$(x_\sharp, 0)$.
\end{enumerate}

\begin{figure}
  \centredfigurecontent[width=0.7\textwidth]{%
    cosine_general-critical%
  }{
    Terminal points along the $x$-axis, as $B$~increases with $A$~fixed.
  }
\end{figure}

\begin{figure}
  \centredfigurecontent[width=\textwidth]{%
    cosine_general-physical-viable%
  }{
    Non-viable domain~$\Phi < 0$~\figurestyle{grey shaded}
    for the known solution~(\ref{eq:cosine-scaled-laplace-solution})
    and viability function~(\ref{eq:cosine-viability-function}),
    as $B$~increases with $A$~fixed.
    Also shown are the unphysical region~$T < 0$~\figurestyle{black shaded}
    and the terminal curve~$\Phi = 0$~\figurestyle{black dashed}.
    In each plot, the left edge is~$x = 0$ and the right edge is~$x = 2$.
  }
\end{figure}

\chapter{Polar coordinates}
\label{ch:polar}

In this chapter, I apply boundary tracing to the thermal radiation problem,
using the fundamental line-source solution
as the known solution to
Laplace's equation~(\ref{eq:laplace-steady-conduction}).

\section{Line-source solution}
\label{sec:polar.line}

Consider the line-source solution
to Laplace's equation~(\ref{eq:laplace-steady-conduction}),
which is logarithmic in the cylindrical radius~$r = \sqrt{x^2 + y^2}$.
In the context of steady conduction,
the only dimensionally consistent form for this is
\begin{important}{equation}
  T = T_0 \log \roundbr*{\frac{r_0}{r}}
  \label{eq:line-laplace-solution}
\end{important}
for some temperature~$T_0$ and radius~$r_0$.%
\footnote{
  Whereas the power law~$T = C r^n$ is scale-invariant,
  requiring only \emph{one} constant~$C$,
  the logarithm does not possess this property;
  separate constants are required for~$T$ and~$r$.
}
Note that the region~$r > r_0$ is unphysical,
since~$T < 0$ therein.

The analysis to follow shall be performed
in the usual polar coordinates~$(r, \phi)$
(see Figure~\tbd),
given by the transformation
\begin{align}
  x &= r \cos\phi, \label{eq:polar-x-transformation} \\
  y &= r \sin\phi. \label{eq:polar-y-transformation}
\end{align}
The local basis vectors are
\begin{align}
  \localvec{r} &=
    \cos\phi \basisvec{x} + \sin\phi \basisvec{y},
    \label{eq:r-local-basis-vector} \\
  \localvec{\phi} &=
    r \roundbr[\big]{-\sin\phi \basisvec{x} + \cos\phi \basisvec{y}},
    \label{eq:phi-local-basis-vector}
\end{align}
the scale factors,
\begin{align}
  \scalefac[r] &= 1, \label{eq:r-scale-factor} \\
  \scalefac[\phi] &= r, \label{eq:phi-scale-factor}
\end{align}
and the local orthonormal basis vectors,
\begin{align}
  \basisvec{r} &= \cos\phi \basisvec{x} + \sin\phi \basisvec{y},
    \label{eq:r-basis-vector} \\
  \basisvec{\phi} &= -\sin\phi \basisvec{x} + \cos\phi \basisvec{y}.
    \label{eq:phi-basis-vector}
\end{align}

\section{Scaling}
\label{sec:polar.scaling}

While the known solution~(\ref{eq:line-laplace-solution})
already has intrinsic temperature and length scales $T_0$ and~$r_0$,
the final scaling should also account for
the radiation boundary condition~(\ref{eq:radiation-boundary-condition}).

Let
\begin{align}
  \scaled{T} &= T / \tau, \label{eq:line-scaled-temperature} \\
  \scaled{r} &= r / \varrho, \label{eq:line-scaled-r}
\end{align}
with~$\tau$ and~$\varrho$ to be chosen later.
Noting that
\begin{equation}
  \scaleddel = \varrho \del,
  \label{eq:line-scaled-del}
\end{equation}
the radiation boundary condition~(\ref{eq:radiation-boundary-condition})
and the known solution~(\ref{eq:line-laplace-solution})
become
\begin{align}
  \normalvec \dotp \scaleddel \scaled{T}
    &= -\squarebr*{c \varrho \tau^3} \scaled{T}^4,
    \label{eq:line-scaled-radiation-boundary-condition-with-groups}
    \\[\fraclinespace]
  \scaled{T}
    &=
      \squarebr*{\frac{T_0}{\tau}}
      \log \roundbr*{\frac{\squarebr{r_0 / \varrho}}{\scaled{r}}}.
    \label{eq:line-scaled-laplace-solution-with-groups}
\end{align}
There are three dimensionless groups
but only two free scales~$\tau$ and~$\varrho$,
so one of the groups cannot be eliminated.
To keep the logarithmic as simple as possible,
put
\begin{align}
  \tau &= T_0,
    \label{eq:line-temperature-scale} \\
  \varrho &= r_0,
    \label{eq:line-length-scale}
\end{align}
and define the dimensionless group
\begin{equation}
  A = \frac{1}{c r_0 {T_0}^3}.
  \label{eq:line-dimensionless-group}
\end{equation}
Dropping \scalingaccents, the scaled boundary condition~%
  (\ref{eq:line-scaled-radiation-boundary-condition-with-groups})
and known solution~(\ref{eq:line-scaled-laplace-solution-with-groups})
become
\begin{important}{align}
  \normalvec \dotp \del T &= -\frac{T^4}{A},
    \label{eq:line-scaled-radiation-boundary-condition} \\[\fraclinespace]
  T &= -\log r.
    \label{eq:line-scaled-laplace-solution}
\end{important}

\section{Viable domain}
\label{sec:polar.viable}

Comparing the radiation condition~%
  (\ref{eq:line-scaled-radiation-boundary-condition})
to the generic flux condition~(\ref{eq:flux-boundary-condition}),
it follows that the flux function is
\begin{align*}
  F
  &= -\frac{T^4}{A} \\[\fraclinespace]
  &= -\frac{\log^4 r}{A}.
    \yesnumber
    \label{eq:line-flux-function}
\end{align*}
The radial and azimuthal components of the gradient~$\del T$ are
\begin{align}
  P &= \pder{T}{r} = -\frac{1}{r},
    \label{eq:line-gradient-u-component} \\[\fraclinespace]
  Q &= \frac{\pd T}{r \pd\phi} = 0,
    \label{eq:line-gradient-v-component}
\end{align}
and so the viability function is given by
\begin{align*}
  \Phi
  &= (\del T)^2 - F^2 \\[\fraclinespace]
  &= \frac{1}{r^2} - \frac{\log^8 r}{A^2} \\[\fraclinespace]
  &= \frac{A^2 - r^2 \log^8 r}{A^2 r^2} \\[\fraclinespace]
  &= \frac{A^2 - \psi^2}{A^2 r^2},
    \yesnumber
    \label{eq:line-viability-function}
\end{align*}
where $\psi$~is the auxiliary function
\begin{equation}
  \psi (r) \defeq r \log^4 r.
  \label{eq:line-auxiliary-function}
\end{equation}
The viable domain is therefore the region~$\psi (r) \le A$.

\subsection{Auxiliary function properties}
\label{sec:polar.viable.psi}

Since the region~$r > 1$ is unphysical
(corresponding to negative temperature),
it is only necessary to consider~$0 \le r \le 1$.
On this interval, $\psi$~is positive
except at the endpoints~$r = 0$ and~$r = 1$, where it vanishes.
Observing that the slope
\begin{equation}
  \tder{\psi}{r} = (4 + \log r) \log^3 r
  \label{eq:line-psi-derivative}
\end{equation}
changes sign from positive to negative
as $r$~increases through~$\ee^{-4}$,
it follows that $\psi$~has a single maximum on~$0 \le r \le 1$ at
\begin{equation}
  r = r_\nat \defeq \ee^{-4} = 0.01832,
  \label{eq:line-r-natural}
\end{equation}
where it takes the maximal value
\begin{equation}
  \psi
  = A_\nat
  \defeq \psi (r_\nat)
  = (4 / \ee)^4
  = 4.6888.
  \label{eq:line-a-natural}
\end{equation}
This is shown in Figure~\tbd.

\subsection{Three regimes}
\label{sec:polar.viable.regimes}

The topology of the viable domain~$\psi (r) \le A$ will depend
on whether the dimensionless group~$A$ is
greater than, equal to or less than
the critical value~(\ref{eq:line-a-natural}),
as this decides the number of roots of the equation~$\psi (r) = A$:
\begin{enumerate}
  \item
    \emph{Cold regime}, $A > A_\nat$ (Figure~\tbd):
    $\psi (r)$~is never equal to~$A$.
    The entire space~$0 \le r \le 1$ is viable,
    and there is no terminal curve.
  \item
    \emph{Transition}, $A = A_\nat$ (Figure~\tbd):
    $\psi (r)$~is equal to~$A$ at~$r = r_\nat$ only.
    The entire space is still viable,
    but now the terminal curve~$r = r_\nat$ has appeared.
    The terminal curve~$r = r_\nat$ is in fact a critical terminal curve,
    since it is a contour of the known solution~%
      (\ref{eq:line-scaled-laplace-solution}).
  \item
    \emph{Hot regime}, $A < A_\nat$ (Figure~\tbd):
    $\psi (r)$~is equal to~$A$ at~$r = r_\flat$
    and at~$r = r_\sharp$,
    both dependent on~$A$,
    with~$0 <  r_\flat < r_\nat < r_\sharp < 1$.
    The critical terminal curve now consists of
    the two circles~$r = r_\flat$ and~$r = r_\sharp$.
    A non-viable moat~$r_\flat < r < r_\sharp$
    separates an inner viable island~$0 \le r \le r_\flat$
    from an outer viable mainland~$r \ge r_\sharp$.
\end{enumerate}
Figure~\tbd{}
shows the regimes displayed in the style of a bifurcation diagram,
with state space on the vertical axis
and the parameter~$A$ on the horizontal axis.

\section{Boundary tracing}
\label{sec:polar.tracing}

Using~(\ref{eq:r-scale-factor}) and~(\ref{eq:phi-scale-factor})
(which are unchanged by the scaling)
along with~(\ref{eq:line-flux-function})
through~(\ref{eq:line-viability-function}),
the boundary tracing ODE~(\ref{eq:tracing-ode-coordinate-parametrisation-u})
becomes
\begin{important}{equation}
  \tder{r}{\phi} = \mp \frac{\sqrt{A^2 - \psi^2}}{\log^4 r},
  \label{eq:line-tracing-ode-coordinate-parametrisation-r}
\end{important}
so that the traced boundaries are given by
\begin{important}{equation}
  \phi = \mp \int \frac{\log^4 r \td r}{\sqrt{A^2 - \psi^2}}.
  \label{eq:line-traced-boundary-integral}
\end{important}
This integral is not elementary,
and the traced boundaries must be determined numerically.

At this point one should keep in mind the end goal,
which is to construct interesting domains
for which the radiation condition~%
  (\ref{eq:line-scaled-radiation-boundary-condition})
is satisfied along the boundary,
given steady conduction of the form~(\ref{eq:line-scaled-laplace-solution}).
Since the nature of the line-source singularity at~$r = 0$
is to supply heat radially at a constant power per length,
the only sensible configuration is for the domain to
completely surround the singularity~$r = 0$ without touching it,
with the power supplied by the singularity balanced
by that lost along the outer boundary by radiation.
Therefore one seeks closed curves,
which surround the origin,
and which, are made by patching together the traced boundaries~%
  (\ref{eq:line-traced-boundary-integral}).

Moreover, it was already seen in Section~\ref{sec:cartesian.slab.tracing}
that only convex domains are acceptable,
lest there be self-incident radiation which is not accounted for
by the simple radiation condition~%
  (\ref{eq:line-scaled-radiation-boundary-condition}).
The patching together of traced boundaries to form closed curves
must therefore be done carefully.

\subsection{Cold regime}
\label{sec:polar.tracing.cold}

Since the sought-after closed curve must surround the origin
and mark out a convex domain,
it may be parametrised in the form~$r = r (\phi)$,
a single-valued function.
Travelling once around the closed curve,
the azimuthal angle~$\phi$ will run through a full turn,
so $r (\phi)$~must also be $2 \pi$-periodic.

Now in the cold regime~$A > A_\nat$,
recall that $\psi (r)$~is always strictly less than~$A$.
It follows that for the two branches of traced boundaries,
corresponding to the two choices of sign in the boundary tracing ODE~%
  (\ref{eq:line-tracing-ode-coordinate-parametrisation-r}),
the upper branch always has $\td r / \td\phi$~negative,
and the lower branch always has $\td r / \td\phi$~positive.%
\footnote{
  For both of the symbols~$\pm$ and~$\mp$,
  the terms \emph{upper} and \emph{lower}
  shall always refer to the choice of sign with that corresponding position.
}
For the sought-after closed curve,
this means that $r (\phi)$~will be strictly decreasing
along the portions which originate from the upper branch,
and strictly increasing along the portions from the lower branch.

Since $r (\phi)$~must be periodic,
the closed curve must therefore consist of both
upper-branch and lower-branch portions.
Travelling in the direction of increasing~$\phi$,
there must eventually be a switching
from the upper branch to the lower branch,
and hence a jump discontinuity in~$\td r / \td\phi$
from negative to positive.
But this corresponds to a non-convex corner
(Figure~\tbd),
so no convex domains can be constructed from traced boundaries
in the cold regime.

\subsection{Transition}
\label{sec:polar.tracing.transition}

For the transition case~$A = A_\nat$,
the two branches of the boundary tracing ODE~%
  (\ref{eq:line-tracing-ode-coordinate-parametrisation-r})
are again segregated by the sign of~$\td r / \td\phi$,
with the exception of the critical terminal curve~$r = r_\nat$,
itself a traced boundary belonging to both branches,
along which~$\psi (r) = A$ and~$\td r / \td\phi = 0$.

Any other closed curve made from traced boundaries
will again consist of portions where $r (\phi)$~is strictly increasing
and other portions where it is strictly decreasing.
To avoid the non-convex corner of Section~\ref{sec:polar.tracing.cold},
any switching from the upper branch to the lower
must occur along~$r = r_\nat$,
so that $\td r / \td\phi$~remains zero across the switch
(thus avoiding the jump from negative to positive).

\chapter{Bipolar coordinates}
\label{ch:bipolar}

In this chapter,
I apply boundary tracing for the conduction--radiation problem
in bipolar coordinates,
taking the known solution
to Laplace's equation~(\ref{eq:laplace-steady-conduction})
which corresponds to equal and opposite line sources.


\backmatter

\bibliographystyle{abbrv-conway}
\bibliography{thesis-bibliography}

\end{document}
