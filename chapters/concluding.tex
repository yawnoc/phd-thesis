\chapter{Concluding remarks}
\label{ch:concluding}

In this thesis,
we have applied the theory of boundary tracing
in the physically significant contexts
of thermal radiation and capillarity.
After deriving general versions of the boundary tracing ODE
(for both coordinate parametrisation
and the numerically robust arc-length parametrisation)
applicable in any orthogonal coordinate system,
we have thoroughly analysed
the traced boundaries that can be produced in various scenarios:

\section{Thermal radiation}
\label{sec:concluding.radiation}

In the first part of this thesis we have examined the use of boundary tracing
in a simple conduction--radiation problem,
consisting of Laplace's equation in the interior
and thermal radiation along the boundary
in accordance with the Stefan--Boltzmann law.
While exact solutions to Laplace's equation are plenty,
the quartic dependence of the boundary flux on the temperature
is difficult to handle analytically using conventional approaches;
thus boundary tracing is a well-suited approach to tackling this problem.

For each of several analytical solutions to Laplace's equation,
we have constructed a vast array of new domains
which admit exactly that solution
to the conduction--radiation problem.
These results have been solidly backed
by numerical verification using finite elements.
Although certain constructed domains are inadmissible
due to non-convexity,
we have demonstrated how to salvage useful results from among them
by identifying cases
where the amount of self-viewing radiation is negligibly small.
In determining the physically realisable lengths and temperatures
in various instances,
we have shown that the results of boundary tracing
are indeed applicable in practical contexts,
and not just mathematical curiosities of theoretical interest.

\thematicbreak

The methods employed here
generalise in a straightforward manner
to any problem
that couples Laplace's equation with a flux boundary condition.
Corrosion modelling~\cite{
  bryan-2002-singular-nonlinear-elliptic-corrosion,
  vogelius-1998-nonlinear-elliptic-bvp-corrosion
}
is a suitable candidate for future exploration,
in addition to convective heat transfer
in the context of steady conduction.
For such investigations involving Laplace's equation,
we note that it is possible to recast the traced boundaries
as solutions to a differential equation in the complex plane;
details are given in Appendix~\ref{ch:complex}\@.
Given the intimate connection between
harmonic functions and the powerful method of conformal mapping,
this idea may prove useful in future endeavours.

The requirement of convexity was a significant impediment
in the search for radiation boundaries.
Since Laplace's equation is linear,
we imagine that a suitable perturbation of the known solution
might nudge a slightly non-convex boundary
into a convex one.
Such a technique could extend the boundary tracing results here;
more generally it could be used to investigate
subtle changes of boundary shape
for any BVP where the underlying PDE is linear.

\section{Capillarity}
\label{sec:concluding.capillarity}

In the second part of this thesis,
we have used numerical boundary tracing
to analyse corner rounding in capillary wedges.
Numerical wedge solutions have been computed in all wedge regimes
using the finite element method,
with special treatment in the small wedge case
to handle the singularity in the corner.
While it seems pointless to produce traced boundaries
from a known solution that has been computed numerically,
we have observed crucially that
a different contact angle can be used for boundary tracing
to that of the known solution.
In convex wedges,
this novel observation has enabled us to generate
a one-parameter family of rounding candidates
from a single numerical wedge solution,
and although smooth rounding curves
are not possible in re-entrant wedges,
we have likewise produced a continuum of pseudo-roundings.
The solution along any new boundary
may be obtained by simply evaluating the known solution:
this is the reward of boundary tracing.

In pursuit of a perfectly flat dip-coating profile
for a re-entrant corner,
we have revisited a previous analysis
by Anderson~\cite{anderson-2002-thesis-boundary-tracing-pdes}
and recognised its congruence with the roughness theory
of Wenzel~\cite{wenzel-1936-resistance-solid-surfaces-wetting}.
Applying this to the dip-coating problem,
we have shown that
a near-level dip-coating profile can be obtained
by rounding a re-entrant corner in the shape of a contour
and then applying a position-dependent amount of roughness.
The latter can be achieved in a very practical manner
by etching grooves of regular shape at suitable variable spacing.

As to the general modification of a convex wedge,
we have built on a comparison observation
of Anderson~\cite{anderson-2002-thesis-boundary-tracing-pdes}
by giving a new physical interpretation
which fits perfectly with the earlier roughness theory.
The improved understanding has allowed us to obtain a useful corollary,
regarding truncation of a convex capillary wedge
and the associated effect on the height rise.

\thematicbreak

We have seen the utility of numerical boundary tracing
in the analysis of capillary wedges.
While our observation regarding different contact angles
arose in the context of capillarity,
the same idea is applicable
to any flux boundary condition with a mutable parameter.

Finally, boundary tracing may be used
as a supplement to conventional numerical approaches,
rather than as a replacement.
Any numerical solution to a BVP will have a sunk cost
by virtue of being computed.
Boundary tracing is a means of obtaining \emph{additional} results cheaply,
by simply solving some ODEs.
