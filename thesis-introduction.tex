\chapter{Introduction and Literature Review}

The usual approach
to investigating the effect of domain shape
on solutions to a boundary value problem (BVP)
is simply to solve the BVP in various domains.
For any given domain shape, one seeks solutions
to the associated partial differential equation (PDE) in the interior
which also satisfy the prescribed boundary conditions on the boundary.
In linear problems possessing sufficient symmetry,
the superposition principle
together with separation of variables or integral transforms
typically allows one to complete this task,
but in nonlinear problems,
exact solutions are rare and often only available for simple geometries.
From a numerical viewpoint
it can also be computationally expensive
to consider many different domain shapes,
since the BVP must be solved from scratch for each new domain.

An alternative strategy is to
consider the boundary value problem in reverse:
given a solution to the PDE, are there new boundaries
which also satisfy the prescribed boundary conditions?
If so, it is conceivable that these new boundaries may be used
to construct new domains
which admit the same solution to the given boundary value problem.
Such an approach has been employed in an \adhoc{} fashion
by Anderssen~\etal~\cite{anderssen-1969-ion-uptake-growing-roots}
to numerically determine the profile of a growing root
and by McNabb~\etal~\cite{mcnabb-1991-theoretical-derivation-freezing-times}
to estimate finite measures of cooling time in ellipsoids.
The first systematic investigation of this
reverse strategy for flux boundary conditions,
known as \emph{boundary tracing},
was carried out by
Anderson~\etal~\cite{anderson-2007-boundary-tracing-i-theory}
with applications to the Laplace--Young equation
of capillarity~\cite{anderson-2006-exact-solutions-laplace-young}
and other PDEs, including the Helmholtz,
constant mean curvature and
Poisson equations~\cite{anderson-2007-boundary-tracing-ii-applications}.

Over a decade has since passed
with little further work in the area,
and the aim of this thesis is to
continue the application of boundary tracing
to physically significant contexts.
I explore its use in thermal radiation problems
and extend the work on capillarity.
I begin by giving a brief introduction to boundary tracing
and reviewing the relevant known results
in thermal radiation and capillarity.

\section{Boundary tracing}

This section provides a quick overview of
the theory of \emph{boundary tracing},
which is well-described by
Anderson~\etal~\cite{anderson-2007-boundary-tracing-i-theory}
with further details given in Anderson's thesis~\cite{}.
Consider a BVP, consisting of
a PDE in some two-dimensional domain~$\Omega$
along with the boundary condition
\begin{equation}
  \normalvec \dotp \del T = F (x, y, T, \norm{\del T})
  \label{eq:flux-boundary-condition}
\end{equation}
on its boundary~$\pd\Omega$,
where $F$~is a given \emph{flux function}.
Whereas the usual goal is
to determine the solution~$T$ for a given domain~$\Omega$,
the aim of boundary tracing is
to seek~$\Omega$ for a given~$T$.
Specifically, one takes any known solution~$T = T (x, y)$ to the PDE
and seeks \emph{traced boundaries}, which are curves
along which the flux condition~(\ref{eq:flux-boundary-condition}) holds.
Once the traced boundaries have been found,
they may be used to construct new domains
for which the solution to the BVP is also~$T$.
Surprisingly, this reverse problem
of determining boundaries from a known solution
often has infinitely many answers,
leading to far more new and interesting domain shapes
than expected~\cite{anderson-2007-boundary-tracing-ii-applications}.
