\chapter{Introduction and Literature Review}
\label{ch:introduction}

The usual approach
to investigating the effect of domain shape
on solutions to a boundary value problem (BVP)
is simply to solve the BVP for various domains.
For any given domain shape, one seeks solutions
to the associated partial differential equation (PDE) in the interior
which also satisfy the prescribed boundary conditions on the boundary.
In linear problems possessing sufficient symmetry,
the superposition principle
together with separation of variables or integral transforms
typically allows one to complete this task,
but in nonlinear problems,
exact solutions are rare and often only available for simple geometries.
From a numerical viewpoint
it can also be computationally expensive
to consider many different domain shapes,
since the BVP must be solved from scratch for each new domain.

An alternative strategy is to
consider the BVP in reverse:
given a solution to the PDE\@, are there new boundaries
which also satisfy the prescribed boundary conditions?
If so, it is conceivable that these new boundaries may be used
to construct new domains
which admit the same solution to the given BVP\@.
Such an approach has been employed in an \adhoc{} fashion
by Anderssen~\etal~\cite{anderssen-1969-ion-uptake-growing-roots}
to numerically determine the profile of a growing root
and by McNabb~\etal~\cite{mcnabb-1991-theoretical-derivation-freezing-times}
to estimate finite measures of cooling time in ellipsoids.
The first systematic investigation of this
reverse strategy for flux boundary conditions,
known as \emph{boundary tracing},
was carried out by
Anderson~\etal~\cite{anderson-2007-boundary-tracing-i-theory}
with applications to the \laplaceyoung{} equation
of capillarity~\cite{anderson-2006-exact-solutions-laplace-young}
and other PDEs, including the Helmholtz,
constant mean curvature and
Poisson equations~\cite{anderson-2007-boundary-tracing-ii-applications}.

Over a decade has since passed
with little further work in the area,
and the aim of this thesis is to
continue the application of boundary tracing
to physically significant contexts.
I explore its use in thermal radiation problems
and extend the work on capillarity.
I begin by giving a brief introduction to boundary tracing
and reviewing the relevant known results
in thermal radiation and capillarity.

\section{Boundary tracing}
\label{sec:introduction.tracing}

In this section, I provide a quick overview of
the theory of \emph{boundary tracing},
which is well-described by
Anderson~\etal~\cite{anderson-2007-boundary-tracing-i-theory}
with further details given in
Anderson's thesis~\cite{anderson-2002-thesis-boundary-tracing-pdes}.
Consider a BVP\@, consisting of
a PDE in some two-dimensional domain~$\Omega$
along with the boundary condition
\begin{important}{equation}
  \normalvec \dotp \del T = F \roundbr[\big]{x, y, T, \norm{\del T}}
  \label{eq:flux-boundary-condition}
\end{important}
on its boundary~$\pd\Omega$,
where $\normalvec$~is the outward-pointing unit normal
and $F$~is a given \emph{flux function}.
Whereas the usual goal is
to determine the solution~$T$ for a given domain~$\Omega$,
the aim of boundary tracing is
to seek~$\Omega$ for a given~$T$.
Specifically, one takes any known solution~$T = T (x, y)$ to the PDE
and seeks \emph{traced boundaries}, which are curves
along which the flux condition~(\ref{eq:flux-boundary-condition}) holds.
Once the traced boundaries have been found,
they may be used to construct new domains
for which the solution to the BVP is also~$T$.
Surprisingly, this reverse problem
of determining boundaries from a known solution
often has infinitely many answers,
leading to far more new and interesting domain shapes
than expected~\cite{anderson-2007-boundary-tracing-ii-applications}.

The flux condition~(\ref{eq:flux-boundary-condition})
is best understood as a geometric constraint:
at any point~$(x, y)$ in the domain of~$T$,
it determines the possible local orientations for~$\normalvec$,
and therefore, the possible local orientations
of the sought-after traced boundaries
(which have normal~$\normalvec$).
With $\theta$ denoting the angle between~$\normalvec$ and~$\del T$,
the boundary condition~(\ref{eq:flux-boundary-condition}) may be rewritten as
\begin{equation}
  \cos\theta = \frac{F}{\norm{\del T}},
  \label{eq:flux-boundary-condition-cosine}
\end{equation}
and there are three cases at any given point:
\begin{enumerate}
  \item
    If~$\norm{\del T} > \abs{F}$,
    then~(\ref{eq:flux-boundary-condition-cosine})
    has a conjugate pair of solutions in~$\theta$,
    and there are two choices for~$\normalvec$,
    symmetric about~$\del T$,
    corresponding to two possible local orientations
    for the traced boundaries,
    which cross the local $T$-contour at an angle.
  \item
    If~$\norm{\del T} = \abs{F}$,
    then $\normalvec$~is either parallel~($\theta = 0$)
    or antiparallel~($\theta = \pi$) to~$\del T$,
    corresponding to traced boundaries which are tangential
    to the local $T$-contour.
  \item
    If~$\norm{\del T} < \abs{F}$,
    then the right hand side of~(\ref{eq:flux-boundary-condition-cosine})
    exceeds unity in magnitude,
    and there are no solutions in~$\theta$,
    i.e.~traced boundaries do not exist.
\end{enumerate}
With these observations in mind,
the domain of the known solution~$T$
is partitioned into
\begin{itemize}
  \item
    the \emph{viable domain},
    wherein~$\norm{\del T} \ge \abs{F}$
    and there are two possible branches of traced boundaries,
    and
  \item
    the \emph{non-viable domain},
    wherein~$\norm{\del T} < \abs{F}$
    and traced boundaries cannot exist.
\end{itemize}
The border between these two regions is
\begin{itemize}
  \item
    the \emph{terminal curve},
    along which~$\norm{\del T} = \abs{F}$
    and traced boundaries are tangential to the local $T$-contour.
\end{itemize}
A point along the terminal curve is called a \emph{terminal point},
and is one of the following:
\begin{enumerate}
  \item
    An \emph{ordinary} terminal point (Figure~\tbd):
    the local $T$-contour crosses the terminal curve at a non-zero angle.
    The traced boundaries (which are tangential to the local $T$-contour)
    terminate in a cusp at the terminal point,
    for they cannot enter the non-viable domain.
  \item
    A \emph{critical} terminal point:
    the local $T$-contour touches the terminal curve tangentially.
    This results in one of the following:
    \begin{enumerate}
      \item
        The \emph{hyperbolic} case (Figure~\tbd):
        the $T$-contour lies toward the viable side of the terminal curve.
        Two smooth traced boundaries pass through the terminal point,
        at which the traced boundaries,
        the local $T$-contour and the terminal curve
        all touch.
      \item
        The \emph{elliptic} case (Figure~\tbd):
        the $T$-contour lies toward the non-viable side of the terminal curve,
        and no smooth traced boundaries pass through the terminal point.
      \item
        The \emph{degenerate} case:
        the $T$-contour and the entire terminal curve
        are in fact the same curve.
        Thus the terminal curve consists solely of critical terminal points,
        and is therefore called a \emph{critical terminal curve}.
        This curve is itself a traced boundary,
        unto which other traced boundaries attach smoothly.
    \end{enumerate}
    The degenerate case occurs
    when the known solution~$T$ possesses sufficient symmetry.
    Anderson~\etal~\cite{anderson-2007-boundary-tracing-i-theory}
    note that they have not encountered any case
    where a non-discrete proper subset of the terminal curve
    coincides with a $T$-contour;
    neither have I encountered such a case.
\end{enumerate}

To determine the sought-after traced boundaries,
simply choose an appropriate coordinate system and parametrisation,
rewrite the flux condition~(\ref{eq:flux-boundary-condition})
as an ordinary differential equation (ODE) for the traced boundaries,
and integrate.
By patching together the traced boundaries which result,
new domains may be constructed
which also admit the solution~$T$ to the given BVP\@.
The only restriction on the manner of patching
is that the boundary normal~$\normalvec$ have consistent orientation
with respect to the flux condition~(\ref{eq:flux-boundary-condition}),
and it is here that
the distinction between ordinary and critical terminal points
becomes important.
Generally speaking one must avoid ordinary terminal points;
the cusp formed by the two traced boundaries
will not have consistent boundary orientation,
except possibly for vanishing or discontinuous~$F$.

Anderson~\etal~\cite{anderson-2007-boundary-tracing-i-theory}
have also derived results for the curvature of traced boundaries
and given a very neat analysis of boundary tracing
by mapping the curves onto the manifold~$z^2 = (\del T)^2 - F^2$.
While both of these are of utmost theoretical importance
(indeed a proper understanding of and classification system for
critical terminal points was a result of the manifold analysis),
they are not required for the boundary tracing work in this thesis.

To summarise, boundary tracing is a method in which
a known solution to a PDE is used
to generate new domains which admit the same solution
to the associated BVP
with flux boundary condition~(\ref{eq:flux-boundary-condition}).
It is fitting to conclude this overview with the observation that
boundary tracing is not a perturbative method.
No approximation is required
to convert the flux condition~(\ref{eq:flux-boundary-condition})
into an ODE for the traced boundaries,
and although numerical procedures may be required
to do the subsequent integration,
the underlying theory is exact.

\section{Thermal radiation}

\section{Capillary wedges}

\section{Thesis overview}
